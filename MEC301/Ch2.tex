\chapter{Torseurs}
Le torseur est un objet français (Cocorico !), permettant de représenter toutes les actions que subit un solide.\\
C'est un outils qui facilite les calculs et la formulation des lois.
En effet, il faudrait diviser les lois en plusieurs théorème pour complètement inclure les informations contenues dans un torseur.

Les torseurs sont un manière de d'écrire un champ de vecteur, ils sont tous caractérisés par trois paramètres : deux vecteurs et un point.
\section{Actions mécaniques}
\section{Moment d'une force}
\begin{defi}
On appelle moment en $A$ de la force $\overrightarrow{F}$ passant par le point $P$, du solide $\Sigma$ :
$$\overrightarrow{M_{A}(\overrightarrow{F})}=\overrightarrow{AP}\land\overrightarrow{A}$$
\end{defi}
Un moment représente la capacité d'une force a crée une rotation autour d'un axe.\\
On peut définir la formule de transport des moments qui permet, connaissant le moment en un point $A$ du solide, de calculer le moments de cette force sur n'importe quel point du solide.
\begin{prop}
Soient $A$ et $B$ deux points de l'espace et $\overrightarrow{M_{A}}(\overrightarrow{F})$, $\overrightarrow{M_{B}}(\overrightarrow{F})$ leurs moments associés de la force $\overrightarrow{F}$ appliquée en $P$ au solide $\Sigma$.\\
On définit la formule de transport des moments :
$$\overrightarrow{M_{B}}(\overrightarrow{F})=\overrightarrow{M_{A}}(\overrightarrow{F})+\overrightarrow{BA}\land\overrightarrow{F}$$
\end{prop}
\begin{demo}
Soit
\end{demo}
\begin{meth}[Calcul d'un moment avec le bras de levier]
Dans un problème a deux dimensions, nous pouvons calculer un moment sans passer par le produit vectoriel. Pour ce faire, on utilise le bras de levier.\\
On commence par trouver sa norme : $\|\overrightarrow{M_O}(\overrightarrow{F})\|=\|\overrightarrow{F}\|\times\|\overrightarrow{OA}\|$\\
Puis on détermine le vecteur unitaire par lequel il est porté ainsi que son signe :\\
Ici la force $\overrightarrow{F}$
\end{meth}
\section{Torseur force}
\begin{defi}
On appelle torseur force l'objet constitué de deux éléments de réductions :
\begin{itemize}
    \item la résultante des Forces appliquées au système, noté $R$
    \item la résultante des Moments appliqués au point considéré, noté $M_A$ avec $A$ un point du solide $\Sigma$
\end{itemize}
Le torseur force a donc trois paramètres : résultante des forces, résultante des moments et point d'application.
$$\{\tau\}_{A}=\begin{Bmatrix}\overrightarrow{R}\\\overrightarrow{M_A}\end{Bmatrix}_A=\begin{Bmatrix} \displaystyle\sum_{i=1}^{n}\overrightarrow{F_i}\\\displaystyle\sum_{i=1}^{n}\overrightarrow{AP_i}\land\overrightarrow{F_i}\end{Bmatrix}_A=\begin{Bmatrix} R_x & R_y  & R_z\\ M_x & M_y & M_z\end{Bmatrix}_A$$
\end{defi}
\begin{ex}
jfienf
\end{ex}
\begin{rmq}
Dans le cas d'un point matériel le torseur force se réduit seulement à la résultante des forces.
\end{rmq}
\subsection{Torseur de force répartie}

\begin{ex}
     Balle en l'air avec son poids une rotation et la pression de l'aire ou une mongolfière avec la poussée d'archimède et la pression de l'aire et son poids
\end{ex}
\section{Torseur cinématique}
\begin{defi}
On appelle torseur cinématique, l'objet constitué des éléments de réduction :
`\begin{itemize}
    \item 
\end{itemize}
$${\lbrace\chi\rbrace}_{A/R_0}=\begin{Bmatrix}\overrightarrow{\omega}_{\Sigma/R_0}\\\overrightarrow{V}_{\Sigma/R_0}(A)\end{Bmatrix}_{A/R_0}=\begin{Bmatrix} \omega_x & \omega_y  & \omega_z\\ u_A & v_A & w_A\end{Bmatrix}_{A/R_0}$$
\end{defi}
\section{Torseur de liaisons}
\section{Torseur déplacement infinitésimal}
\begin{defi}

$${\lbrace\delta\chi\rbrace}_{A/R_0}=\begin{Bmatrix}\delta\overrightarrow{\omega}_{\Sigma/R_0}\\\delta\overrightarrow{l}_{\Sigma/R_0}(A)\end{Bmatrix}_{A/R_0}=\begin{Bmatrix} \delta\omega_x & \delta\omega_y  & \delta\omega_z\\ \delta u_A & \delta v_A & \delta w_A\end{Bmatrix}_{A/R_0}$$
\end{defi}
\section{Opérations sur les torseurs}
On définit pour les torseurs les opérations de base permettant a ces dernier d'interagir entre-eux.
\begin{prop}
\begin{itemize}
    \item Addition :
    $$\{\tau_1\}_A+\{\tau_2\}_A=\begin{Bmatrix}\overrightarrow{R}_1\\\overrightarrow{M_{A1}}\end{Bmatrix}_A+\begin{Bmatrix}\overrightarrow{R}_2\\\overrightarrow{M_{A2}}\end{Bmatrix}_A=\begin{Bmatrix}\overrightarrow{R}_1+\overrightarrow{R}_2\\\overrightarrow{M_{A1}}+\overrightarrow{M_{A2}}\end{Bmatrix}_A$$
    \item Multiplication par un scalaire :
    $$\{\tau_2\}_A=\lambda\{\tau_1\}_A \Leftrightarrow \{\tau_2\}_A=\begin{Bmatrix}\overrightarrow{R_2}\\\overrightarrow{M_{A2}}\end{Bmatrix}_A=\begin{Bmatrix}\lambda\overrightarrow{R_1}\\\lambda\overrightarrow{M_{A1}}\end{Bmatrix}_A$$
    \item Produit :
    $$\lambda=\{\tau_1\}_A\cdot\{\tau_2\}_A \Leftrightarrow\lambda=\overrightarrow{R_1}\cdot\overrightarrow{M_{A2}}+\overrightarrow{R_2}\cdot\overrightarrow{M_{A1}}$$
\end{itemize}
\end{prop}