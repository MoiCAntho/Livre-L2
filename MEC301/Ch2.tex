\chapter{Torseurs}
Le torseur est un objet français (Cocorico !), permettant de représenter toutes les actions que subit un solide.\\
C'est un outils qui facilite les calculs et la formulation des lois.
En effet, il faudrait diviser les lois en plusieurs théorème pour complètement inclure les informations contenues dans un torseur.

Les torseurs sont un manière de d'écrire un champ de vecteur, ils sont tous caractérisés par trois paramètres : deux vecteurs et un point.

\begin{defi}
On appelle torseur, l'objet constitué de deux vecteurs appelé éléments de réductions ($\overrightarrow{A}$ et $\overrightarrow{B}$) et d'un point $P$.\\
L'élément de réduction $\overrightarrow{B}$ dépend du point $P$ ce qui n'est pas le cas de l'élément $\overrightarrow{A}$.\\
On note un torseur $T$, constitué de $\overrightarrow{A}$ et $\overrightarrow{B}$ au point $P$ comme il suit :
$$\{T\}_P=\begin{Bmatrix}\overrightarrow{A}\\\overrightarrow{B}\end{Bmatrix}_P$$
\end{defi}
Peu importe le type de torseur, l'élément dépendant du point se trouve toujours dans la partie basse de ce dernier.\\
Il peut être utile de pouvoir "déplacer" le torseur en un autre point, notamment pour les opérations.\\
Pour ce faire, on utilise la formule de transport.
\begin{prop}
Soit le torseur $\{T\}_P$ que l'on souhaite déplacer au point $Q$, le nouveau torseur $\{T\}_Q$ s'écrit :
$$\{T\}_Q=\begin{Bmatrix}\overrightarrow{A}\\\overrightarrow{B^\prime}\end{Bmatrix}$$
avec : $\overrightarrow{B^\prime}=\overrightarrow{B_Q}=\overrightarrow{B_P}+\overrightarrow{QP}\land\overrightarrow{A}$
\end{prop}
Cette propriété trouve son origine dans la géométrie du champ de moment.\\
\\
On peut maintenant définir les opérations usuelles sur les torseurs.
Elles permettent a ces dernier de pouvoir interagir entre-eux.
\begin{prop}
\begin{itemize}
    \item Addition :
    $$\{\tau_1\}_A+\{\tau_2\}_A=\begin{Bmatrix}\overrightarrow{R}_1\\\overrightarrow{M_{A1}}\end{Bmatrix}_A+\begin{Bmatrix}\overrightarrow{R}_2\\\overrightarrow{M_{A2}}\end{Bmatrix}_A=\begin{Bmatrix}\overrightarrow{R}_1+\overrightarrow{R}_2\\\overrightarrow{M_{A1}}+\overrightarrow{M_{A2}}\end{Bmatrix}_A$$
    \item Multiplication par un scalaire :
    $$\{\tau_2\}_A=\lambda\{\tau_1\}_A \Leftrightarrow \{\tau_2\}_A=\begin{Bmatrix}\overrightarrow{R_2}\\\overrightarrow{M_{A2}}\end{Bmatrix}_A=\begin{Bmatrix}\lambda\overrightarrow{R_1}\\\lambda\overrightarrow{M_{A1}}\end{Bmatrix}_A$$
    \item Produit :
    $$\lambda=\{\tau_1\}_A\cdot\{\tau_2\}_A \Leftrightarrow\lambda=\overrightarrow{R_1}\cdot\overrightarrow{M_{A2}}+\overrightarrow{R_2}\cdot\overrightarrow{M_{A1}}$$
\end{itemize}
\end{prop}
\section{Torseur force}
\begin{defi}
On appelle torseur force le torseur constitué des éléments de réductions suivants :
\begin{itemize}
    \item la résultante des Forces appliquées au système, noté $R$
    \item la résultante des Moments appliqués au point considéré, noté $M_A$ avec $A$ un point du solide $\Sigma$
\end{itemize}
Le torseur force a donc trois paramètres : résultante des forces, résultante des moments et point d'application.
$$\{\tau\}_{A}=\begin{Bmatrix}\overrightarrow{R}\\\overrightarrow{M_A}\end{Bmatrix}_A=\begin{Bmatrix} \displaystyle\sum_{i=1}^{n}\overrightarrow{F_i}\\\displaystyle\sum_{i=1}^{n}\overrightarrow{AP_i}\land\overrightarrow{F_i}\end{Bmatrix}_A=\begin{Bmatrix} R_x & R_y  & R_z\\ M_x & M_y & M_z\end{Bmatrix}_A$$
\end{defi}
\begin{ex}
jfienf
\end{ex}
\begin{rmq}
Dans le cas d'un point matériel le torseur force se réduit seulement à la résultante des forces.
\end{rmq}
\subsection{Torseur de force répartie}

\begin{ex}
     Balle en l'air avec son poids une rotation et la pression de l'aire ou une mongolfière avec la poussée d'archimède et la pression de l'aire et son poids
\end{ex}
\section{Torseur cinématique}
\begin{defi}
On appelle torseur cinématique, l'objet constitué des éléments de réduction :
`\begin{itemize}
    \item 
\end{itemize}
$${\lbrace\chi\rbrace}_{A/R_0}=\begin{Bmatrix}\overrightarrow{\omega}_{\Sigma/R_0}\\\overrightarrow{V}_{\Sigma/R_0}(A)\end{Bmatrix}_{A/R_0}=\begin{Bmatrix} \omega_x & \omega_y  & \omega_z\\ u_A & v_A & w_A\end{Bmatrix}_{A/R_0}$$
\end{defi}
\section{Torseur de liaisons}
\section{Torseur déplacement infinitésimal}
\begin{defi}

$${\lbrace\delta\chi\rbrace}_{A/R_0}=\begin{Bmatrix}\delta\overrightarrow{\omega}_{\Sigma/R_0}\\\delta\overrightarrow{l}_{\Sigma/R_0}(A)\end{Bmatrix}_{A/R_0}=\begin{Bmatrix} \delta\omega_x & \delta\omega_y  & \delta\omega_z\\ \delta u_A & \delta v_A & \delta w_A\end{Bmatrix}_{A/R_0}$$
\end{defi}
\section{Travail élémentaire et puissance}