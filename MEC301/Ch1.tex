\chapter{Notions élémentaires}
\section{Actions mécaniques}
\section{Moment d'une force}
\begin{defi}
On appelle moment en $A$ de la force $\overrightarrow{F}$ passant par le point $P$, du solide $\Sigma$ :
$$\overrightarrow{M_{A}(\overrightarrow{F})}=\overrightarrow{AP}\land\overrightarrow{A}$$
\end{defi}
Un moment représente la capacité d'une force a crée une rotation autour d'un axe.\\
On peut définir la formule de transport des moments qui permet, connaissant le moment en un point $A$ du solide, de calculer le moments de cette force sur n'importe quel point du solide.
\begin{prop}
Soient $A$ et $B$ deux points de l'espace et $\overrightarrow{M_{A}}(\overrightarrow{F})$, $\overrightarrow{M_{B}}(\overrightarrow{F})$ leurs moments associés de la force $\overrightarrow{F}$ appliquée en $P$ au solide $\Sigma$.\\
On définit la formule de transport des moments :
$$\overrightarrow{M_{B}}(\overrightarrow{F})=\overrightarrow{M_{A}}(\overrightarrow{F})+\overrightarrow{BA}\land\overrightarrow{F}$$
\end{prop}
\begin{demo}
Soit
\end{demo}
\begin{meth}[Calcul d'un moment avec le bras de levier]
Dans un problème a deux dimensions, nous pouvons calculer un moment sans passer par le produit vectoriel. Pour ce faire, on utilise le bras de levier.\\
On commence par trouver sa norme : $\|\overrightarrow{M_O}(\overrightarrow{F})\|=\|\overrightarrow{F}\|\times\|\overrightarrow{OA}\|$\\
Puis on détermine le vecteur unitaire par lequel il est porté ainsi que son signe :\\
Ici la force $\overrightarrow{F}$
\end{meth}