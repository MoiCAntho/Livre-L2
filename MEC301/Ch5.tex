\chapter{Cinétique du solide}
En cinématique, les mouvements des corps sont considérés en omettant l'inertie des ces derniers.
En réalité, les mouvements des systèmes sont liés aux causes d'une part et à leurs inertie d'autre part.\\
\\
La notion d'inertie caractérise la propriété du système de changer plus ou moins rapidement sa vitesse sous l'effet des forces qui lui sont appliquées.
On simplifie souvent les choses, en réduisant l'inertie à la masse. Or le mouvement d'un solide ne dépend pas que de sa masse et des forces exercées sur ce dernier.
En effet il dépend également de sa géométrie, de la distribution de sa masse, ...
\\
\\
L'étude de l'inertie s'effectue dans le cadre de la cinétique.
\section{Géométrie des masses}
\subsection{Notion de masse}
\begin{defi}
A chaque solide $\Sigma$ est associée un quantité (scalaire), noté $m$, qui représente la matière contenue dans ce dernier. On la définie comme il suit :
$$m=\int dm$$
avec $dm$ l'élément infinitésimal de masse
\end{defi}
Sachant cela il est possible de définit la masse à partir des distributions de masse :
\begin{itemize}
    \item Masse volumique $\rho$ (en $kg.m^{-1}$)
    \item Masse surfacique $\sigma$ (en $kg.m^{-2}$)
    \item Masse linéique $\lambda$ (en $kg.m^{-1}$)
\end{itemize}

Par définition de ces trois grandeurs, on peut écrire l'élément infinitésimal de masse $dm$ comme :
\begin{itemize}
    \item $dm=\rho(M)dV$
    \item $dm=\sigma(M)dS$
    \item $dm=\lambda(M)dl$
\end{itemize}
où $dV$, $dS$ et $dl$ sont respectivement les éléments infinitésimaux de volume, de surface et le déplacement élémentaire autour d'un point M.\\
 \\
En définitive, on peut donc écrire :
\begin{itemize}
    \item $m=\int \rho(M)dV$ $\rightarrow$ Très utilisée pour les solides en 3 dimensions
    \item $m=\int \sigma(M)dS$ $\rightarrow$ Utilisée pour les plaques
    \item $m=\int \lambda(M)dl$ $\rightarrow$ Utilisée pour les tiges
\end{itemize}
\begin{ex}
...
\end{ex}
\subsection{Centre de Gravité et référentiel barycentrique}
\section{Torseur cinétique}
\section{Matrice d'inertie}