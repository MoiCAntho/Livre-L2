\chapter{Cinématique du solide}
La définition des grandeurs cinématiques est exactement la même qu'en mécanique du point. C'est à dire que la vitesse est la dérivée de la position, et l'accélération celle de la vitesse. \\
On considère
\section{Expression des vitesses}
$$\overrightarrow{V_{/R_0}}(M)=\overrightarrow{V_a}(M)$$
Nous en déduisons la propriété de composition des vitesses.
\begin{thm}[Théorème : Composition des vitesses]
Soient le référentiel absolue $R_0$ (référentiel fixe) et les référentiels $R_1,\hdots,R_n$ en mouvement les uns par rapport aux autres. \\
On considère un point $M$ fixe dans le référentiel $R_n$.
La vitesse absolue, peut s'écrire :
$$\overrightarrow{V_a}=\overrightarrow{V_{R_n/R_0}}=\overrightarrow{V_{R_n/R_{n-1}}}+\hdots+\overrightarrow{V_{R_2/R_1}}+\overrightarrow{V_{R_1/R_0}}$$
On peut également écrire cette notion de manière torsorielle :
$$\{\chi_{R_n/R_0}\}=\{\chi_{R_n/R_{n-1}}\}$$
\end{thm}
\section{Expression des accélérations}
$$\overrightarrow{\gamma_a}(M)=\overrightarrow{\gamma_e}(M)+\overrightarrow{\gamma_r}(M)+\overrightarrow{\gamma_c}(M)$$
\begin{thm}[Théorème : Composition des accélérations]

\end{thm}
\section{Vitesse de glissement}