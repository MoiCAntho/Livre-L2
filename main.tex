\documentclass[a4paper, 12pt, openright, twoside, french]{book}
\usepackage[utf8]{inputenc}
\usepackage[left=2cm,right=2cm,top=3cm,bottom=3cm]{geometry}
\usepackage[french]{babel}
\usepackage{hyperref}
\usepackage{pst-blur}
\usepackage[pstricks]{bclogo} %% Pas utiliser pdfLaTeX pour le fonctionnement de bclogo
\usepackage{amsmath,amsfonts,amssymb}
\usepackage{fancyhdr}
\usepackage{mathrsfs}
\usepackage{tikz,tkz-tab}
\usepackage{float}
\usepackage{graphicx}
\usepackage{makeidx}
\usepackage{pstricks-add}
\usepackage{multicol}

\newenvironment{defi}{\begin{bclogo}[couleur=blue!30,couleurBord=blue,arrondi=0.1,logo=\bcbook,ombre=true]{Définition}}{\end{bclogo}}
\newenvironment{ex}{\begin{bclogo}[logo=\bccrayon,noborder=true,barre=snake]{Exemple}}{\end{bclogo}}
\newenvironment{prop}{\begin{bclogo}[couleur=red!30,couleurBord=red,ombre=true,arrondi=0.1,logo=\bcoutil]{Propriété}}{\end{bclogo}}
\newenvironment{thm}{\begin{bclogo}[couleur=green!30,couleurBord=green,logo=\bccle ,ombre=true,arrondi=0.1]{Théorème}}{\end{bclogo}}
\newenvironment{meth}{\begin{bclogo}[logo=\bclampe,arrondi=0.1,ombre=true, couleur=yellow!60,couleurBord=yellow]{Méthode : }}{\end{bclogo}}
\newenvironment{demo}{\begin{bclogo}[couleur=gray!20,couleurBord=gray,logo=\bcloupe ,arrondi=0.1, ombre=true, barre = zigzag]{Démonstration}}{\end{bclogo}}
\newenvironment{rmq}{\begin{bclogo}[couleur=orange!30,couleurBord=orange,arrondi=0.1,ombre=true,logo=\bcinfo]{Remarque}}{\end{bclogo}}


\makeatletter
\@addtoreset{chapter}{part}
\makeatother 

\title{Licence $2^{\text{ème}}$ année,\\Physique Mécanique\\ Semestre 1}
\author{Anthony LOUVAT-SEGURA, Vigrile CHEMINOT}
\date{November 2022}

\fancyhead{}
\fancyfoot{}
\fancyfoot[LE,RO]{\thepage}


\begin{document}

\maketitle
\tableofcontents
\part{Mat-304 : Calcul matriciel et fonctions de plusieurs variables}
\chapter{Repérage dans $\mathbb{R}^3$}
\chapter{Fonctions de plusieurs variables}
\chapter{Dérivation en plusieurs variables}
\section{Définition et premières propriétés}
 En dimension 1, la dérivée ne peut être approchée par uniquement 2 directions (par la gauche et par la droite).
 A partir de la dimension 2, il y a infinité de directions par lesquelles approchée la dérivée.\\
 \newline
 Schéma\\
On peut commencer par dérivée selon les vecteurs de bases de $\mathbb{R}^n$ ( les vecteurs $\overrightarrow{u_n}$), les vecteurs
\begin{defi}
    La dérivée partielle de $f:\mathbb{R}^n\to\mathbb{R}$ par rapport à $x_i$, $1\ge i\ge n$, en $\overrightarrow{a}\in\mathscr{D}_f$, noté $\frac{\partial f(\overrightarrow{a})}{\partial x_i}$ ou $f^{\prime}_{x_i}(\overrightarrow{a})$ est la valeur de la limite suivante quand elle existe :
    $$\frac{\partial f(\overrightarrow{a})}{\partial x_i}=\lim_{t\to 0}\frac{f(a_1,...,a_i+t,...,a_n)-f(a_1,...,a_n)}{t}$$
\end{defi}
Dans le cas d'une fonction défini comme $f:\mathbb{R}^2\to\mathbb{R}$, on a :\\
$$\frac{\partial f(x,y)}{\partial x} = \lim_{t\to 0}\frac{f(x+t,y)-f(x,y)}{t}$$
$$\frac{\partial f(x,y)}{\partial y} = \lim_{s\to 0}\frac{f(x+s,y)-f(x,y)}{s}$$
De manière pratique, on choisit la variable par rapport à laquelle on dérive l'expression, on considère toutes les autres constantes et on dérive "normalement".
\begin{ex}
\begin{enumerate}
    \item Soit $f(\overrightarrow{x})=3x_1^3x_3+2x_1x_2^2-5x_3^4$, on a :
    $$\frac{\partial f(\overrightarrow{x})}{\partial x_1} = 9x_1^2x_3+2x_2^2 $$
    $$\frac{\partial f(\overrightarrow{x})}{\partial x_2} = 4x_1x_2$$
    $$\frac{\partial f(\overrightarrow{x})}{\partial x_3} = 3x_1^3-20x_3^3$$
    \item Soit $f(x,y)=e^{xy^2}$, on a :
    $$\frac{\partial f(x,y)}{\partial x} = y^2e^{xy^2}$$
    $$\frac{\partial f(x,y)}{\partial y} = 2xye^{xy^2}$$
\end{enumerate}
\end{ex}
Il possible de dérivée selon une direction quelconque, pour cela on définit la dérivée directionnelle.
\begin{defi}
    Soient $f:\mathbb{R}^n\to\mathbb{R}$, $\overrightarrow{a}\in\mathscr{D}_f$ et $\overrightarrow{h}\in\mathbb{R}^n$.\\
    La dérivée directionnelle de $f$ en $\overrightarrow{a}$ suivant la direction $\overrightarrow{h}$ est la quantité défini par la limite suivante si elle existe:
    $$\frac{\partial f}{\partial\overrightarrow{h}}(\overrightarrow{a})=\lim_{\epsilon\to 0}\frac{f(\overrightarrow{a}+\epsilon\overrightarrow{h})-f(\overrightarrow{a})}{\epsilon}$$ 
    \newline
\end{defi}
\subsection{Gradient et matrice jacobienne}
On définit pour les fonction
\begin{defi}
Soit un champ de vecteur $f:\mathbb{R}^n\to\mathbb{R}^p$, on considère les dérivées partielles selon chacune des variables.\\
Selon la valeur de $p$, on créé deux objets :
\begin{itemize}
    \item Si  p=1, on définit le gradient de la fonction scalaire $f$ en $\overrightarrow{a}$ : $$\overrightarrow{\text{grad}}f(\overrightarrow{a})=\overrightarrow{\nabla} f(\overrightarrow{a})=\begin{pmatrix}\frac{\partial f}{\partial x_1}(\overrightarrow{a})\\\vdots\\\frac{\partial f}{\partial x_n}(\overrightarrow{a})\end{pmatrix}$$
    \item Si $p>1$, on défini la matrice jacobienne $J_f(\overrightarrow{a})$ (de dimension $n\times p$) de la fonction $f$ en $\overrightarrow{a}$ : $$J_f(\overrightarrow{a})=\begin{pmatrix}\frac{\partial f_1}{\partial x_1}(\overrightarrow{a})&\hdots&\frac{\partial f_1}{\partial x_n}(\overrightarrow{a})\\ \vdots & \ddots & \vdots\\ \frac{\partial f_p}{\partial x_1}(\overrightarrow{a}) & \hdots & \frac{\partial f_p}{\partial x_n}(\overrightarrow{a})\end{pmatrix}$$
\end{itemize}
\end{defi}
\section{Différentielle}
\section{Dérivées d'ordres supérieurs}
\chapter{Opérateurs différentielles}
Attention, l'expression des opérateurs changent selon le système de coordonnées.
\section{Le gradient $\protect\overrightarrow{\nabla}$}
Nous avons déjà définit le gradient d'une fonction évalué en un point, on va maintenant définir l'opérateur gradient.
\begin{defi}
On appelle gradient, l'opérateur définit comme il suit :
$$\overrightarrow{\nabla}=\begin{pmatrix}\frac{\partial}{\partial x_1}\\\vdots\\\frac{\partial}{\partial x_n}\end{pmatrix}$$
\newline
Le gradient d'une fonction scalaire $f:\mathbb{R}^n\to\mathbb{R}$ est donc :
$$\overrightarrow{\nabla}f=\begin{pmatrix}\frac{\partial}{\partial x_1}\\\vdots\\\frac{\partial}{\partial x_n}\end{pmatrix}\times f=\begin{pmatrix}\frac{\partial f}{\partial x_1}\\\vdots\\\frac{\partial f}{\partial x_n}\end{pmatrix}=\overrightarrow{grad(\overrightarrow{f})}$$
\end{defi}
\begin{prop}
Soit un champ scalaire $\overrightarrow{f}$ dans l'espace $\mathbb{R}^3$, nous avons pour expression du gradient :
\begin{itemize}
    \item En coordonnées cartésiennes (par définition) : $$\overrightarrow{\nabla}f=\frac{\partial f}{\partial x}\overrightarrow{u_x}+\frac{\partial f}{\partial y}\overrightarrow{u_y}+\frac{\partial f}{\partial z}\overrightarrow{u_z}$$
    \item En coordonnées cylindriques : $$\overrightarrow{\nabla}f=\frac{1}{r}\frac{\partial f}{\partial r}\overrightarrow{u_r}+\frac{\partial f}{\partial \theta}\overrightarrow{u_\theta}+\frac{\partial f}{\partial z}\overrightarrow{u_z}$$
    \item En coordonnées sphériques : $$\overrightarrow{\nabla}f=\frac{\partial r}{\partial x}\overrightarrow{u_r}+\frac{1}{r}\frac{\partial f}{\partial \theta}\overrightarrow{u_\theta}+\frac{1}{r\sin{\theta}}\frac{\partial f}{\partial \phi}\overrightarrow{u_\phi}$$
\end{itemize}
\end{prop}
\begin{demo}
soit
\end{demo}
\section{La divergence $div$}
\begin{defi}
Soit un champ de vecteurs $\overrightarrow{f}:\mathbb{R}^n\mapsto\mathbb{R}^n$ de classe $\mathscr{C}^1$ sur $\mathscr{D}_f$.\\
La divergence de $\overrightarrow{f}$ est donné par :
$$div(\overrightarrow{f})=\overrightarrow{\nabla}.\overrightarrow{f}=\sum_{i=1}^n \frac{\partial f_i}{\partial x_i}$$
\end{defi}
\begin{rmq}
La divergence est également égale à la trace de la matrice jacobienne.
$$div(\overrightarrow{f})=\text{Tr}(J_f)$$
\end{rmq}
\begin{prop}
Soit un champ scalaire $\overrightarrow{f}$ dans l'espace $\mathbb{R}^3$, nous avons pour expression de la divergence :
\begin{itemize}
    \item En coordonnées cartésiennes (par définition) : $$div(\overrightarrow{f})=\frac{\partial f_x}{\partial x}+\frac{\partial f_y}{\partial y}+\frac{\partial f_z}{\partial z}$$
    \item En coordonnées cylindriques : $$div(\overrightarrow{f})=\frac{1}{r}\frac{\partial (r f_r)}{\partial r}+\frac{1}{r}\frac{\partial f_{\theta}}{\partial \theta}+\frac{\partial f_z}{\partial z}$$
    \item En coordonnées sphériques : $$\overrightarrow{\nabla}f=\frac{\partial r}{\partial x}\overrightarrow{u_r}+\frac{1}{r}\frac{\partial f}{\partial \theta}\overrightarrow{u_\theta}+\frac{1}{r\sin{\theta}}\frac{\partial f}{\partial \phi}\overrightarrow{u_\phi}$$
\end{itemize}
\end{prop}
\begin{thm}[Théorème de Green-Ostrogradski]

\end{thm}
\section{Le rotationnel $rot$}
\begin{defi}
Soit un champ de vecteurs $\overrightarrow{f}:\mathbb{R}^3\mapsto\mathbb{R}^3$ de classe $\mathscr{C}^1$ sur $\mathscr{D}_f$, de coordonnées\\ $f=(f_1,f_2,f_3)$.\\
Le rotationnel de $\overrightarrow{f}$ de ce champ est défini par :

$$rot(\overrightarrow{f})=\overrightarrow{\nabla}\land\overrightarrow{f}=\begin{pmatrix}\frac{\partial}{\partial x}\\\frac{\partial}{\partial y}\\\frac{\partial}{\partial z}\end{pmatrix}\land\begin{pmatrix}f_1\\f_2\\f_3\end{pmatrix}=\begin{pmatrix}\frac{\partial f_3}{\partial y}-\frac{\partial f_2}{\partial z}\\\frac{\partial f_1}{\partial z}-\frac{\partial f_3}{\partial x}\\\frac{\partial f_2}{\partial x}-\frac{\partial f_1}{\partial y}\end{pmatrix}$$
\end{defi}

\section{Le laplacien $\Delta$}
\begin{defi}
Soit une champ scalaire $f:\mathbb{R}^n\mapsto\mathbb{R}$ de classe $\mathscr{C}^2$ sur $\mathscr{D}_f$.
Le laplacien, noté $\Delta$ $\nabla^2$, est donné par :
$$\Delta \overrightarrow{f}=\nabla . \overrightarrow{\nabla} f = div(\overrightarrow{grad}f)=\sum_{i=1}^n \frac{\partial^2 f}{\partial {x_i}^2}$$
\end{defi}
\section{Équations aux dérivées partielles}
\chapter{Intégration en plusieurs variables}
\chapter{Calcul matriciel}
\begin{defi}
Une matrice $n\times m$ à  coefficients réels est un tableau de nombres réels de $n$ lignes et $m$ colonnes.\\
On note $a_{ij}$ le coefficient à la i-ème  et à la j-ème colonne.\\
On représente une matrice de la manière suivante :
$$A = \begin{pmatrix}a_{11} & a_{12} & \hdots & a_{1m}\\a_{21} & a_{22} & \hdots & a_{2m}\\ \vdots & \vdots & \ddots & \vdots\\ a_{n1} & a_{n2} & \hdots & a_{nm} \end{pmatrix}$$
\newline
Une matrice $n\times m$ à valeurs réelles appartient à l'ensemble des matrices $n\times m$, noté $\mathbb{M}^{n\times m}(\mathbb{R})$.
\end{defi}
Toute application linéaire ou système d'équation linéaire peut être écrit sous forme matricielle.
\begin{ex}
ex de base
On considère l'application
\end{ex}
\section{Calcul matriciel}
On commence par définir l'addition de deux matrices.
\begin{defi}
Soient deux matrices $A$ et $B$ de dimension $n\times m$.\\
L'addition de ces deux matrices est donnée par :
$$\begin{pmatrix}a_{11} & a_{12} & \hdots & a_{1m}\\a_{21} & a_{22} & \hdots & a_{2m}\\ \vdots & \vdots & \ddots & \vdots\\ a_{n1} & a_{n2} & \hdots & a_{nm} \end{pmatrix}+\begin{pmatrix}b_{11} & b_{12} & \hdots & b_{1m}\\b_{21} & b_{22} & \hdots & b_{2m}\\ \vdots & \vdots & \ddots & \vdots\\ b_{n1} & b_{n2} & \hdots & b_{nm} \end{pmatrix}=\begin{pmatrix}a_{11}+b_{11} & a_{12}+b_{12} & \hdots & a_{1m}+b_{1m}\\a_{21}+b_{21} & a_{22}+b_{22} & \hdots & a_{2m}+b_{2m}\\ \vdots & \vdots & \ddots & \vdots\\ a_{n1}+b_{n1} & a_{n2}+b_{n2} & \hdots & a_{nm}+b_{nm} \end{pmatrix}$$
\end{defi}
\begin{defi}
Soient deux matrices $A$ de dimension $n\times m$ et $B$ de dimension $m\times p$.\\
La multiplication de ces deux matrices nous donnera une matrice $C$ de dimension $n\times p$.\\
Le calcul des coefficients de la matrice $C$ est donné par la formule :
$$c_{ij}=\sum_{k=1}^m a_{ik}\times b_{kj}=a_{i1}b_{1j}+a_{i2}b_{2j}+\hdots+a_{in}b_{nj}$$
\end{defi}
Attention, la multiplication matricielle n'est pas commutative
\begin{ex}
On considère les matrices $A=$ et $B=$
\end{ex}
\begin{defi}
Soit une matrice $A$ de dimension $n\times m$.\\
L'opération de transposition de cette matrice nous renverra la matrice transposée de $A$, $B$ de dimension $m\times n$.\\
Cet opération est défini comme il suit :
$$A^\text{T}$$
\end{defi}
\begin{defi}
Soit une matrice $A$ de dimension $n\times m$.\\
On appelle trace de la matrice $A$ la somme de tous ses termes diagonaux.
$$\text{Tr}(A)=\sum_{i=1}^n a_{ii}$$
\end{defi}
\begin{prop}
Soient $A$ et $B$ de matrice et $\alpha\in\mathbb{R}$.\\
\begin{itemize}
    \item $\text{Tr}(A+B)=\text{Tr}(A)+\text{Tr}(B)$
    \item $\text{Tr}(\alpha A)=\alpha\text{Tr}(A)$
    \item $\text{Tr}(A^\text{T})=\text{Tr}(A)$
    \item Si $A$ et $B$ sont multipliable : $\text{Tr}(AB)=\text{Tr}(BA)$
    \item Si $A$ et $B$ sont semblables : $\text{Tr}(A)=\text{Tr}(B)$
\end{itemize}
\end{prop}
\subsection{Multiplication matricielle}
\subsection{Transposition matricielle}
\subsection{Propriétés et caractéristiques de matrices}
\section{Matrices particulières}
\subsection{Matrice carrée et rectangulaire}
\subsection{Matrice élémentaire et algorithme de Gaus-Jordan}
\subsection{Matrices de passage}
\subsection{Matrices triangulaires}
\section{Déterminants et inversion de matrices}
\subsection{Comatrice et matrice adjointe}
\chapter{Diagonalisation}
  La diagonalisation est le second "problème principal" d'algèbre, le premier étant la résolution de systèmes
  linéaires.
Diagonaliser une matrice revient à la "simplifier".\\
L'intérêt d'un tel procédé est qu'il simplifie certains calculs tel que la multiplication ou l'exponentiation.\\
La diagonalisation consiste à chercher une base $\mathscr{B}$ de l'espace, dans laquelle la matrice $A$ est diagonale.\\
Dans la suite de ce chapitre nous ne considérerons que des matrices carrées.\\
\begin{bclogo}[couleur=blue!30,couleurBord=blue,arrondi=0.1,logo=\bcbook,ombre=true]{Définition}
Une application linéaire $f:\mathbb{R}^{n}\rightarrow\mathbb{R}^{n}$  est dite diagonalisable si et seulement si $\exists \mathscr{B}$ une base de $\mathbb{R}^{n}$ tel que sa matrice représentative $A_{\mathscr{B},\mathscr{B}}(f)$ est diagonale.
\end{bclogo}
\section{Éléments propres}
Il convient dans un premier temps de définir les différents objets qui servirons à la diagonnalisation, ces objets sont appelés éléments propres.
\begin{bclogo}[couleur=blue!30,couleurBord=blue,arrondi=0.1,logo=\bcbook,ombre=true]{Définition}
Soit A une matrice $n\times n$.
\begin{itemize}
    \item[$\bullet$] On dit que $\lambda\in\mathbb{C}$ est une valeur propre de $A$ s'il existe $x\in\mathbb{C}^{n}$ avec $x\neq 0$ tel que $Ax = \lambda x$.
    \item[$\bullet$] On appelle alors le vecteur $x$ le vecteur propre de $A$ associé à la valeur propre $\lambda$.
    \item[$\bullet$] On appelle spectre de $A$ l'ensemble des valeurs propres de $A$.
    \item[$\bullet$] On appelle sous espace propre de $A$ (associé à la valeur propre $\lambda$), noté $E_{\lambda}$, l'ensemble de tous les vecteurs $x$ tel que
$Ax=\lambda x \Leftrightarrow (A-\lambda I_n)x=0$.\\
Autrement dit, $E_{\lambda}=\ker(A-\lambda I_n)=\{x\in E | Ax=\lambda x\}$\\
\end{itemize}
\end{bclogo}
\begin{bclogo}[logo=\bccrayon,noborder=true,barre=snake]{Exemple}
Soit la matrice $A=\begin{pmatrix}
5 & 2 \\
4 & 3
\end{pmatrix}$\\
Par définition d'une valeur propre $\lambda$, nous cherchons un vecteur $x$ tel que $Ax=\lambda x$.
Dans le cas présent, on a :\\
$$\begin{pmatrix} 5 & 2\\ 4 & 3\end{pmatrix}\times\begin{pmatrix}x_1\\x_2\end{pmatrix}=\lambda\times\begin{pmatrix}x_1\\x_2\end{pmatrix}$$\\
Il nous faut donc résoudre le système suivant :\\
$$\begin{cases}5x_1+2x_2=\lambda x_1\\4x_1 + 3x_2=\lambda x_2\end{cases}$$
Sa résolution nous renvoie notamment le vecteur $x=(7,7)$ et il se trouve que $Ax=7\times x$.\\
On a donc que $\lambda = 7$ une valeur propre de la matrice $A$ et $x=(1,1)$ est vecteur un propre de la matrice $A$ associé à la valeur propre $\lambda = 7$.\\
\end{bclogo}
L'utilisation de la définition d'une valeur propre pour son calcul est une opération assez fastidieuse, c'est pour cela que l'on passe par d'autres moyens pour les déterminer.\\
On utilise pour cela le polynôme caractéristique de la matrice $A$.
\section{Polynôme caractéristique et calcul des éléments propres}
Le calcul des éléments propres est plus facile en passant par le polynôme caractéristique.
\begin{bclogo}[couleur=blue!30,couleurBord=blue,arrondi=0.1,logo=\bcbook,ombre=true]{Définition}
Soit une application linéaire $f:\mathbb{R}^n\to\mathbb{R}^n$ et sa matrice représentative dans la base $\mathscr{B}$ $A_{\mathscr{B}}(f)$.
On appelle polynôme caractéristique de l'application $f$, le polynôme défini de la façon suivante :
$$P_{f}(\lambda)=P_{A}(\lambda)=\det(A-\lambda I_{n})=\begin{vmatrix}
a_{11}-\lambda & a_{12} & \hdots & a_{1n}\\
a_{21} & a_{22}-\lambda &&\vdots \\
\vdots & & \ddots & \vdots\\
a_{n1} &\hdots&\hdots& a_{nn}-\lambda
\end{vmatrix}$$
\\
\end{bclogo}

\begin{bclogo}[couleur=green!30,couleurBord=green,logo=\bccle ,ombre=true,arrondi=0.1]{Proposition}
Les valeurs propres $\lambda$ de $A$ sont les racines du polynôme caractéristique.
$$\lambda \text{ est une valeur propre de la matrice } A \Leftrightarrow P_A(\lambda) = 0$$
\end{bclogo}

\begin{bclogo}[logo=\bccrayon,noborder=true,barre=snake]{Exemple}
Soit la matrice $A$ définie comme $A=\begin{pmatrix}
7 & 4 \\
3 & 6 \\
\end{pmatrix}$.\\
On commence par chercher les valeurs propres de $A$.
Par la proposition précédente, on a :\\
$$P_A(\lambda) = | A-\lambda I_3 | =\begin{vmatrix}
7 - \lambda & 4\\
3 & 6- \lambda \\
\end{vmatrix} = (7-\lambda)(6-\lambda) -12$$\\
On cherche donc les racines de $P_A(\lambda)$.
\begin{align*}
   & P_A(\lambda)=0\\
\Leftrightarrow & {\lambda}^{2}-13\lambda +30 =0\\
\Leftrightarrow & {\lambda}_{1}=3 \text{ et } {\lambda}_{2}=10 
\end{align*}

Les valeurs propres de la matrice $A$ sont donc ${\lambda}_{1}=3 \text{ et } {\lambda}_{2}=10$
\end{bclogo}
On constate, assez aisément, que la détermination des valeurs propres à l'aide du polynôme caractéristique est beaucoup plus facile et rapide.
\begin{bclogo}[couleur=red!30,couleurBord=red,ombre=true,arrondi=0.1,logo=\bcoutil]{Propriétés}
Soit la matrice $A$ et son polynôme caractéristique $P_A(\lambda)$.
\begin{itemize}
\item Si l'on injecte $0$ dans le polynôme caractéristique il nous renverra la valeur du déterminant de cette matrice :\
$$P_A(0)=\det(A)$$
\item Le polynôme caractéristique possède $n$ racines dans l'ensemble $\mathbb{C}$
\item Deux matrices semblables ont le même polynôme caractéristique.
\item Le polynôme caractéristique d'une matrice est égale à celui de sa transposée $P_A(\lambda)=P_{A^\text{T}}(\lambda)$.
\end{itemize}
\end{bclogo}

Le polynôme caractéristique nous donne un moyen simple de déterminer les valeurs propres.\\
De ces valeurs propres, on peut déduire le reste des éléments propres de la matrice.\\
\\
Afin de pouvoir continuer sereinement, nous allons introduire les multiplicités algébriques et géométriques, qui seront utiles pour la suite.

\begin{bclogo}[couleur=blue!30,couleurBord=blue,arrondi=0.1,logo=\bcbook,ombre=true]{Définition}
\begin{itemize}
    \item On appelle multiplicité géométrique d'une valeur propre $\lambda$ : la dimension du sous espace propre associé à la valeur propre $\lambda$.
    \item On appelle multiplicité algébrique d'une valeur propre $\lambda$ : la multiplicité de $\lambda$ en tant que racine du polynôme caractéristique.
\end{itemize}
\end{bclogo}
\begin{prop}
Soit $A\in\mathbb{M}^{n\times n}$, la trace de $A$ est égale à la somme des valeurs propres multipliées avec leur multiplicité propre.
$$\text{Tr}(A)=\sum_{i=1}^n \lambda_i\times a_i$$
\end{prop}
Cette propriété permet une première vérification de la validité des éléments propres.
\begin{bclogo}[logo=\bclampe,arrondi=0.1,ombre=true, couleur=yellow!60,couleurBord=yellow]{Méthode : Détermination des éléments propres}
\begin{enumerate}
    \item Déterminer le polynôme caractéristique
    \item Trouver les valeurs propres de $A$, en déduire le spectre de $A$
    \item Rechercher les vecteurs propres associés au valeurs propres $\lambda$
    \item Déterminer les sous espaces propres $E_{\lambda}$ associés aux valeurs propres $\lambda$.\\
    Pour ce faire, il suffit de trouver le noyau de la matrice $A-\lambda I_n$ $\Leftrightarrow \ker(A-\lambda I_n)$.\\
\end{enumerate}
\end{bclogo}
\begin{ex}
On se donne la matrice $A=\begin{pmatrix}3&-1&1\\0&2&0\\1&-1&3\end{pmatrix}$\\
On commence par poser $A-\lambda I_3$ :
$$A-\lambda I_n = \begin{pmatrix}3&-1&1\\0&2&0\\1&-1&3\end{pmatrix}-\begin{pmatrix}\lambda&0&0\\0&\lambda&0\\0&0&\lambda\end{pmatrix}=\begin{pmatrix}3-\lambda&-1&1\\0&2-\lambda&0\\1&-1&3-\lambda\end{pmatrix}=B$$
Le polynôme caractéristique nous est donné par le déterminant de cette nouvelle matrice $B$.
$$P_A(\lambda)=\det(A-\lambda I_3)=\det(B)=\begin{vmatrix}3-\lambda&-1&1\\0&2-\lambda&0\\1&-1&3-\lambda\end{vmatrix}$$
On choisi de développer le déterminant selon la deuxième ligne, en effet celui-ci nous serra plus facile a calculer.  On a donc :
$$P_A(\lambda)=\det(A-\lambda I_3)=(2-\lambda)((3-\lambda)^2-1)$$
On sait que les racines du polynôme caractéristique, sont les valeurs propres de $A$, donc :
\begin{align*}
    \Leftrightarrow & P_A(\lambda) = 0\\
    \Leftrightarrow & (2-\lambda)((3-\lambda)^2-1) = 0\\
    \Leftrightarrow & (2-\lambda)(\lambda^2-6\lambda+8) = 0
\end{align*}
Le premier facteur nous renvoie $\lambda = 2$ et le deuxième facteur nous donne $\lambda_1=2$ et $\lambda_2=4$.\\
On a donc les valeurs propres de $A$ qui sont: $\lambda_1 =2$ valeur propre de multiplicité algébrique $2$, $\lambda_2=4$ valeur propre de $A$ de multiplicité algébrique $1$.\\
On en déduit le spectre de $A$ : $Spec(A)=\{2;4\}$
\\
\\
On souhaite maintenant déterminer les vecteurs propres $x$ associées aux valeurs propres $\lambda$.\\
Pour une valeur propre $\lambda$ donnée, la recherche du vecteur propre associée passe par la résolution de l'égalité $(A-\lambda I_3)x=0$.\\
Pour la valeur propre $\lambda_1 = 2$ on a $(A-2I_3)x=0$.On commence par poser $A-2I_3$ :
$$A-2I_3=\begin{pmatrix}3&-1&1\\0&2&0\\1&-1&3\end{pmatrix}-\begin{pmatrix}2&0&0\\0&2&0\\0&0&2\end{pmatrix}=\begin{pmatrix}1&-1&1\\0&0&0\\1&-1&1\end{pmatrix}=C$$
Il nous faut donc résoudre :
\begin{align*}
    \Leftrightarrow & (A-2I_3)x=0\\
    \Leftrightarrow & Cx=0\\
    \Leftrightarrow & \begin{pmatrix}1&-1&1\\0&0&0\\1&-1&1\end{pmatrix}\times\begin{pmatrix}x_1\\x_2\\x_3\end{pmatrix}=\overrightarrow{0}\\
    \Leftrightarrow & \begin{cases}x_1-x_2+x3=0\\0=0\\x_1-x_2+x_3=0\end{cases}\\
    \Leftrightarrow & x_1-x_2+x_3=0\\
    \Leftrightarrow & x_1=x_2-x_3
    \Rightarrow \begin{pmatrix}x_2-x_3\\x_2\\x_3\end{pmatrix}=x
\end{align*}
On peut décomposer le vecteur :
$$\begin{pmatrix}x_2-x_3\\x_2\\x_3\end{pmatrix}=\begin{pmatrix}x_2\\x_2\\0\end{pmatrix}+\begin{pmatrix}-x_3\\0\\x_3\end{pmatrix}=x_2\times\begin{pmatrix}1\\1\\0\end{pmatrix}+x_2\times\begin{pmatrix}-1\\0\\1\end{pmatrix}$$
Il y a donc deux vecteurs propres $v_1=\begin{pmatrix}1\\1\\0\end{pmatrix}$ et $v_2=\begin{pmatrix}-1\\0\\1\end{pmatrix}$ associées a la valeur propre $\lambda_1=2$.\\
\newline
Pour la valeur propre $\lambda_2=4$ on pose $(A-4I_3)x=0$.
On effectue un raisonnement analogue au précédent :
$$A-4 I_3=\begin{pmatrix}3&-1&1\\0&2&0\\1&-1&-1\end{pmatrix}-\begin{pmatrix}4&0&0\\0&4&0\\0&0&4\end{pmatrix}=\begin{pmatrix}-1&-1&1\\0&-2&0\\1&-1&-1\end{pmatrix}=D$$
On cherche $\ker(A-4I_3)$ :
\begin{align*}
    \Leftrightarrow & \ker(A-4I_3)\\
    \Leftrightarrow & \ker D\\
    \Leftrightarrow & Dx=0\\
    \Leftrightarrow & \begin{pmatrix}-1&-1&1\\0&-2&0\\1&-1&-1\end{pmatrix}\times\begin{pmatrix}x_1\\x_2\\x_3\end{pmatrix}=\begin{pmatrix}0\\0\\0\end{pmatrix}\\
    \Leftrightarrow & \begin{cases}-x_1-x_2+x_3=0\\-2x_2=0\\x_1-x_2-x_3=0\end{cases}\\
    \Leftrightarrow & \begin{cases}-x_1+x_3=0\\ x_2=0\\x_1-x_3\end{cases}\\
    \Leftrightarrow & x_1-x_3=0\\
    \Leftrightarrow & x_1=x_3 \Rightarrow \begin{pmatrix}x_1\\0\\x_1\end{pmatrix}
\end{align*}
Ici il y a un vecteur propre associé à la valeur propre $\lambda_2=4$ qui est $v_1=\begin{pmatrix}1\\0\\1\end{pmatrix}$.\\
\newline 
Et enfin, a partir des vecteurs propres on déduit les sous espaces propres, ici :
$$E_{\lambda_1}=\text{Vect}\begin{pmatrix}\begin{pmatrix}1\\1\\0\end{pmatrix},\begin{pmatrix}-1\\0\\1\end{pmatrix}\end{pmatrix}$$

$$E_{\lambda_2}=\text{Vect}\begin{pmatrix}\begin{pmatrix}1\\0\\1\end{pmatrix}\end{pmatrix}$$
\end{ex}


\section{Diagonalisation}
Revenons sur la définition d'une matrice diagonalisable, pour en donner une définition plus mathématique.
\begin{bclogo}[couleur=blue!30,couleurBord=blue,arrondi=0.1,logo=\bcbook,ombre=true]{Définition}
La matrice $A\in\mathbb{M}^{n\times n}$ est diagonalisable, s'il existe une matrice diagonale $D$ et une matrice $P$ inversible telles que $A=PDP^{-1}$, où $P$ est la matrice de passage entre la base canonique et la base $\mathscr{B}$ où la matrice $D$ existe.
\end{bclogo}
Une propriété qui viens directement a l'esprit est que les matrices $A$ et $D$ sont semblables (par définition).\\
Tout l'enjeu sera donc de déterminer la matrice de passage $P$.\\
Néanmoins, il peut être intéressant de savoir si une matrice est diagonalisable. 
\begin{thm}[Théorème: Critère de diagonalisation]
Soit $A\in\mathbb{M}^{n\times n}(\mathbb{R})$.
$\text{La matrice}\ A\ \text{est diagonalisable}\ \Leftrightarrow\forall i=1,\hdots,k\ \text{on a}\ \text{dim}(E_{\lambda_i})=\text{multiplicité algébrique de}\ \lambda_i$
\end{thm}
\begin{thm}[Corollaire]
Si $P_A(\lambda)$ admet $n$ racines réelles distinctes $\Leftrightarrow$ $A$ est diagonalisable
\end{thm}
Une fois le critère de diagonalisation ou son corollaire vérifié on peut déduire la matrice diagonale $D$ ainsi que le la matrice de passage $P$.
\begin{prop}
Soit la matrice $A\in\mathbb{M}^{n\times n}$ et ses valeurs propres $\lambda_1,\hdots, \lambda_k$ et leurs multiplicités $a_1,\hdots,a_k$ associées.
La matrice diagonale $D$ est donné par :\\
$$\begin{pmatrix}
\lambda_1 & 0 & \hdots & \hdots & \hdots & \hdots & \hdots & \hdots & \hdots & 0\\
0 & \ddots & 0 & & & & & & & \vdots\\
\vdots & 0 & \lambda_1 & 0 & & & & & & \vdots\\
\vdots & & 0 & \lambda_2 & 0 & & & & & \vdots \\
\vdots & & & 0 & \ddots & 0 & & & & \vdots\\
\vdots & & & & 0 & \lambda_2 & 0 & & & \vdots\\
\vdots & & & & & 0 & \ddots & 0 & & \vdots\\
\vdots & & & & & & 0 & \lambda_k & 0 & \vdots\\
\vdots & & & & & & & 0 & \ddots & 0 \\
0 & \hdots & \hdots & \hdots & \hdots & \hdots & \hdots & \hdots & 0 & \lambda_k
\end{pmatrix}$$
où les valeurs propres $\lambda_i$ se répète autant fois que la valeur de leur multiplicité $a_i$.
\end{prop}
Nous avons la matrice diagonale, maintenant il nous faut trouver la matrice de passage $P$.
\begin{prop}
Soit la matrice $A\in\mathbb{M}^{n\times n}$ et ses vecteurs propres $x_1,\hdots,x_n$.
La matrice de passage entre la base canonique $\mathscr{B}_c$ et la base où la matrice diagonale se trouve $\mathscr{B}_D$, est la matrice composée de tous les vecteurs propres de $A$.
$$P_{\mathscr{B}_c,\mathscr{B}_D}=\begin{pmatrix}x_1\ |& \hdots\ | & x_n\end{pmatrix}$$
\end{prop}

L'ordre des valeurs propres n'influe pas sur le résultat.\\
Il est néanmoins nécessaire de bien faire attention a garder l'ordre choisi sinon les calculs s'en trouveront erroné.
\begin{meth}[Diagonalisation d'une matrice]
\begin{enumerate}
    \item Déterminer le polynôme caractéristique
    \item Trouver les valeurs propres de $A$.
    \item Factoriser le polynôme caractéristique.
    \item Rechercher les sous espaces propres et leurs dimensions (multiplicité géométrique)
    \item Vérifier le critère de diagonalisation ou son corollaire.
    \item Trouver les bases $\mathscr{B}_i$ de tous les sous espaces propres.
    \item Déduire la matrice $P$ et $P^{-1}$
\end{enumerate}
\end{meth}
La diagonalisation est une notion qui nous aide à déterminer des extremums de fonctions (\textit{Cf.Chapitre 8 : Extremums en plusieurs variables}), permet d'analyser des relations de récurrences ou encore permet de résoudre des systèmes différentiels linéaires (\textit{Cf.Mat-307,Partie 2,Chapitre 2 : Systèmes différentiels}). 
\begin{ex}
On considère la suite de Fibonnacci définie de manière récurrente comme il suis : 
$$u_{t+1}=u_{t}+u_{t-1}\ \text{avec } u_0=u_1=1$$
On cherche a exprimée le terme générale de la suite (forme explicite).\\
Pour ce faire ...
\end{ex}
\section{Matrices symétriques et formes quadratiques}
\subsection{Éléments propres et diagonalisation d'une matrice symétrique}
On rappel la définition d'une matrice symétrique et d'une matrice orthogonale.
\begin{defi}
Soit $A\in\mathbb{M}^{n\times n}(\mathbb{R})$.\\
$$A\text{ est symétrique }\Leftrightarrow\ A=A^{\text{T}}$$
$$A\text{ est orthogonale }\Leftrightarrow\ AA^{\text{T}}=I_n\ \Leftrightarrow\ A^{-1}=A^{\text{T}}$$
\end{defi}
\begin{ex}
On considère $A=$ et $B=$
\end{ex}
\begin{thm}[Théorème des axes principaux]
Toute matrice symétrique $A$ peut être diagonalisée à l'aide d'une matrice orthogonale : il existe $\Gamma\in\mathbb{M}^{n\times n}(\mathbb{R})$ orthogonale et une matrice diagonale réelle $\Lambda$ telles que :
$$A=\Gamma\Lambda\Gamma^{-1}=\Gamma\Lambda\Gamma^{T}$$
\end{thm}
\begin{bclogo}[couleur=green!30,couleurBord=green,logo=\bccle ,ombre=true,arrondi=0.1]{Critère de Silvester}
Une matrice $A$ est semi-définie (définie) positive si et seulement si tous ses mineurs principales sont positifs, c'est à  dire :
$$a_{11}\ge 0, \begin{vmatrix} a_{11}&a_{12}\\ a_{21} & a_{22}\end{vmatrix}\ge 0,\begin{vmatrix}a_{11} & a_{12} & a_{13}\\ a_{21} & a_{22} & a_{23}\\ a_{31} & a_{32} & a_{33}\end{vmatrix}\ge 0, ...$$
Une matrice $A$ est semi-définie (définie) négative si et seulement si les signes des mineurs principaux de $A$ alternent :
$$a_{11}\leq 0, \begin{vmatrix} a_{11}&a_{12}\\ a_{21} & a_{22}\end{vmatrix}\leq 0,\begin{vmatrix}a_{11} & a_{12} & a_{13}\\ a_{21} & a_{22} & a_{23}\\ a_{31} & a_{32} & a_{33}\end{vmatrix}\leq 0, ...$$
Avec les $\leq$ ou $\ge$ sont respectivement remplacés pas $<$ et $>$ dans le cas des matrices définies
\end{bclogo}
\chapter{Extremums en plusieurs variables}
\newpage
\part{Mat-307 : Courbes paramétrées et équations différentielles pour la physique}
\part{Courbes}
\chapter{Courbes paramétrées}
La trajectoire d'un corps dans un plan est déterminé par le couple de coordonnées $(x,y)$ dépendant du temps $t$, c'est une équation paramétrique.
\begin{defi}
Soient $f$ et $g$ deux fonctions définies sur $I\subseteq
\mathbb{R}$.\\
Le point $M(t)$ de coordonnées $(f(t),g(t))$ décrit une courbe du plan $C$ appelée courbe paramétrée (de paramètre $t$).
La fonction de $I$ sur $(C)$ qui à $t$ associe $M(t)$ est un paramétrage de $(C)$.\\
Les équations $\begin{cases}x=f(t)\\y=g(t)\end{cases}$ définissent une représentation paramétrique de $\mathscr{C}$.\\
Notations : $(x=x(t),y=y(t))$ ou $t\mapsto(x(t),y(t))$
\end{defi}
Nous étudierons les propriétés des courbes paramétrées, qui peuvent être de deux natures :
\begin{itemize}
    \item Cinématique : dépendantes du paramètre $t$.\\
    ex : vitesse, accélération, ...
    \item Géométrique : indépendante du paramètre $t$.\\
    ex : tangentes, ...
\end{itemize}
\begin{rmq}
Par convention, on nomme le paramètre $t$ le temps, bien que ce dernier peut n'avoir aucun rapport avec ce dernier.\\
De même, les vecteurs correspondants aux dérivées première et seconde, sont appelés respectivement vitesse et accélération.
\end{rmq}
\section{Paramétrage et représentation graphique}
La paramétrisation d'une courbe n'est jamais unique et il est possible de passer d'un paramétrage a l'autre.
\section{Étude analytique d'une courbe paramétrée}
\subsection{Domaine de définition et intervalle d'étude}
\begin{defi}
Le domaine de définition $I$ du paramétrage est l'intersection des domaines de définition des fonctions $x(t)$ et $y(t)$.
$$\mathscr{D}_\mathscr{C}=\mathscr{D}_x\cap\mathscr{D}_y$$
\end{defi}
Une fois le domaine de définition déterminée, on cherche à réduire le domaine de définition à un intervalle d'étude afin de simplifier l'étude.
Pour ce faire, on utilise les propriétés de périodicité et de parités.
\begin{prop}
Soit la courbe $\mathscr{C}$ défini par $(x(t),y(t))$.On étudie la périodicité des deux coordonnées.\\
La coordonnée $x(t)$ est périodique si $x(t+T_1)=x(t)$ et la coordonnée $y(t)$ est périodique si $y(t+T_2)=y(t)$.\\
\newline
Si $T_1=T_2=T$ la période commune est $T$.\\
Si $T_1\neq T_2$ alors il faut déterminer la période commune $T$.\\
Pour ce faire, on a $T=\text{PPCM}(T_1,T_2)$.\\
La réduction de l'intervalle pour une courbe périodique de période $T$ est $[a,a+T]$ avec $a=0$ ou $a=\frac{T}{2}$.
\end{prop}
On rappelle que les fonctions sinus et cosinus sont $2\pi$ périodique, et la tangente est $\pi$ périodique.\\
Dans la plupart des cas, on fait en sorte que $0$ soit dans l'intervalle pour pouvoir exploiter les propriétés de symétries.
\begin{prop}
Soit la courbe $\mathscr{C}$ défini par $(x(t),y(t))$.\\
On étudie les propriétés de parités de chacune des coordonnées.
\begin{itemize}
    \item Si $x$ et $y$ sont impaires, pour tout $t$ la courbe est symétrique par rapport au centre $O$.
    \item Si $x$ est impaire et $y$ est paire, pour tout $t$ la courbe est symétrique par rapport à l'axe $(O_y)$.
    \item Si $x$ est paire et $y$ est impaire, pour tout $t$ la courbe est symétrique par rapport à l'axe $(O_x)$
    \item Si $x$ et $y$ sont paires, pour tout $t$ la courbe revient sur ces pas.
\end{itemize}
\end{prop}
\begin{ex}
On se donne un courbe $\mathscr{C}$ défini par  :
$$\begin{array}{cccc}
    \mathscr{C} \ : & \mathscr{D}_{\mathscr{C}} & \to & \mathbb{R} \\
         & t & \mapsto & \begin{cases}x(t)=\sin(\frac{3t}{2})\\y(t)=\sin(\frac{t}{3})\end{cases}
\end{array}$$
On recherche la période de chaque coordonnées :
\begin{align*}
    & x(t+T)\\
    \Leftrightarrow & \sin\left(\frac{3t}{2}+2\pi\right) \\
    \Leftrightarrow & \sin\left(3t+4\pi\right)\\
    \Leftrightarrow & \sin\left(t+\frac{4\pi}{3}\right)
\end{align*}
On déduit donc que la période de $x(t)$ est $T=\frac{4\pi}{3}$.
\begin{align*}
    & y(t+T)\\
    \Leftrightarrow & \sin\left(\frac{t}{3}+2\pi\right) \\
    \Leftrightarrow & \sin\left(t+6\pi\right)
\end{align*}
On déduit donc que la période de $y(t)$ est $T=6\pi$.\\
On cherche maintenant la période commune : en cherchant $PPCM(4,6)$.
On obtient une période commune $T=12\pi$
\end{ex}
\subsection{Étude des branches infinies}
Une fois l'intervalle d'étude
\begin{defi}
Soit une courbe $\mathscr{C}$ défini par ses coordonnées $(x(t),y(t))$ sur son domaine de définition $\mathscr{D}_\mathscr{C}$. On que dit que la courbe $\mathscr{C}$ possède une branche infinie si au moins l'une des quantités suivantes : $a$, $l$ ou $m$ tend vers l'infini.\\
$$\lim\limits_{t\to a}\mathscr{C}=\begin{cases}\lim\limits_{t\to a}x(t)=l\\\lim\limits_{t\to a}y(t)=m\end{cases}$$
\end{defi}
La définition ci-dessus, nous indique l'existence de branches infinies.\\
Mais il faut maintenant déterminer si la courbe $\mathscr{C}$ possède des asymptotes ou des branches paraboliques.\\
Cette information nous permet de tracer les courbes plus facilement.
\begin{prop}
Soit une courbe $\mathscr{C}$ défini par ses coordonnées $(x(t),y(t))$ sur son domaine de définition $\mathscr{D}_\mathscr{C}$.On détermine la nature des branches infinies ainsi que son équation, en suivant les critères ci-après :
\begin{itemize}
    \item Si $\lim\limits_{t\to a}x(t)=\pm\infty$ et $\lim\limits_{t\to a}y(t)=y_0$ avec $y_0\in\mathbb{R}$, la courbe admet une asymptote horizontale d'équation $y=y_0$.
    \item Si $\lim\limits_{t\to a}x(t)=x_0$ et $\lim\limits_{t\to a}y(t)=\pm\infty$ avec $x_0\in\mathbb{R}$, la courbe admet un asymptote verticale d'équation $x=x_0$
    \item Si $\lim\limits_{t\to a}x(t)=\pm\infty$ et $\lim\limits_{t\to a}y(t)=\pm\infty$, alors possible asymptote ou branche parabolique :
    \begin{itemize}
        \item Si $\lim\limits_{t\to a}\frac{y(t)}{x(t)}=\pm\infty$, la courbe admet une branche parabolique de direction asymptotique $O_y$
        \item Si $\lim\limits_{t\to a}\frac{y(t)}{x(t)}=0$, la courbe admet une branche parabolique de direction asymptotique $O_x$
        \item Si $\lim\limits_{t\to a}\frac{y(t)}{x(t)}=a$, il faut continuer l'étude :
        \begin{itemize}
            \item Si $\lim\limits_{t\to a} y(t)-ax(t)=\pm\infty$, la courbe admet une branche parabolique de direction asymptotique $y=ax$.
            \item Si $\lim\limits_{t\to a} y(t)-ax(t)=b$ avec $b\in\mathbb{R}$, la courbe admet une asymptote oblique d'équation $y=ax+b$
        \end{itemize}
    \end{itemize}
\end{itemize}
\end{prop}
SChéma et exemples
\subsection{Étude locale et points singuliers}
On commence dans un premier temps par  définir le vecteur $\overrightarrow{OM(t)}$ (autrement appelé $\overrightarrow{M(t)}$), qui est défini par $\begin{pmatrix}x(t)\\y(t)\end{pmatrix}$.\\
Afin d'amorcer une étude locale d'une courbe paramétrée il convient de dériver ce vecteur, on a donc : $\overrightarrow{M^{\prime}(t)}=\begin{pmatrix}x^{\prime}(t)\\y^{\prime}(t)\end{pmatrix}$, on obtient le vecteur vitesse.\\
Tout comme l'étude d'une fonction, tout ce qu'il de plus classique, l'essentiel de l'étude déroule au niveau des points d'annulations de ces dérivées.
\subsubsection{Tangentes}
\begin{prop}
Soit $\mathscr{C}$ une courbe paramétrée par ses coordonnées $x(t)$ et $y(t)$.\\
Nous considérons leurs dérivées, respectivement $x^{\prime}(t)$ et $y^{\prime}(t)$.
\begin{itemize}
    \item Si $x^{\prime}(t_0)=0$ et $y^{\prime}(t_0)\neq0$, la courbe admettra une tangente verticale en $t=t_0$.
    \item Si $x^{\prime}(t_0)\neq0$ et $y^{\prime}(t_0)=0$, la courbe admettra une tangente horizontale en $t=t_0$.
\end{itemize}
\end{prop}
On peut se convaincre assez facilement de la direction des tangentes à l'aide du petit raisonnement suivant.\\
En effet, si $x^{\prime}(t_0)=0$, la dérivée est entièrement portée par le vecteur unitaire $\overrightarrow{y}$, de ce fait la tangente ne peut être que verticale, et réciproquement.
\begin{ex}
On considère la courbe $\mathscr{C}$ définie par :
$$\begin{array}{cccc}
    \mathscr{C} \ : & \mathscr{D}_{\mathscr{C}} & \to & \mathbb{R} \\
         & t & \mapsto & \begin{cases}x(t)=\frac{4t^2-1}{t^3+1}\\y(t)=\frac{4t^3-t}{t^3+1}\end{cases}
\end{array}$$
\begin{minipage}{0.6\linewidth}
On dérive la composante $x(t)$ :
$$x^{\prime}(t)=\frac{x(-4x^3+3x+8)}{(t^3+1)^2}$$
On résout l'équation $x^{\prime}(t)=0$.\\
On obtient $S=\{0;1,46\}$\\
La courbe admet deux tangentes verticales.\\
\\
On dérive la composante $y(t)$ :
$$y^{\prime}(t)=\frac{2t^3+12t^2-1}{(t^3+1)^2}$$
On résout l'équation $y^{\prime}(t)=0$.\\
On obtient  $S=\left\{-5,99 ; -0,3; 0,28 \right\}$ .\\
La courbe admet trois tangentes horizontales.

\end{minipage}
\begin{minipage}{0.4\linewidth}
\begin{pspicture*}(-3, -3)(3,5)
\psgrid[subgriddiv=0,griddots=10,gridlabels=7pt,gridcolor=gray]
\parametricplot[plotstyle=curve,algebraic,linecolor=red]{-5}{5}{(4*t^2-1)/(t^3+1) | (4*t^3-t)/(t^3+1) }
\psaxes[ticks=none,labels=none]{->}(0,0)(-3,-3)(3,5)
\end{pspicture*}
\end{minipage}
\end{ex}

\subsubsection{Points réguliers et singuliers}
\begin{defi}
Soit $\mathscr{C}$ une courbe paramétrée par ses coordonnées $x(t)$ et $y(t)$.
Nous considérons le vecteur vitesse $\overrightarrow{v(t)}$ du point $M(t)$.
\begin{itemize}
    \item Si $v(t_0)\neq\overrightarrow{0}$ alors la courbe admet en $t=t_0$ un point régulier.\\
    \item Si $v(t_0)=\overrightarrow{0}$ alors la courbe admet en $t=t_0$ un point singulier.
\end{itemize}
\end{defi}
Il est nécessaire de bien comprendre ce que sont ces deux points.
Un point régulier est "un point normal de la courbe", c'est à dire que la courbe est tangente au vecteur vitesse $\overrightarrow{v(t)}$.\\
Un point singuliers quant à lui est un point très particulier qui peut être de plusieurs natures.
Néanmoins, son étude locale est, par définition, impossible en se cantonnant uniquement au vecteur vitesse.
\\
\\
Pour remédier à cela, on va faire un développement limité en $t_0$, au minimum à l'ordre 3.
On rappel la formule de Taylor, pour une fonction $f$ en un point $x_0$ à l'ordre $n$ : $$P_n(x)=\sum_{i=1}^{n}f^{(i)}(x_0)\frac{(x-x_0)^{i}}{i!}$$
On calcul les développements limités des deux coordonnées.
$$x(t)=x(t_0)+v_x(t_0)(t-t_0)+x^{\prime\prime}(t_0)\frac{(t-t_0)^2}{2!}+x^{(3)}(t_0)\frac{(t-t_0)^3}{3!}+...+o((t-t_0)^n)$$
et
$$y(t)=y(t_0)+v_y(t_0)(t-t_0)+y^{\prime\prime}(t_0)\frac{(t-t_0)^2}{2!}+y^{(3)}(t_0)\frac{(t-t_0)^3}{3!}+...+o((t-t_0)^n)$$
Puis on rassemble ces deux développement limités en un vecteur :
$$\overrightarrow{M(t)}=\begin{pmatrix}x(t_0)\\y(t_0\end{pmatrix}+\begin{pmatrix}v_x(t_0)\\v_y(t_0)\end{pmatrix}(t-t_0)+\begin{pmatrix}x^{\prime\prime}(t_0)\\y^{\prime\prime}(t_0)\end{pmatrix}\frac{(t-t_0)^2}{2}+\begin{pmatrix}x^{(3)}(t_0)\\y^{(3)}(t_0)\end{pmatrix}\frac{(t-t_0)^3}{6}+...+o(\|\overrightarrow{M(t)}\|^n)$$
Or par définition d'un point singulier ($\overrightarrow{v(t_0)}=\overrightarrow{0}$), on a :
$$\overrightarrow{M(t)}=\begin{pmatrix}x(t_0)\\y(t_0)\end{pmatrix}+\begin{pmatrix}x^{\prime\prime}(t_0)\\y^{\prime\prime}(t_0)\end{pmatrix}\frac{(t-t_0)^2}{2}+\begin{pmatrix}x^{(3)}(t_0)\\y^{(3)}(t_0)\end{pmatrix}\frac{(t-t_0)^3}{6}+...+o(\|\overrightarrow{M(t)}\|^n)$$
Une fois cela fait nous cherchons les deux premiers vecteurs (associées a des degrés supérieurs à $2$ dans le développement limité), non colinéaires.
Ces deux vecteurs nous permettrons de définir la nature du point singulier, ainsi que le sens de parcours de la courbe à travers celui-ci.

\begin{prop}
Soit une courbe $\mathscr{C}$ ayant un point singulier au point $t=t_0$.\\
Le développement limité de la courbe $\mathscr{C}$ au point $t=t_0$ à l'ordre $n$ est le suivant :
$$\overrightarrow{M(t)}=\begin{pmatrix}x(t_0)\\y(t_0)\end{pmatrix}+...+\underbrace{\begin{pmatrix}x^{(p)}(t_0)\\y^{(p)}(t_0)\end{pmatrix}}_{\overrightarrow{l}}\frac{(t-t_0)^p}{p!}+...+\underbrace{\begin{pmatrix}x^{(q)}(t_0)\\y^{(q)}(t_0)\end{pmatrix}}_{\overrightarrow{m}}\frac{(t-t_0)^q}{q!}+...+o(\|\overrightarrow{M(t)}\|^n)$$
Avec les vecteurs $\overrightarrow{l}$ et $\overrightarrow{m}$, les deux premiers vecteurs non colinéaires.\\
La nature du point singulier est donnée par les critères suivants :
\begin{itemize}
    \item Si $p$ est impair et $q$ est pair, alors il s'agit d'un point régulier.
    \item Si $p$ est impair et $q$ est impair, alors il s'agit d'un point d'inflexion.
    \item Si $p$ est pair et $q$ est impair, alors il s'agit d'un point de rebroussement de $1^{\text{ère}}$ espèce.
    \item Si $p$ est pair et $q$ est pair, alors il s'agit d'un point de rebroussement de $2^{\text{nde}}$ espèce.
\end{itemize}
\end{prop}
propriétés sur le sens de parcours demander à la prof

Images à faires
\begin{ex}
...
\end{ex}

\subsubsection{Convexité}
Dans le cas d'un point d'inflexion, il peut-être utile de chercher si avant et après lui la courbe est convexe ou concave.
\subsection{Tableau de variation}
\subsection{Applications}
\begin{meth}
Soit une courbe $\mathscr{C}$, voici le déroulement de son étude :
\begin{enumerate}
    \item Détermination de son ensemble de définition
    \item Étude de la périodicité et des symétries pour un éventuelle réduction de l'intervalle d'étude.
    \item Étude des limites et des branches infinies, déterminer les asymptotes.
    \item Étude locale, recherche des tangentes et des possibles points de rebroussements et d'inflexion.
    \item Dressage du tableau de variation.
    \item Dessin de la courbe $\mathscr{C}$.
\end{enumerate}
\end{meth}
\begin{ex}

On considère la courbe $\mathscr{C}$ définie par :
$$\begin{array}{cccc}
    \mathscr{C} \ : & \mathscr{D}_{\mathscr{C}} & \to & \mathbb{R} \\
         & t & \mapsto & \begin{cases}x(t)=\frac{2t}{1+t^2}\\y(t)=\frac{2+t^3}{1+t^2}\end{cases}
\end{array}$$
On commence dans un premier temps par définir l'ensemble de définition.
Dans notre cas, le dénominateur est commun aux deux fonctions.\\
On cherche les valeurs interdites du dénominateur:
\begin{align*}
    & 1+t^2=0\\
    \Leftrightarrow & t^2=-1\\
    \Leftrightarrow & S=\{\varnothing\}
\end{align*}
En effet, une racine carrée ne pouvant être négative dans $\mathbb{R}$, les fonctions n'admettent aucune valeur interdite.
On obtient : $\mathscr{D}_\mathscr{C}=\mathbb{R}$

\begin{center}
\begin{pspicture*}(-3, -4)(3,5)
\psgrid[subgriddiv=0,griddots=10,gridlabels=7pt,gridcolor=gray]
\parametricplot[plotstyle=curve,algebraic,linecolor=red]{-5}{5}{2*t/(1+t^2) | (2+t^3)/(1+t^2) }
\psaxes[ticks=none,labels=none]{->}(0,0)(-3,-4)(3,5)
\end{pspicture*}
\end{center}
\end{ex}
\section{Courbes en polaire}
Cette section se concentrera sur l'étude des fonctions défini
\begin{defi}
Une courbe en polaire est une courbe paramétrée par :
$$\begin{array}{cccc}
    \mathscr{C} \ : & \mathscr{D}_{f} & \to & \mathbb{R} \\
         & \theta & \mapsto & r(\theta)
\end{array}$$
où $r(\theta)$ est la distance algébrique du point $M$ à l'origine.\\
Autrement dit, $\overrightarrow{OM(\theta)}=r(\theta)\overrightarrow{u_r(\theta)}=r(\theta)\begin{pmatrix}\cos\theta\\\sin\theta\end{pmatrix}$
\end{defi}
Le fait que $r(\theta)$ soit une distance algébrique traduit le fait que ...\\
Le point $M(\theta)$ est donc bien à une distance $|r(\theta)|$ mais dans la direction $-\overrightarrow{u_r(\theta)}$.
\begin{prop}
Soit $\mathscr{C}$ une courbe en polaire, un paramétrage de la courbe $\mathscr{C}$ serait :
$$r(\theta)\Leftrightarrow\begin{cases}x(\theta)=r(\theta)\cos(\theta)\\y(\theta)=r(\theta)\sin(\theta)\end{cases}$$
\end{prop}
La plupart du temps, ramener une courbe en polaire en paramétrée n'est pas un choix judicieux pour son étude.
\subsection{Domaine de définition et intervalle d'étude}
Périodicité :
Si la période n'est pas multiple de $2\pi$, alors il faut faire des rotations pour déterminer la courbe dans son ensemble.
Symétries :

\subsection{Étude des branches infinies}

\begin{prop}
\begin{itemize}
    \item Si $r(\theta)$ est périodique :
    \begin{itemize}
        \item Si $\lim\limits_{\theta\to \theta_0}r(\theta)\sin{\theta-\theta_0}=l$ avec $l\in\mathbb{R}$, la courbe admet une asymptote oblique d'équation $y=l$ dans le repère tourné d'angle $\theta_0$.
    \end{itemize}
    \item Si $r(\theta)$ n'est pas périodique :
    \begin{itemize}
        \item Si $\lim\limits_{\theta\to\pm\infty}r(\theta)=\pm\infty$, la courbe tend vers l'infini en spiralant.
        \item Si $\lim\limits_{\theta\to\pm\infty}r(\theta)=0$, la courbe tend vers $0$ en spiralant.
        \item Si $\lim\limits_{\theta\to\pm\infty}r(\theta)=l$ avec $l\in\mathbb{R}^{*}$, la courbe s'enroule vers le cercle centré à l'origine et de rayon $|l|$.
    \end{itemize}
\end{itemize}
\end{prop}
\subsection{Étude locale}
\subsection{Tableau de variation}
\subsection{Applications}
\begin{ex}
Soit la courbe $\mathscr{C}$ défini par la fonction $r(\theta)$:
$$\begin{array}{ccccc}
    r(\theta) & : & \mathscr{D}_r & \to & \mathbb{R} \\
     & & \theta & \mapsto & 1+\frac{1}{\theta - \frac{\pi}{4}}
\end{array}$$
\end{ex}
\section{Coniques}
\chapter{Propriétés métrique des courbes}
\chapter{Intégrales curvilignes}
\part{Équations et systèmes différentielles}
\chapter{Généralités}

\begin{defi}
Une équation différentielle est une équation reliant une fonction et ses dérivées successives.\\
L'inconnue est donc une fonction.\\
On peut l'écrire de façon très générale sous la forme $F(x,y,y^{\prime},y^{\prime\prime},\hdots,y^{(n)})=0$ où $y(x)$ est la fonction inconnue.
\end{defi}
\begin{defi}
On appelle système différentiel un ensemble d'équations différentielles, qui ne peuvent être résolues indépendamment les unes des autres.\\
Il se présente sous la forme :
$$\begin{cases}
a_n y_1^{(n)}+a_{n-1}y_1^{(n-1)}(t)+\hdots+a_2 y_1^{\prime\prime}(t)+a_1 y_1^{\prime}(t)+a_0y_1(t) = b_1(t)\\
\vdots \\
c_n y_k^{(n)} + c_{n-1}y_k^{(n-1)}(t)+\hdots+c_2 y_k^{\prime\prime}(t)+c_1 y_k^{\prime}(t)+c_0y_k(t) = b_k(t)
\end{cases}$$
où les $y_k$ sont les fonctions inconnues, $a_n$ et $c_n$ des coefficients et $b_k(t)$ représente les second membres des équations.
\end{defi}

Courbe intégrale
\chapter{Méthodes de résolutions explicites}
Nous allons dans ce chapitre explorer les différentes méthodes de résolution (selon le type d'équation) qui permettent de résoudre une équation différentielle ou un système différentielle de manière exacte.\\
Il est important de comprendre que nous ne connaissons aucune méthode de résolution générale, seulement des méthodes s'appliquant dans certains cas.
\section{Équations différentielles linéaires d'ordre $n$}
\begin{defi}
Une équation différentielle linéaire d'ordre est une équation de la forme :
$$a_n(t)y^{(n)}+a_{n-1}(t)y^{(n-1)}+\hdots+a_1(t)y^{\prime}+a_0(t)y+b(t)=0$$
\begin{align*}
    \text{où } & y \text{ est la fonction inconnue}\\
    & a_0(t),\hdots,a_n(t) \text{ sont des fonctions indépendantes de }y
\end{align*}
\end{defi}

\begin{prop}
On appelle polynôme caractéristique, le polynôme qui associe a chaque dérivées d'ordre n de l'équation homogène une variable $r$ à la puissance $n$.
$$a_ny^{(n)}+a_{n-1}y^{(n-1)}+\hdots+a_1y^{\prime}+a_0y+b(t)=0\Leftrightarrow a_nr^{n}+a_{n-1}y^{n-1}+\hdots+a_1r+a_0$$
\end{prop}
\begin{thm}[Théorème]
Si le polynôme caractéristique n'admet pour racine uniquement des racines simples $r_1,\hdots,r_n$ $\in\mathbb{C}$, l'ensemble des solutions est alors l'espace vectoriel engendré par $\{e^{r_1t},\hdots,e^{r_nt}\}$.
\end{thm}
\begin{thm}[Théorème]
Si le polynôme caractéristique $P(r)$ admet des racines multiples, pour chaque racine $r_k$ de multiplicité $m>1$
\end{thm}

\begin{ex}
\begin{itemize}
    \item Ordre 1 : $y^{\prime}-ty=0$
    \item Ordre 2 : $y^{\prime\prime}+2y^{\prime}+y=0$
    \item Ordre 5 : $y^{(5)}+y^{(4)}-4y^{(3)}-16y^{\prime\prime}-20y^{\prime}-12y=0$
\end{itemize}
\end{ex}




\section{Équations différentielles à variables séparables}
\begin{defi}
On dit qu'une équation différentielle est à variable séparées si l'on peut factoriser $f(y,t)$ en termes ne dépendant que de $y$ ou que de $t$.
$$y^{\prime}=f(y,t)=g(t)h(y)$$
\end{defi}
\begin{ex}
\begin{itemize}
    \item Soit $y^{\prime}(t)=t^2y(t)$ \\
    On peut factoriser par $g(t)=t^2$ et $h(y)=y$, on a donc bien $f(y,t)=g(t)h(y)$. L'équation est a variables séparables.
    \\
    \item Soit $y^{\prime}(t)y^2(t)=e^t$\\
    On peut la réécrire : $y^{\prime}(t)=\frac{e^t}{y^2(t)}$\\
    On peut factoriser par $g(t)=e^t$ et $h(y)=\frac{1}{y^2}$. L'équation est à variables séparables.\\
    \item L'équation $y^{\prime}(t)+y(t)-t^2=0$ n'est pas à variables séparables car l'on ne peut pas factoriser cette dernière.
\end{itemize}
\end{ex}
\begin{meth}[Résolution d'une équation différentielle à variables séparées]
\begin{enumerate}
    \item On vérifie, par l'intermédiaire de la définition, si l'équation est à variables séparables.
    \item On pose $\frac{y^{\prime}}{h(y)}=g(t)$.
    \item On intègre la quantité précédemment déterminer $\int \frac{y^{\prime}}{h(y)dt}=\int g(t)dt$.
    \item On effectue le changement de variable $u=y$ donc $du=y^{\prime}dt$, on obtient :\\ $\int\frac{du}{h(u)}=\int g(t) dt$.
    \item On obtient : $H(u)=G(t)+C$ où $H$ et $F$ sont respectivement les primitives de $f$ et $\frac{1}{g}$ et $C$ une constante.
    \item On cherche $H^{-1}$ la fonction réciproque $H$.
    \item On effectue $y(t)=u=H^{-1}(F(t)+C)$
\end{enumerate}
\end{meth}
\begin{ex}
\begin{itemize}
    \item Soit l'équation $(e^t+1)y^{\prime}+e^ty^2=0$.\\
    On souhaite savoir si l'équation est à variables séparables.\\
    On peut réécrire : $y^{\prime}=\frac{e^t}{e^t+1}y^2$ avec $g(t)=\frac{e^t}{e^t+1}$ et $h(y)=y^2$.\\
    L'équation est à variables séparables.\\
    On pose donc :\\
    $$\int\frac{y^{\prime}}{h(y)}\ dt=\int g(t)\ dt$$
    $$\Leftrightarrow\int\frac{y^{\prime}}{y^2}\ dt=\int \frac{-e^t}{e^t+1}\ dt$$
    $$\Leftrightarrow-\int -\frac{y^{\prime}}{y^2}\ dt=\int \frac{-e^t}{e^t+1}\ dt$$
    $$\Leftrightarrow-\frac{1}{y}+C_1= \ln(e^t+1) +C_2$$
    $$\Leftrightarrow y=\frac{1}{\ln(e^t+1)+C}\ \text{avec}\ C=C_2-C_1$$
    La solution de l'équation est donc : $y(t)=\frac{1}{\ln(e^t+1)+C}$
    \item Soit l'équation $y^{\prime}=e^ty^2(t)$.\\
    On voit facilement que l'équation est à variables séparables.\\
    On pose :
    \begin{align*}
        & \frac{y^{\prime}(t)}{y^2(t)}=e^t\\
        \Leftrightarrow & \int\frac{y^{\prime}(t)}{y^2(t)}=\int e^t\ dt\\
        \Leftrightarrow & \int\frac{du}{u^2}=\int e^t\ dt\ \text{avec } u=y\Rightarrow du=y^{\prime}(t)dt\\
        \Leftrightarrow & \frac{-1}{u}=e^t+C\\
    \end{align*}
\end{itemize}
\end{ex}
\section{Systèmes différentielles linéaires}
\subsection{Résolution par diagonalisation}
Dans un premier temps nous allons nous intéresser à la résolution des systèmes différentiels linéaire d'ordre 1 homogène.
\begin{thm}[Théorème]
Soit un système différentiel linéaire d'ordre 1 et sa représentation matriciel.
$$
\begin{cases}
y_1^{\prime}=a_1 y_1 + b(t)\\
\vdots\\
y_{n}^{\prime}=c_1 y_n + b_n(t)
\end{cases}\Leftrightarrow\ Y^{\prime}(t)=AY(t)+B(t)$$
On s'intéresse aux solutions homogènes ($B=0$) d'un tel système.
On pose $Y=PZ$, le système devient : $Z^{\prime}=DZ$
Où $D$ est la matrice diagonale semblable à $A$ 
 et $P$ la matrice de passage vers la base où $D$ existe.\\
Les solutions d'un tel système sont données par :
$$\begin{cases}
z_1=C_1e^{\lambda_1 t}\\
\vdots\\
z_n=C_ne^{\lambda_n t}
\end{cases}$$
Seulement nous cherchons les solutions de $Y$ et non de $Z$, par définition de $Z$ il suffit de poser :
$$Y=PZ$$
\end{thm}
\begin{demo}
Soit un système différentiel linéaire d'ordre 1 et sa représentation matriciel :
$$Y^{\prime}=AY$$
Par hypothèse, nous considérons $A$ diagonalisable, donc :
$$Y^{\prime}=PDP^{-1}$$
On pose $P^{-1}Y=Z$ on peut réécrire : 
$$PZ=PDZ$$
On dérive la quantité $PZ$, $P$ étant constante on obtient $PZ^{\prime}$, donc :
$$PZ^{\prime}=PDZ$$
La matrice $P$ est, par définition, inversible (matrice de passage), on peut donc simplifier par $P$.
$$Z^{\prime}=DZ$$
On obtient donc le système différentiel suivant :
$$\begin{cases}
z_1^{\prime}=\lambda_1 z_1\\
\vdots\\
z_n^{\prime}=\lambda_n z_n\\
\end{cases}$$
Et en résolvant, les équations une a une du système on obtient finalement :
$$\begin{cases}
z_1(t)=C_1 e^{\lambda_1 t}\\
\vdots\\
z_n(t)=C_n e^{\lambda_n t}\\
\end{cases}$$
Donc $z(t)=C_1 e^{\lambda_1 t}+\hdots+C_n e^{\lambda_n t}$
\end{demo}
\begin{ex}
Nous souhaitons résoudre le système différentiel suivant :
$$\begin{cases}
-x^{\prime}+x+3y = 0\\
-y^{\prime}+4x-3y = 0
\end{cases}$$
On commence par mettre le système sous forme résolue :
$$\begin{cases}
x^{\prime}=x+3y\\
y^{\prime}=4x-3y
\end{cases}$$
On cherche sa représentation matriciel, on pose $Y=\begin{pmatrix}x\\y\end{pmatrix}$.\\
On peut écrire :
$$Y^{\prime}=\begin{pmatrix}x^{\prime}\\y^{\prime}\end{pmatrix}=\begin{pmatrix}1&3\\4&-3\end{pmatrix}\begin{pmatrix}x\\y\end{pmatrix}$$
On pose : $Z=\begin{pmatrix}z_1\\z_2\end{pmatrix}=PV$.
On peut réécrire le système comme $Z^{\prime}=DZ$.\\
Partons à la recherche de la matrice diagonale $D$ et notamment des éléments propres de $A$.\\
Ici $P_A(\lambda)=(1-\lambda)(-3-\lambda)-12$, sa résolution nous renvoie : $\lambda_1=-5$ et $\lambda_2=3$.\\
On recherche les vecteurs propres associés aux valeurs propres $\lambda_1$ et $\lambda_2$.\\
La recherche des vecteurs et sous-espaces propres nous donne :\\
La matrice $A$ est diagonalisable, on en déduit les matrices $D$ et $P$ :
$$D=\begin{pmatrix}-5&0\\0&3\end{pmatrix}\ \text{et}\ P=\begin{pmatrix}-1&3\\2&2\end{pmatrix}$$
On peut donc réécrire le système comme :
$$Z^{\prime}=\begin{pmatrix}z_1^{\prime}\\z_2^{\prime}\end{pmatrix}=\begin{pmatrix}-5&0\\0&3\end{pmatrix}\begin{pmatrix}z_1\\z_2\end{pmatrix}$$
Il nous faut donc résoudre :
$$\begin{cases}
z_1^{\prime}=-5z_1\\
z_2^{\prime}=3z_2
\end{cases}$$
Par le théorème précédent, on déduit :
$$\begin{cases}
z_1(t)=C_1 e^{-5t}\\
z_2(t)=C_2 e^{3t}
\end{cases}$$
Pour trouver les solution pour $y$ il suffit de poser :
\begin{align*}
    \Leftrightarrow & Y=PZ\\
    \Leftrightarrow & Y=\begin{pmatrix}-1&3\\2&2\end{pmatrix}\begin{pmatrix}z_1\\z_2\end{pmatrix}\\
    \Leftrightarrow & Y=\begin{pmatrix}-z_1+3z_2\\2(z_1+z_2)\end{pmatrix}\\
    \Leftrightarrow & Y=\begin{pmatrix}-C_1e^{-5t}+3C_2e^{3t}\\2(C_1e^{-5t}+C_2e^{3t})\end{pmatrix}
\end{align*}
Les solutions du système sont donc :
$$\begin{cases}
x(t)=-C_1e^{-5t}+3C_1e^{3t}\\
y(t)=2(C_1e^{-5t}+C_2e^{3t})\\
\end{cases}$$
\end{ex}
On s'intéresse désormais aux solutions particulières de tels systèmes.
\begin{thm}[Théorème]
Soit un système différentiel linéaire d'ordre 1 et sa représentation matriciel.
$$
\begin{cases}
y_1^{\prime}=a_1 y_1 +\hdots+a_n y_n + b\\
\vdots\\
y_{n}^{\prime}=c_1 y_1 +\hdots+c_n y_n + b_n
\end{cases}\Leftrightarrow\ Y^{\prime}=AY+B$$
On pose $Y=PV$ où $P$ est la matrice de passage entre la base canonique et la base où se trouve la matrice diagonale.\\
Les solutions particulières d'un tel système sont données comme il suit :
$$V^{\prime}=DV+\underbrace{P^{-1}B}_{C}\ \Leftrightarrow\ 
\begin{cases}
v_1^{\prime}=\lambda_1 v+c_1\\
\vdots\\
v_{n}^{\prime}=\lambda_n +c_n
\end{cases}$$
Pour revenir sur la variable $y$ il suffit d'appliquer la matrice $P^{-1}$ sur le système défini par la variable $v$.
\end{thm}
\begin{demo}
Soit le système sous forme matricielle, $Y^{\prime}=AY+B$.\\
On considère le vecteur $V$ tel que $Y=PV$
\begin{align*}
    \Leftrightarrow & Y^{\prime}=AY+B\\
    \Leftrightarrow & (PV)^{\prime}=APV+B\\
    \Leftrightarrow & PV^{\prime}=APV+B\\
    \Leftrightarrow & V^{\prime}=\underbrace{P^{-1}AP}_{D}V+P^{-1}B\\
    \Leftrightarrow & V^{\prime}=DV+P^{-1}B
\end{align*}
\end{demo}
\begin{ex}
On considère le système différentiel :
$$\begin{cases}
-z_1^{\prime}+z_1-z_2=e^{t}\\
-z_2^{\prime}+2z_1+4z_2=0
\end{cases}$$
On isole la plus haute dérivée :
$$\begin{cases}
z_1^{\prime}=z_1-z_2+e^{t}\\
z_2^{\prime}=2z_1+4z_2
\end{cases}$$
On réécrit le système sous sa forme matricielle, pour ce faire on pose $Y=\begin{pmatrix}z_1\\z_2\end{pmatrix}$. On a donc :
$$Y^{\prime}=\begin{pmatrix}1&-1\\2&4\end{pmatrix}Y+\begin{pmatrix}e^t\\0\end{pmatrix}$$
On essaie de diagonaliser la matrice $A$.\\
On commence donc par rechercher les éléments propres.\\
Ici $P_A(\lambda)=(1-\lambda)(4-\lambda)-4$ et sa résolution nous donne $\lambda_1=3$ et $\lambda_2=2$.\\
La recherche des sous-espaces propres nous renvoie :
$$E_{\lambda_1}=\text{vect}\begin{pmatrix}\begin{pmatrix}1\\-2\end{pmatrix}\end{pmatrix}$$
$$E_{\lambda_2}=\text{vect}\begin{pmatrix}\begin{pmatrix}1\\-1\end{pmatrix}\end{pmatrix}$$
La matrice est diagonalisable. On déduit donc :\\
$$D=\begin{pmatrix}3&0\\0&2\end{pmatrix}\ \text{et}\ P=\begin{pmatrix}1&1\\-2&-1\end{pmatrix}$$
On en profite pour déterminer la matrice de passage inverse :
\begin{align*}
    \Leftrightarrow & \begin{pmatrix}[cc|cc]
   1 & 1 & 1 & 0 \\  
   -2 & -1 & 0 & 1
 \end{pmatrix}\\
 \Leftrightarrow & \begin{pmatrix}[cc|cc]1 & 1 & 1 & 0\\
 0&1&2&1\end{pmatrix}\\
 \Leftrightarrow & \begin{pmatrix}[cc|cc]1&0&-1&-1\\0&1&2&1\end{pmatrix}
\end{align*}
On déduit que $P^{-1}=\begin{pmatrix}-1&-1\\2&1\end{pmatrix}$.\\
On défini $V$ tel que $Y=PV$, on a donc $V'=DV+P^{-1}B$.\\
\begin{align*}
    \Leftrightarrow & V^{\prime}=DV+P^{-1}B \\
    \Leftrightarrow & V^{\prime}=\begin{pmatrix}3&0\\0&2\end{pmatrix}\times V +\begin{pmatrix}-1&-1\\2&1\end{pmatrix}\begin{pmatrix}-e^t\\0\end{pmatrix}\\
    \Leftrightarrow & V^{\prime}=\begin{pmatrix}3&0\\0&2\end{pmatrix}\times V +\begin{pmatrix}e^t\\-2e^t\end{pmatrix}\\
    \Leftrightarrow & \begin{cases}
    v_1^{\prime}=3v_1+e^t\\
    v_2^{\prime}=2v_2-2e^t
    \end{cases}
\end{align*}
On résout donc les deux équations indépendamment.
\begin{enumerate}
    \item Recherche de la solution générale de l'équation $v_1^{\prime}=3v_1+e^t\ (E)$.
    \begin{enumerate}
        \item Résolution de l'équation homogène $v_{1h}$ :\\
        On détermine l'équation homogène associée $(E_0)$ : $v_1^{\prime}=3v_1$.\\
        Par l'un des théorèmes précédent, on sait que les solutions de cette équation sont de la forme :
        $$v_{1h}(t)=C_1e^{3t}$$
        La fonction $v_{1h}=C_1e^{3t}$ est solution générale de l'équation $(E)$
        \item Recherche d'une solution particulière $v_{1p}$:\\
        On utilise la méthode de variation de la constante.
        Pour ce faire on pose $C_1=k(t)$, on a donc :
        $$v_1(t)=k(t)e^{3t}$$
        $$v_1^{\prime}=k^{\prime}(t)e^{3t}+3k(t)e^{3t}$$
        On peut réécrire $(E)$ de la manière suivante :
        \begin{align*}
            \Leftrightarrow & k^{\prime}(t)e^{3t}+3k(t)e^{3t}=3k(t)e^{3t}+e^t\\
            \Leftrightarrow & k^{\prime}(t)=e^{-2t}\\
            \Leftrightarrow & k(t) = \int e^{-2t}\ dt\\
            \Leftrightarrow & k(t) = \frac{1}{-2}\int e^{-2t}\\
            \Leftrightarrow & k(t) = \frac{1}{-2}e^{-2t}
        \end{align*}
        On injecte la fonction $k(t)$ dans la solution homogène :\\
        $$v_{1p}=\frac{e^{-t}}{-2}e^{3t}=\frac{e^t}{-2}$$
        La fonction $v_{1p}(t)$ est une solution particulière de l'équation $(E)$.
        \item Déduction de la solution générale de $(E)$ :
        Par le principe de superposition, on déduit la solution générale de l'équation :
        $$v_1(t)=v_{1h}+v_{1p}$$
        $$v_1(t)=C_1e^{3t}-\frac{e^t}{2}$$
    \end{enumerate}
    \item Recherche de la solution générale $v_2^{\prime}=2v_2-2e^t$ :\\
    \begin{enumerate}
        \item Résolution de l'équation homogène associée $(E_0)$ :\\
        Par le même raisonnement que précédemment, on trouve que :
        $$v_{1h}(t)=C_2e^{2t}$$
        \item Recherche d'une solution particulière :\\
        On réutilise la méthode de variation de la constante.
        On pose $C_2=k(t)$, alors :
        $$v_2=k(t)e^{2t}$$
        $$v_2^{\prime}=k^{\prime}(t)e^{2t}+2k(t)e^{2t}$$
        On a donc :
        \begin{align*}
            \Leftrightarrow & k^{\prime}(t)e^{2t}+2k(t)e^{2t}=2k(t)e^{2t}-2e^t\\
            \Leftrightarrow & k^{\prime}(t)=-2e^{-t}\\
            \Leftrightarrow & k(t) = \int -2e^{-t}\ dt\\
            \Leftrightarrow & k(t) = -2\int e^{-t}\\
            \Leftrightarrow & k(t) = 2e^{-t}
        \end{align*}
        On injecte la fonction $k(t)$ dans la solution homogène :
        $$v_{1p}=2e^{-t}e^{2t}=2e^{t}$$
        La fonction $v_{1p}$ est solution particulière de l'équation $(E)$.
        \item Déduction de la solution générale :
        Par le principe de superposition, on a :
        $$v_2(t)=C_2e^{2t}+2e^t$$
    \end{enumerate}
\end{enumerate}
On peut alors réécrire le système précédent comme :
$$\begin{cases}
v_1(t)=C_1e^{3t}-\frac{e^t}{2}\\
v_2(t)=C_2e^{2t}+2e^t
\end{cases}$$
Par la définition du vecteur $V$ on peut écrire :
\begin{align*}
    \Leftrightarrow & Y=PV\\
    \Leftrightarrow & Y=\begin{pmatrix}1&1\\-2&-1\end{pmatrix}\begin{pmatrix}v_1\\v_2\end{pmatrix}\\
    \Leftrightarrow & Y=\begin{pmatrix}v_1 +v_2\\-2v_1-v_2\end{pmatrix}\\
    \Leftrightarrow & Y=\begin{pmatrix}C_1e^{3t}-\frac{e^t}{2}+C_2e^{2t}+2e^t\\-2C_1e^{3t}+e^t-C_2e^{2t}-2e^t\end{pmatrix}
\end{align*}
Le système a donc pour solutions :
$$\begin{cases}
z_1(t)=C_1e^{3t}-\frac{e^t}{2}+C_2e^{2t}+2e^t\\
z_2(t)=-2C_1e^{3t}+e^t-C_2e^{2t}-2e^t
\end{cases}$$
\end{ex}
\subsection{Comportement des solutions selon les valeurs propres}
\subsection{Tracés de systèmes différentiels}
\begin{ex}
On reprend l'exemple de la résolution d'un système homogène. On a le système suivant, ses solutions et ses conditions :
$$\begin{array}{ccc}
    \begin{cases}x^{\prime}=x+3y\\y^{\prime}=4x-3y\end{cases} & \begin{cases}x(t)=-C_1e^{-5t}+3C_2e^{3t}\\y(t)=2(C_1e^{-5t}+C_2e^{3t})\\\end{cases} & \text{les conditions}\ x(0)=a\ \text{et}\ y(0)=b\\
\end{array}$$
On commence par déterminer les constantes $C_1=\mu$ et $C_2=\nu$ en fonction des conditions initiales :
\begin{align*}
    \Leftrightarrow & \begin{cases}x(0)=-\mu e^{-5t}+3\nu e^{3t}\\y(0)=2(\mu e^{-5t}+\nu e^{3t})\end{cases}\\
    \Leftrightarrow & \begin{cases}-\mu e^{-5t}+3\nu e^{3t}=a\\2(\mu e^{-5t}+\nu e^{3t})=b\end{cases}\\
    \Leftrightarrow & \begin{cases}-\mu +3\nu=a\\2\mu+2\nu=b\end{cases}\\
    \Leftrightarrow & \begin{cases}-\mu+3\nu=a\\8\nu=b+2a\end{cases}\\
    \Leftrightarrow & \begin{cases}-\mu+3\nu=a\\\nu = \frac{b+2a}{8}\end{cases}\\
    \Leftrightarrow & \begin{cases}-\mu =a-\frac{3b-6a}{8}\\\nu=\frac{b+2a}{8}\end{cases}\\
    \Leftrightarrow & \begin{cases}\mu = \frac{3b+2a}{8}\\\nu=\frac{b+2a}{8}
    \end{cases}
\end{align*}
On peut donc réécrire les solutions :
$$\begin{cases}
x(t)=\left( \frac{-3b-2a}{8} \right)e^{-5t}+\left( \frac{3b+6a}{8} \right)e^{3t}\\
y(t)=\left(\frac{6b+2a}{8}\right)e^{-5t}+\left(\frac{2b+4a}{8}\right)e^{3t}
\end{cases}$$
$$\Leftrightarrow\ \begin{cases}x(t)=\frac{1}{8}\left(\left(-3b-2a\right)e^{-5t}+3\left(b+2a\right)e^{3t}\right)\\y(t)=\frac{1}{4}\left(\left(3b+a\right)e^{-5t}+\left(b+2a\right)e^{3t}\right)\end{cases}$$
\end{ex}
\section{Intégrales premières ou constantes du mouvement}
Lorsqu'il n'est pas possible de résoudre de manière explicite une équation ou un système différentielle, il peut être judicieux de passer par une intégrale première (autrement appelée constante du mouvement.
\begin{defi}
Pour un système autonome, la fonction $V$ est une intégrale première si et seulement si la quantité $\overrightarrow{\nabla}V\times f=0$
\end{defi}



\begin{ex}
\begin{itemize}
    \item Équations de prédation de Lokta-Volterra :\\
    Il s'agit d'un couple d'équation différentielle, décrivant l'évolution des proies et des prédateurs présent dans un milieu.
    On pose $x(t)$ le nombre de proie et $y(t)$ le nombre de prédateurs.
    On suppose que le nombre de proie est linéairement proportionnel au nombre de prédateur et réciproquement.
    $$\begin{cases}
    x^{\prime}(t)=ax(t)-bx(t)y(t)\\
    y^{\prime}(t)=-cy(t)+dy(t)x(t)
    \end{cases}$$
    Où les constantes $a$, $b$, $c$ et $d$ sont positives dont chacune représente respectivement :
    \begin{itemize}
        \item la natalité naturelle des proies
        \item la mortalité liée aux prédateurs des proies
        \item la mortalité naturelle des prédateurs
        \item la natalité des prédateurs liée au nombre de prédateur
    \end{itemize}
    On constate que le système n'est pas linéaire, on ne peut donc pas résoudre le système par les méthodes présentés précédemment.
    On donne la constante du mouvement :\\
    $$V(x,y)=dx-c\ln(x)+by-a\ln(y)$$
    Les solutions du systèmes sont donc des courbes de niveaux de $V(x,y)$
\end{itemize}
\end{ex}
\section{Utilisation des formes différentielles}
\section{Autres méthodes}


\section{Comportement des solutions}
\subsection{Comportement asymptotique}
\subsection{Comportement selon les valeurs propres}














Il est nécessaire pour la résolution des systèmes de les "convertir" en problèmes de Cauchy.
\begin{prop}
On peut réécrire une système différentiel sous une forme matriciel.
\begin{itemize}
    \item Pour un système d'ordre  1 : $$Y^{\prime}=AY+B$$
    \item Pour un système d'ordre n :$$Y^{(n)}=AY+B$$
    où $Y$ est le vecteur contenant toutes les dérivées d'ordre inférieur à $n$ :
    $$Y=\begin{pmatrix}y_1^{\prime}\\\vdots\\y_k^{\prime}\\\vdots\\y_1^{(n)}\\\vdots\\y_k^{(n)}\end{pmatrix}$$
\end{itemize}
\end{prop}
\begin{ex}
On considère le système :\\
\begin{center}
    $\begin{cases}\dot x = x-y+e^t\\\dot y = 2x+4y\end{cases}$ Ici :  $Y=\begin{pmatrix} x \\ y\end{pmatrix}$ , $A=\begin{pmatrix}1&-1\\2&4\end{pmatrix}$ et $b(t)=\begin{pmatrix}e^t\\0\end{pmatrix}$.
\end{center}.
On peut donc écrire : $\dot Y = \begin{pmatrix}1&-1\\2&4\end{pmatrix}Y+\begin{pmatrix}e^t\\0\end{pmatrix}$\\
Soit le système :\\
\begin{center}
    $\begin{cases}x^{\prime\prime}=-2\omega^2x+\omega^2y\\y^{\prime\prime}=\omega^2x-2\omega^2y\end{cases}$
\end{center}
On pose donc le vecteur $Y$ contenant toutes les dérivées d'ordres inférieurs à $2$.\\
On a donc $Y=(x,y,x^{\prime},y^{\prime})$.
Puis on fait sorte de retrouver le même système par multiplication. On obtient :
$$A=\begin{pmatrix}0&0&1&0\\0&0&0&1\\-2\omega^2&\omega^2&0&0\\\omega^2&-2\omega^2&0&0\end{pmatrix}$$
Donc on peut écrire : $Y^{\prime}=\begin{pmatrix}x^{\prime}\\y^{\prime}\\x^{\prime\prime}\\y^{\prime\prime}\end{pmatrix}=\begin{pmatrix}0&0&1&0\\0&0&0&1\\-2\omega^2&\omega^2&0&0\\\omega^2&-2\omega^2&0&0\end{pmatrix}\begin{pmatrix}x\\y\\x^{\prime}\\y^{\prime}\end{pmatrix}$
\end{ex}
\section{Systèmes différentiels linéaires d'ordre 1}

\subsubsection{Diagonalisation dans $\mathbb{C}$}
\subsection{Résolution par triangulation}

\chapter{Introduction au calcul variationnel}
blaaa blabla

\section{Lagrangien}
\begin{defi}
$\mathcal{L}\ \mathscr{L}$
\end{defi}
\begin{thm}[Théorème d'Euler-Lagrange]
$$\frac{d}{dt}\frac{\partial\mathcal{L}}{\partial\dot x_i}=\frac{\partial\mathcal{L}}{\partial x_i}$$
\end{thm}
\section{Hamiltonien}
\begin{defi}
$$\mathcal{H}(q_i,p_i,t)=\sum_k^N\dot q_k p_k -\mathcal{L}(q_i,\dot q_i,t)$$
\end{defi}
\part{Mec-301 : Mécanique du solide indéformable}
Introduction ::
Tourabi Ali, CM $\&$ TD\\
\chapter{Torseurs}
Le torseur est un objet français (Cocorico !), permettant de représenter toutes les actions que subit un solide.\\
C'est un outils qui facilite les calculs et la formulation des lois.
En effet, il faudrait diviser les lois en plusieurs théorème pour complètement inclure les informations contenues dans un torseur.

Les torseurs sont un manière de d'écrire un champ de vecteur, ils sont tous caractérisés par trois paramètres : deux vecteurs et un point.
\section{Actions mécaniques}
\section{Moment d'une force}
\begin{defi}
On appelle moment en $A$ de la force $\overrightarrow{F}$ passant par le point $P$, du solide $\Sigma$ :
$$\overrightarrow{M_{A}(\overrightarrow{F})}=\overrightarrow{AP}\land\overrightarrow{A}$$
\end{defi}
Un moment représente la capacité d'une force a crée une rotation autour d'un axe.\\
On peut définir la formule de transport des moments qui permet, connaissant le moment en un point $A$ du solide, de calculer le moments de cette force sur n'importe quel point du solide.
\begin{prop}
Soient $A$ et $B$ deux points de l'espace et $\overrightarrow{M_{A}}(\overrightarrow{F})$, $\overrightarrow{M_{B}}(\overrightarrow{F})$ leurs moments associés de la force $\overrightarrow{F}$ appliquée en $P$ au solide $\Sigma$.\\
On définit la formule de transport des moments :
$$\overrightarrow{M_{B}}(\overrightarrow{F})=\overrightarrow{M_{A}}(\overrightarrow{F})+\overrightarrow{BA}\land\overrightarrow{F}$$
\end{prop}
\begin{demo}
Soit
\end{demo}
\begin{meth}[Calcul d'un moment avec le bras de levier]
Dans un problème a deux dimensions, nous pouvons calculer un moment sans passer par le produit vectoriel. Pour ce faire, on utilise le bras de levier.\\
On commence par trouver sa norme : $\|\overrightarrow{M_O}(\overrightarrow{F})\|=\|\overrightarrow{F}\|\times\|\overrightarrow{OA}\|$\\
Puis on détermine le vecteur unitaire par lequel il est porté ainsi que son signe :\\
Ici la force $\overrightarrow{F}$
\end{meth}
\section{Torseur force}
\begin{defi}
On appelle torseur force l'objet constitué de deux éléments de réductions :
\begin{itemize}
    \item la résultante des Forces appliquées au système, noté $R$
    \item la résultante des Moments appliqués au point considéré, noté $M_A$ avec $A$ un point du solide $\Sigma$
\end{itemize}
Le torseur force a donc trois paramètres : résultante des forces, résultante des moments et point d'application.
$$\{\tau\}_{A}=\begin{Bmatrix}\overrightarrow{R}\\\overrightarrow{M_A}\end{Bmatrix}_A=\begin{Bmatrix} \displaystyle\sum_{i=1}^{n}\overrightarrow{F_i}\\\displaystyle\sum_{i=1}^{n}\overrightarrow{AP_i}\land\overrightarrow{F_i}\end{Bmatrix}_A=\begin{Bmatrix} R_x & R_y  & R_z\\ M_x & M_y & M_z\end{Bmatrix}_A$$
\end{defi}
\begin{ex}
jfienf
\end{ex}
\begin{rmq}
Dans le cas d'un point matériel le torseur force se réduit seulement à la résultante des forces.
\end{rmq}
\subsection{Torseur de force répartie}

\begin{ex}
     Balle en l'air avec son poids une rotation et la pression de l'aire ou une mongolfière avec la poussée d'archimède et la pression de l'aire et son poids
\end{ex}
\section{Torseur cinématique}
\begin{defi}
On appelle torseur cinématique, l'objet constitué des éléments de réduction :
`\begin{itemize}
    \item 
\end{itemize}
$${\lbrace\chi\rbrace}_{A/R_0}=\begin{Bmatrix}\overrightarrow{\omega}_{\Sigma/R_0}\\\overrightarrow{V}_{\Sigma/R_0}(A)\end{Bmatrix}_{A/R_0}=\begin{Bmatrix} \omega_x & \omega_y  & \omega_z\\ u_A & v_A & w_A\end{Bmatrix}_{A/R_0}$$
\end{defi}
\section{Torseur de liaisons}
\section{Torseur déplacement infinitésimal}
\begin{defi}

$${\lbrace\delta\chi\rbrace}_{A/R_0}=\begin{Bmatrix}\delta\overrightarrow{\omega}_{\Sigma/R_0}\\\delta\overrightarrow{l}_{\Sigma/R_0}(A)\end{Bmatrix}_{A/R_0}=\begin{Bmatrix} \delta\omega_x & \delta\omega_y  & \delta\omega_z\\ \delta u_A & \delta v_A & \delta w_A\end{Bmatrix}_{A/R_0}$$
\end{defi}
\section{Opérations sur les torseurs}
On définit pour les torseurs les opérations de base permettant a ces dernier d'interagir entre-eux.
\begin{prop}
\begin{itemize}
    \item Addition :
    $$\{\tau_1\}_A+\{\tau_2\}_A=\begin{Bmatrix}\overrightarrow{R}_1\\\overrightarrow{M_{A1}}\end{Bmatrix}_A+\begin{Bmatrix}\overrightarrow{R}_2\\\overrightarrow{M_{A2}}\end{Bmatrix}_A=\begin{Bmatrix}\overrightarrow{R}_1+\overrightarrow{R}_2\\\overrightarrow{M_{A1}}+\overrightarrow{M_{A2}}\end{Bmatrix}_A$$
    \item Multiplication par un scalaire :
    $$\{\tau_2\}_A=\lambda\{\tau_1\}_A \Leftrightarrow \{\tau_2\}_A=\begin{Bmatrix}\overrightarrow{R_2}\\\overrightarrow{M_{A2}}\end{Bmatrix}_A=\begin{Bmatrix}\lambda\overrightarrow{R_1}\\\lambda\overrightarrow{M_{A1}}\end{Bmatrix}_A$$
    \item Produit :
    $$\lambda=\{\tau_1\}_A\cdot\{\tau_2\}_A \Leftrightarrow\lambda=\overrightarrow{R_1}\cdot\overrightarrow{M_{A2}}+\overrightarrow{R_2}\cdot\overrightarrow{M_{A1}}$$
\end{itemize}
\end{prop}
\chapter{Statique du solide}
\section{Principe fondamentale de la statique}
\section{Principe des actions réciproques}
\section{Applications aux cas usuelles}
\section{Frottements secs}
\chapter{Cinématique du solide}
La définition des grandeurs cinématiques est exactement la même qu'en mécanique du point. C'est à dire que la vitesse est la dérivée de la position, et l'accélération celle de la vitesse. \\
On considère
\section{Expression des vitesses}
$$\overrightarrow{V_{/R_0}}(M)=\overrightarrow{V_a}(M)$$
Nous en déduisons la propriété de composition des vitesses.
\begin{thm}[Théorème : Composition des vitesses]
Soient le référentiel absolue $R_0$ (référentiel fixe) et les référentiels $R_1,\hdots,R_n$ en mouvement les uns par rapport aux autres. \\
On considère un point $M$ fixe dans le référentiel $R_n$.
La vitesse absolue, peut s'écrire :
$$\overrightarrow{V_a}=\overrightarrow{V_{R_n/R_0}}=\overrightarrow{V_{R_n/R_{n-1}}}+\hdots+\overrightarrow{V_{R_2/R_1}}+\overrightarrow{V_{R_1/R_0}}$$
On peut également écrire cette notion de manière torsorielle :
$$\{\chi_{R_n/R_0}\}=\{\chi_{R_n/R_{n-1}}\}$$
\end{thm}
\section{Expression des accélérations}
$$\overrightarrow{\gamma_a}(M)=\overrightarrow{\gamma_e}(M)+\overrightarrow{\gamma_r}(M)+\overrightarrow{\gamma_c}(M)$$
\begin{thm}[Théorème : Composition des accélérations]

\end{thm}
\section{Vitesse de glissement}
\chapter{Cinétique du solide}
En cinématique, les mouvements des corps sont considérés en omettant l'inertie des ces derniers.
En réalité, les mouvements des systèmes sont liés aux causes d'une part et à leurs inertie d'autre part.\\
\\
La notion d'inertie caractérise la propriété du système de changer plus ou moins rapidement sa vitesse sous l'effet des forces qui lui sont appliquées.
On simplifie souvent les choses, en réduisant l'inertie à la masse. Or le mouvement d'un solide ne dépend pas que de sa masse et des forces exercées sur ce dernier.
En effet il dépend également de sa géométrie, de la distribution de sa masse, ...
\\
\\
L'étude de l'inertie s'effectue dans le cadre de la cinétique.
\section{Géométrie des masses}
\subsection{Notion de masse}
\begin{defi}
A chaque solide $\Sigma$ est associée un quantité (scalaire), noté $m$, qui représente la matière contenue dans ce dernier. On la définie comme il suit :
$$m=\int dm$$
avec $dm$ l'élément infinitésimal de masse
\end{defi}
Sachant cela il est possible de définit la masse à partir des distributions de masse :
\begin{itemize}
    \item Masse volumique $\rho$ (en $kg.m^{-1}$)
    \item Masse surfacique $\sigma$ (en $kg.m^{-2}$)
    \item Masse linéique $\lambda$ (en $kg.m^{-1}$)
\end{itemize}

Par définition de ces trois grandeurs, on peut écrire l'élément infinitésimal de masse $dm$ comme :
\begin{itemize}
    \item $dm=\rho(M)dV$
    \item $dm=\sigma(M)dS$
    \item $dm=\lambda(M)dl$
\end{itemize}
où $dV$, $dS$ et $dl$ sont respectivement les éléments infinitésimaux de volume, de surface et le déplacement élémentaire autour d'un point M.\\
 \\
En définitive, on peut donc écrire :
\begin{itemize}
    \item $m=\int \rho(M)dV$ $\rightarrow$ Très utilisée pour les solides en 3 dimensions
    \item $m=\int \sigma(M)dS$ $\rightarrow$ Utilisée pour les plaques
    \item $m=\int \lambda(M)dl$ $\rightarrow$ Utilisée pour les tiges
\end{itemize}
\begin{ex}
...
\end{ex}
\subsection{Centre de Gravité et référentiel barycentrique}
\section{Torseur cinétique}
\section{Matrice d'inertie}

\part{Phy-301 : Électromagnétisme 1}
\chapter{Électrostatique}
\section{Loi de Coulomb, champ électrostatique}
\subsection{Loi de Coulomb}
La loi de Coulomb dicte les interactions entre des corps chargés ponctuels. La force qui s'applique sur des charges électrique dépend de leur signe. Soit $q_1$ et $q_2$ deux charges électriques ponctuelles : 
\begin{itemize}
    \item Si $q_1 q_2>0$ (charges de même signe) $\rightarrow$ force répulsive
    \item Si $q_1 q_2<0$ (charges de même signe) $\rightarrow$ force attractive
\end{itemize}

\begin{bclogo}[logo=\bccrayon,noborder=true,barre=snake]{Exemple}
\begin{minipage}{0.3\textwidth}
\begin{tikzpicture}[
scale=1,
axis/.style={very thick, ->, >=stealth'},
important line/.style={thick},
dashed line/.style={dashed, thin},
pile/.style={thick, ->, >=stealth', shorten <=2pt, shorten
>=2pt},
every node/.style={color=black}
]
\draw (4,1.5) node{$\bullet$} ;
\draw (4,1.5) node[above]{$O_2$} ;
\draw (4,1.5) node[below]{$q_2$} ;
\draw (0,0) node{$\bullet$} ;
\draw (0,0) node[above]{$O_1$} ;
\draw (0,0) node[below]{$q_1$} ;
\draw [dashed] (0,0) -- (4,1.5) ;
\draw (2,0.75) node[below]{$r$} ;
\draw[line width=2pt,blue,-stealth](0,0)--(1,0.375) node[below, yshift=-1mm]{$\boldsymbol{\overrightarrow{u_{1\rightarrow 2}}}$};
\end{tikzpicture}
\end{minipage}
\begin{minipage}{0.7\textwidth}
$r=||\overrightarrow{O_1 O_2}||$

$K=(4\pi \epsilon_0)^{-1}=9*10^9 Nm^2C^{-2}$ ; $\epsilon_0 = (36\pi*10^9)^{-1} Fm^{-1}$

\textbf{$\overrightarrow{f_{1\rightarrow2}}=K\dfrac{q_1 q_2}{r^2}\overrightarrow{u_{12}}$} : force exercée par $q_1$ sur $q_2$
\end{minipage}
\end{bclogo}

\chapter{Magnétisme et Électromgnétisme}
\chapter{Forces électromagnétiques}
\chapter{Champ magnétique}
\chapter{Induction électromagnétique}

\begin{defi}
On appelle induction le phénomène qui conduit à l'apparition d'une force électromotrice engendrée par le magnétisme.
\end{defi}
\section{Force de Laplace}
\begin{defi}
La force de Laplace est la force électromagnétique qu'exerce un champ magnétique $\overrightarrow{B}$ sur un élément conducteur du circuit $\overrightarrow{dl}$ parcouru par une intensité $I$.
$$d\overrightarrow{F_{lap}}=I\overrightarrow{dl}\land\overrightarrow{B}$$
\end{defi}
\section{Loi de Faraday}
\begin{thm}[Loi de Faraday]
La variation temporelle du flux magnétique à travers un circuit fermé engendre une force électromotrice induite.
\begin{align*}
    & e=-\frac{d\phi}{dt} && \text{où}\ \phi=\iint\overrightarrow{B}\cdot\overrightarrow{d^2S}
\end{align*}
\end{thm}
\chapter{Équations de Maxwell}
\part{Phy-302 : Thermodynamique}
\chapter{Transformations thermodynamiques}
\section{Description d'un système thermodynamique}
\subsection{Systèmes thermodynamiques}
Afin d'amorcer une étude thermodynamique (tout comme n'importe quel domaine de la physique), il faut définir le système étudié.
\begin{defi}
On appelle système thermodynamique, l'ensemble des corps étudiés contenus dans un volume délimité par une enveloppe, réelle ou fictive.\\
On distingue donc un milieu intérieur (le système) et un milieu extérieur.
Selon les échanges que peut avoir le système avec un milieu extérieur on peut le qualifier de différents adjectifs.
\newline
\begin{itemize}
    \item On dit que le système est ouvert, s'il peut échanger avec le milieu extérieur de la matière et de l'énergie.
    \item On dit que le système est fermé, s'il peut échanger que de l'énergie avec le milieu extérieur.
    \item On dit que le système est isolé, s'il ne peut échanger ni énergie ni matière avec le milieu extérieur.
\end{itemize}
\end{defi}
\begin{ex}
\begin{itemize}
    \item Un verre d'eau est un système ouvert, il peut échanger de l'énergie et de la matière avec le milieu extérieur (évaporation, liquéfaction).
    \item Le circuit de refroidissement d'un réfrigérateur est un système fermé, il ne peut pas échanger de matière (liquide en circuit fermé) mais il peut échanger de l'énergie avec le milieu extérieur.
    \item L'univers est considéré comme un système isolé (il n'est pas censé avoir de milieu extérieur donc aucun échange n'est possible).
\end{itemize}
\end{ex}
\subsection{Grandeurs thermodynamiques et variables d'états}
\begin{defi}
Une variable d'état est une grandeur physique (mesurable) caractérisant l'état d'un système.
\end{defi}
On se sert de tels variables pour l'établissement d'équations d'état.\\
\newline
Ces variables peuvent-être qualifier :
\begin{itemize}
    \item d'extensive, c'est à dire une grandeur qui dépend de la "taille".\\
    Pour le dire autrement, une grandeur est extensive si pour deux systèmes disjoints, leur réunion est la somme de ces grandeurs.
    \item d'intensive, c'est a dire une grandeur qui peut être mesuré de manière ponctuelle, elle ne dépend pas de la "taille" du système. Ces grandeurs ne sont pas additives.
\end{itemize}
\begin{ex}
Le volume $V$, la masse $m$ et la quantité de matière $n$ sont des grandeurs extensives.\\
La température $T$ ou $\theta$ et la pression $P$ sont des grandeurs intensives
\end{ex}
A partir de ces grandeurs on définit l'équilibre d'un système thermodynamique.
\begin{defi}
Un système est à l'équilibre thermodynamique si toutes ses variables d'état sont invariantes dans le temps (constantes) sans transfert de matière ou d'énergie.
\end{defi}
\subsubsection{Pression}
\subsubsection{Température}
\subsection{Équations d'états}
\begin{defi}
On appelle équation d'état une relation mathématique entre les différentes variables d'état caractérisant le système.
\end{defi}
\subsection{Coefficients thermoélastiques}
\begin{defi}
    On définit trois grandeurs intensives nommées coefficients thermoélastiques.\\
    \begin{itemize}
        \item Coefficient de dilatation isobare : $\alpha = \frac{1}{V}\left(\frac{\partial V}{\partial T}\right)_P$
        \item Coefficient de compression isochore : $\beta = $
        \item Coefficient de compressibilité isotherme : $\chi_T = $
    \end{itemize} 
\end{defi}
L'intéret de tels coefficients est qu'il facilement accessible de manière expérimentale,
et permettent d'accéder très facilement a une équation d'état de n'importe quel matériau.
\section{Transformations}
\section{Représentations graphique}
\chapter{Premier principe}
\chapter{Second principe}
\chapter{Machines thermiques}
\chapter{Transitions de phase des corps purs}
\part{Annexes}
\chapter{Formulaire trigonométrie}
\chapter{Dérivées et intégrales usuelles}
\chapter{Développement limités usuelles}

\end{document}
