\chapter{Transformations thermodynamiques}
\section{Description d'un système thermodynamique}
\subsection{Systèmes thermodynamiques}
Afin d'amorcer une étude thermodynamique (tout comme n'importe quel domaine de la physique), il faut définir le système étudié.
\begin{defi}
On appelle système thermodynamique, l'ensemble des corps étudiés contenus dans un volume délimité par une enveloppe, réelle ou fictive.\\
On distingue donc un milieu intérieur (le système) et un milieu extérieur.
Selon les échanges que peut avoir le système avec un milieu extérieur on peut le qualifier de différents adjectifs.
\newline
\begin{itemize}
    \item On dit que le système est ouvert, s'il peut échanger avec le milieu extérieur de la matière et de l'énergie.
    \item On dit que le système est fermé, s'il peut échanger que de l'énergie avec le milieu extérieur.
    \item On dit que le système est isolé, s'il ne peut échanger ni énergie ni matière avec le milieu extérieur.
\end{itemize}
\end{defi}
\begin{ex}
\begin{itemize}
    \item Un verre d'eau est un système ouvert, il peut échanger de l'énergie et de la matière avec le milieu extérieur (évaporation, liquéfaction).
    \item Le circuit de refroidissement d'un réfrigérateur est un système fermé, il ne peut pas échanger de matière (liquide en circuit fermé) mais il peut échanger de l'énergie avec le milieu extérieur.
    \item L'univers est considéré comme un système isolé (il n'est pas censé avoir de milieu extérieur donc aucun échange n'est possible).
\end{itemize}
\end{ex}
\subsection{Grandeurs thermodynamiques et variables d'états}
\begin{defi}
Une variable d'état est une grandeur physique (mesurable) caractérisant l'état d'un système.
\end{defi}
On se sert de tels variables pour l'établissement d'équations d'état.\\
\newline
Ces variables peuvent-être qualifier :
\begin{itemize}
    \item d'extensive, c'est à dire une grandeur qui dépend de la "taille".\\
    Pour le dire autrement, une grandeur est extensive si pour deux systèmes disjoints, leur réunion est la somme de ces grandeurs.
    \item d'intensive, c'est a dire une grandeur qui peut être mesuré de manière ponctuelle, elle ne dépend pas de la "taille" du système. Ces grandeurs ne sont pas additives.
\end{itemize}
\begin{ex}
Le volume $V$, la masse $m$ et la quantité de matière $n$ sont des grandeurs extensives.\\
La température $T$ ou $\theta$ et la pression $P$ sont des grandeurs intensives\\
Par exemple, le mélange deux verres d'eau de $10$cL à une température de $20^\circ$C nous donne un volume de $20$cL (grandeur extensive) mais pas une température $40^\circ$C (grandeur intensive).

\end{ex}
A partir de ces grandeurs on définit l'équilibre d'un système thermodynamique.
\begin{defi}
Un système est à l'équilibre thermodynamique si toutes ses variables d'état sont invariantes dans le temps (constantes) sans transfert de matière ou d'énergie.
\end{defi}
\subsubsection{Pression}
\subsubsection{Température}
\subsection{Équations d'états}
\begin{defi}
On appelle équation d'état une relation mathématique entre les différentes variables d'état caractérisant le système.
\end{defi}
\subsection{Coefficients thermoélastiques}
\begin{defi}
    On définit trois grandeurs intensives nommées coefficients thermoélastiques.\\
    \begin{itemize}
        \item Coefficient de dilatation isobare : $\alpha = \displaystyle \frac{1}{V}\left(\frac{\partial V}{\partial T}\right)_P$
        \item Coefficient de compression isochore : $\beta = \displaystyle \frac{1}{P}\left(\frac{\partial P}{\partial T}\right)_P $
        \item Coefficient de compressibilité isotherme : $\chi_T = \displaystyle -\frac{1}{P}\left(\frac{\partial P}{\partial T}\right)_V$
    \end{itemize} 
\end{defi}
L'intérêt de tels coefficients est qu'ils sont facilement accessibles de manière expérimentale,
et permettent d'accéder très rapidement a une équation d'état de n'importe quel matériau.
\section{Transformations}
\begin{defi}
    On appelle transformation thermodynamique l'évolution d'un système d'un état d'équilibre initial a un état d'équilibre final.
\end{defi}
On peut qualifier une transformation thermodynamique de plusieurs adjectif, que sont les suivants :\\
\begin{itemize}
    \item Selon les caractéristiques du système :
    \begin{multicols}{2}
    \begin{itemize}
    \item isotherme : température $T$ constante
    \item isochore : volume $V$ constant
    \item isobare : pression $P$ constante
    \item isoénergétique : énergie interne $U$ constante
    \item isenthalpique : enthalpie $H$ constante
    \item isentropique : entropie $S$ constante
    \end{itemize}
    \end{multicols}
    \item Selon les caractéristiques du milieu extérieur :
    \begin{multicols}{2}
    \begin{itemize}
    \item monobare : pression extérieur constante
    \item monotherme : température extérieur constante
    \end{itemize}
    \end{multicols}
    \item Selon le type d'échange avec le milieu extérieur :
    \begin{multicols}{2}
    \begin{itemize}
        \item adiabatique : aucun transfert thermique $\delta Q =0$
        \item réversible : aucune création d'entropie $\sigma_s = 0$
    \end{itemize}
    \end{multicols}
\end{itemize}

\section{Modèles}
\subsection{Modèle du gaz parfait}
Au prix de quelques approximations, il est possible d'étudier un gaz avec le modèle le plus simple qui soit, celui du gaz parfait.
\begin{defi}
Dans le modèle du gaz parfait on considère que :
\begin{itemize}
    \item On suppose toutes les interactions internes du gaz (entre les molécule du gaz) sont négligeable .
\end{itemize}
Son équation d'état est :
$$PV=nRT$$
\begin{align*}
    \text{avec } & P\text{ : La pression du gaz en pascal, } Pa\\
    & V\text{ : Le volume du gaz en mètres cubes, } m^3\\
    & n\text{ : La quantité de matière contenue dans le gaz en mole, } mol\\
    & R\text{ : La constante d'état des gaz parfait qui vaut } 8,314\ J.mol^{-1}.K^{-1}\\
    & T\text{ : La température du gaz en kelvin, } K
\end{align*}
\end{defi}
\begin{rmq}
Tout gaz peut se comporter comme un gaz parfait s'il suffisamment dilué ( $V\to\infty$ ).\\ Ou de manière analogue si la pression est suffisamment faible ( $P\to 0$ ).
\end{rmq}
\subsection{Modèle de Van der Waals, gaz réel}
\begin{defi}
Son équation d'état est :
$$\left(P+n^2\frac{a}{V^2}\right)\left(V-nb\right)=nRT$$
\end{defi}
\subsection{Modèle}
\section{Représentations graphique}