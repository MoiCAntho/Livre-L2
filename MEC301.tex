\chapter{Torseurs}
Le torseur est un objet français (Cocorico !), permettant de représenter toutes les actions que subit un solide.\\
C'est un outils qui facilite les calculs et la formulation des lois.
En effet, il faudrait diviser les lois en plusieurs théorème pour complètement inclure les informations contenues dans un torseur.
\section{Actions mécaniques}
\section{Moment d'une force}
\begin{defi}
On appelle moment en $A$ de la force $\overrightarrow{F}$ passant par le point $P$, du solide $\Sigma$ :
$$\overrightarrow{M_{A}(\overrightarrow{F})}=\overrightarrow{AP}\land\overrightarrow{A}$$
\end{defi}
Un moment représente la capacité d'une force a crée une rotation autour d'un axe.\\
On peut définir la formule de transport des moments qui permet, connaissant le moment en un point $A$ du solide, de calculer le moments de cette force sur n'importe quel point du solide.
\begin{prop}
Soient $A$ et $B$ deux points de l'espace et $\overrightarrow{M_{A}}(\overrightarrow{F})$, $\overrightarrow{M_{B}}(\overrightarrow{F})$ leurs moments associés de la force $\overrightarrow{F}$ appliquée en $P$ au solide $\Sigma$.\\
On définit la formule de transport des moments :
$$\overrightarrow{M_{B}}(\overrightarrow{F})=\overrightarrow{M_{A}}(\overrightarrow{F})+\overrightarrow{BA}\land\overrightarrow{F}$$
\end{prop}
\begin{demo}
Soit
\end{demo}
\begin{meth}[Calcul d'un moment avec le bras de levier]
fr

\end{meth}
\section{Torseur \index{torseur} force}
\begin{defi}

$$\{\tau\}_{A}=\begin{Bmatrix}\overrightarrow{R}\\\overrightarrow{M_A}\end{Bmatrix}_A=\begin{Bmatrix} \sum_{1}^{n}\overrightarrow{F_i}\\\sum_{1}^{n}\overrightarrow{AP_i}\land\overrightarrow{F_i}\end{Bmatrix}_A=\begin{Bmatrix} R_x & R_y  & R_z\\ M_x & M_y & M_z\end{Bmatrix}_A$$
\end{defi}
\begin{rmq}
Dans le cas d'un point matériel le torseur force se réduit seulement à la résultante des forces.
\end{rmq}
\subsection{Éléments de réduction}
\section{Torseur de force répartie}
\section{Torseur cinématique}
\begin{defi}

$${\lbrace\chi\rbrace}_{A/R_0}=\begin{Bmatrix}\overrightarrow{\omega}_{\Sigma/R_0}\\\overrightarrow{V}_{\Sigma/R_0}(A)\end{Bmatrix}_{A/R_0}=\begin{Bmatrix} \omega_x & \omega_y  & \omega_z\\ u_A & v_A & w_A\end{Bmatrix}_{A/R_0}$$
\end{defi}
\section{Torseur de liaisons}
\section{Torseur déplacement infinitésimal}
\begin{defi}

$${\lbrace\delta\chi\rbrace}_{A/R_0}=\begin{Bmatrix}\delta\overrightarrow{\omega}_{\Sigma/R_0}\\\delta\overrightarrow{l}_{\Sigma/R_0}(A)\end{Bmatrix}_{A/R_0}=\begin{Bmatrix} \delta\omega_x & \delta\omega_y  & \delta\omega_z\\ \delta u_A & \delta v_A & \delta w_A\end{Bmatrix}_{A/R_0}$$
\end{defi}
\section{Opérations sur les torseurs}
\chapter{Statique du solide}
\section{Principe fondamentale de la statique}
\section{Principe des actions réciproques}
\section{Applications aux cas usuelles}
\section{Frottements secs}
\chapter{Cinématique du solide}
\section{Composition des vitesses}
\section{Composition des accélérations}
\chapter{Cinétique du solide}
En cinématique, les mouvements des corps sont considérés en omettant l'inertie des ces derniers.
En réalité, les mouvements des systèmes sont liés aux causes d'une part et à leurs inertie d'autre part.\\
\\
La notion d'inertie caractérise la propriété du système de changer plus ou moins rapidement sa vitesse sous l'effet des forces qui lui sont appliquées.
On simplifie souvent les choses, en réduisant l'inertie à la masse. Or le mouvement d'un solide ne dépend pas que de sa masse et des forces exercées sur ce dernier.
En effet il dépend également de sa géométrie, de la distribution de sa masse, ...
\\
\\
L'étude de l'inertie s'effectue dans le cadre de la cinétique.
\section{Géométrie des masses}
\subsection{Notion de masse}
\begin{defi}
A chaque solide $\Sigma$ est associée un quantité (scalaire), noté $m$, qui représente la matière contenue dans ce dernier. On la définie comme il suit :
$$m=\int dm$$
avec $dm$ l'élément infinitésimal de masse
\end{defi}
Sachant cela il est possible de définit la masse à partir des distributions de masse :
\begin{itemize}
    \item Masse volumique $\rho$ (en $kg.m^{-1}$)
    \item Masse surfacique $\sigma$ (en $kg.m^{-2}$)
    \item Masse linéique $\lambda$ (en $kg.m^{-1}$)
\end{itemize}
Par définition de ces trois grandeurs, on peut écrire l'élément infinitésimal de masse $dm$ comme :
\begin{itemize}
    \item $dm=\rho(M)dV$
    \item $dm=\sigma(M)dS$
    \item $dm=\lambda(M)dl$
\end{itemize}
où $dV$, $dS$ et $dl$ sont respectivement les éléments infinitésimaux de volume, de surface et le déplacement élémentaire autour d'un point M.\\
 \\
En définitive, on peut donc écrire :
\begin{itemize}
    \item $m=\int \rho(M)dV$ $\rightarrow$ Très utilisée pour les solides en 3 dimensions
    \item $m=\int \sigma(M)dS$ $\rightarrow$ Utilisée pour les plaques
    \item $m=\int \lambda(M)dl$ $\rightarrow$ Utilisée pour les tiges
\end{itemize}
\begin{ex}
...
\end{ex}
\subsection{Centre de Gravité et référentiel barycentrique}
\section{Torseur cinétique}
\section{Matrice d'inertie}