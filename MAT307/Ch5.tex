\chapter{Méthodes de résolutions explicites}
Nous allons dans ce chapitre explorer les différentes méthodes de résolution (selon le type d'équation) qui permettent de résoudre une équation différentielle ou un système différentielle de manière exacte.\\
Il est important de comprendre que nous ne connaissons aucune méthode de résolution générale, seulement des méthodes s'appliquant dans certains cas.
\section{Équations différentielles linéaires d'ordre $n$}
\begin{defi}
Une équation différentielle linéaire d'ordre est une équation de la forme :
$$a_n(t)y^{(n)}+a_{n-1}(t)y^{(n-1)}+\hdots+a_1(t)y^{\prime}+a_0(t)y+b(t)=0$$
\begin{align*}
    \text{où } & y \text{ est la fonction inconnue}\\
    & a_0(t),\hdots,a_n(t) \text{ sont des fonctions indépendantes de }y
\end{align*}
\end{defi}

\begin{prop}
On appelle polynôme caractéristique, le polynôme qui associe a chaque dérivées d'ordre n de l'équation homogène une variable $r$ à la puissance $n$.
$$a_ny^{(n)}+a_{n-1}y^{(n-1)}+\hdots+a_1y^{\prime}+a_0y+b(t)=0\Leftrightarrow a_nr^{n}+a_{n-1}y^{n-1}+\hdots+a_1r+a_0$$
\end{prop}
\begin{thm}[Théorème]
Si le polynôme caractéristique n'admet pour racine uniquement des racines simples $r_1,\hdots,r_n$ $\in\mathbb{C}$, l'ensemble des solutions est alors l'espace vectoriel engendré par $\{e^{r_1t},\hdots,e^{r_nt}\}$.
\end{thm}
\begin{thm}[Théorème]
Si le polynôme caractéristique $P(r)$ admet des racines multiples, pour chaque racine $r_k$ de multiplicité $m>1$
\end{thm}

\begin{ex}
\begin{itemize}
    \item Ordre 1 : $y^{\prime}-ty=0$
    \item Ordre 2 : $y^{\prime\prime}+2y^{\prime}+y=0$
    \item Ordre 5 : $y^{(5)}+y^{(4)}-4y^{(3)}-16y^{\prime\prime}-20y^{\prime}-12y=0$
\end{itemize}
\end{ex}




\section{Équations différentielles à variables séparables}
\begin{defi}
On dit qu'une équation différentielle est à variable séparées si l'on peut factoriser $f(y,t)$ en termes ne dépendant que de $y$ ou que de $t$.
$$y^{\prime}=f(y,t)=g(t)h(y)$$
\end{defi}
\begin{ex}
\begin{itemize}
    \item Soit $y^{\prime}(t)=t^2y(t)$ \\
    On peut factoriser par $g(t)=t^2$ et $h(y)=y$, on a donc bien $f(y,t)=g(t)h(y)$. L'équation est a variables séparables.
    \\
    \item Soit $y^{\prime}(t)y^2(t)=e^t$\\
    On peut la réécrire : $y^{\prime}(t)=\frac{e^t}{y^2(t)}$\\
    On peut factoriser par $g(t)=e^t$ et $h(y)=\frac{1}{y^2}$. L'équation est à variables séparables.\\
    \item L'équation $y^{\prime}(t)+y(t)-t^2=0$ n'est pas à variables séparables car l'on ne peut pas factoriser cette dernière.
\end{itemize}
\end{ex}
\begin{meth}[Résolution d'une équation différentielle à variables séparées]
\begin{enumerate}
    \item On vérifie, par l'intermédiaire de la définition, si l'équation est à variables séparables.
    \item On pose $\frac{y^{\prime}}{h(y)}=g(t)$.
    \item On intègre la quantité précédemment déterminer $\int \frac{y^{\prime}}{h(y)dt}=\int g(t)dt$.
    \item On effectue le changement de variable $u=y$ donc $du=y^{\prime}dt$, on obtient :\\ $\int\frac{du}{h(u)}=\int g(t) dt$.
    \item On obtient : $H(u)=G(t)+C$ où $H$ et $F$ sont respectivement les primitives de $f$ et $\frac{1}{g}$ et $C$ une constante.
    \item On cherche $H^{-1}$ la fonction réciproque $H$.
    \item On effectue $y(t)=u=H^{-1}(F(t)+C)$
\end{enumerate}
\end{meth}
\begin{ex}
\begin{itemize}
    \item Soit l'équation $(e^t+1)y^{\prime}+e^ty^2=0$.\\
    On souhaite savoir si l'équation est à variables séparables.\\
    On peut réécrire : $y^{\prime}=\frac{e^t}{e^t+1}y^2$ avec $g(t)=\frac{e^t}{e^t+1}$ et $h(y)=y^2$.\\
    L'équation est à variables séparables.\\
    On pose donc :\\
    $$\int\frac{y^{\prime}}{h(y)}\ dt=\int g(t)\ dt$$
    $$\Leftrightarrow\int\frac{y^{\prime}}{y^2}\ dt=\int \frac{-e^t}{e^t+1}\ dt$$
    $$\Leftrightarrow-\int -\frac{y^{\prime}}{y^2}\ dt=\int \frac{-e^t}{e^t+1}\ dt$$
    $$\Leftrightarrow-\frac{1}{y}+C_1= \ln(e^t+1) +C_2$$
    $$\Leftrightarrow y=\frac{1}{\ln(e^t+1)+C}\ \text{avec}\ C=C_2-C_1$$
    La solution de l'équation est donc : $y(t)=\frac{1}{\ln(e^t+1)+C}$
    \item Soit l'équation $y^{\prime}=e^ty^2(t)$.\\
    On voit facilement que l'équation est à variables séparables.\\
    On pose :
    \begin{align*}
        & \frac{y^{\prime}(t)}{y^2(t)}=e^t\\
        \Leftrightarrow & \int\frac{y^{\prime}(t)}{y^2(t)}=\int e^t\ dt\\
        \Leftrightarrow & \int\frac{du}{u^2}=\int e^t\ dt\ \text{avec } u=y\Rightarrow du=y^{\prime}(t)dt\\
        \Leftrightarrow & \frac{-1}{u}=e^t+C\\
    \end{align*}
\end{itemize}
\end{ex}
\section{Systèmes différentielles linéaires}
\subsection{Résolution par diagonalisation}
Dans un premier temps nous allons nous intéresser à la résolution des systèmes différentiels linéaire d'ordre 1 homogène.
\begin{thm}[Théorème]
Soit un système différentiel linéaire d'ordre 1 et sa représentation matriciel.
$$
\begin{cases}
y_1^{\prime}=a_1 y_1 + b(t)\\
\vdots\\
y_{n}^{\prime}=c_1 y_n + b_n(t)
\end{cases}\Leftrightarrow\ Y^{\prime}(t)=AY(t)+B(t)$$
On s'intéresse aux solutions homogènes ($B=0$) d'un tel système.
On pose $Y=PZ$, le système devient : $Z^{\prime}=DZ$
Où $D$ est la matrice diagonale semblable à $A$ 
 et $P$ la matrice de passage vers la base où $D$ existe.\\
Les solutions d'un tel système sont données par :
$$\begin{cases}
z_1=C_1e^{\lambda_1 t}\\
\vdots\\
z_n=C_ne^{\lambda_n t}
\end{cases}$$
Seulement nous cherchons les solutions de $Y$ et non de $Z$, par définition de $Z$ il suffit de poser :
$$Y=PZ$$
\end{thm}
\begin{demo}
Soit un système différentiel linéaire d'ordre 1 et sa représentation matriciel :
$$Y^{\prime}=AY$$
Par hypothèse, nous considérons $A$ diagonalisable, donc :
$$Y^{\prime}=PDP^{-1}$$
On pose $P^{-1}Y=Z$ on peut réécrire : 
$$PZ=PDZ$$
On dérive la quantité $PZ$, $P$ étant constante on obtient $PZ^{\prime}$, donc :
$$PZ^{\prime}=PDZ$$
La matrice $P$ est, par définition, inversible (matrice de passage), on peut donc simplifier par $P$.
$$Z^{\prime}=DZ$$
On obtient donc le système différentiel suivant :
$$\begin{cases}
z_1^{\prime}=\lambda_1 z_1\\
\vdots\\
z_n^{\prime}=\lambda_n z_n\\
\end{cases}$$
Et en résolvant, les équations une a une du système on obtient finalement :
$$\begin{cases}
z_1(t)=C_1 e^{\lambda_1 t}\\
\vdots\\
z_n(t)=C_n e^{\lambda_n t}\\
\end{cases}$$
Donc $z(t)=C_1 e^{\lambda_1 t}+\hdots+C_n e^{\lambda_n t}$
\end{demo}
\begin{ex}
Nous souhaitons résoudre le système différentiel suivant :
$$\begin{cases}
-x^{\prime}+x+3y = 0\\
-y^{\prime}+4x-3y = 0
\end{cases}$$
On commence par mettre le système sous forme résolue :
$$\begin{cases}
x^{\prime}=x+3y\\
y^{\prime}=4x-3y
\end{cases}$$
On cherche sa représentation matriciel, on pose $Y=\begin{pmatrix}x\\y\end{pmatrix}$.\\
On peut écrire :
$$Y^{\prime}=\begin{pmatrix}x^{\prime}\\y^{\prime}\end{pmatrix}=\begin{pmatrix}1&3\\4&-3\end{pmatrix}\begin{pmatrix}x\\y\end{pmatrix}$$
On pose : $Z=\begin{pmatrix}z_1\\z_2\end{pmatrix}=PV$.
On peut réécrire le système comme $Z^{\prime}=DZ$.\\
Partons à la recherche de la matrice diagonale $D$ et notamment des éléments propres de $A$.\\
Ici $P_A(\lambda)=(1-\lambda)(-3-\lambda)-12$, sa résolution nous renvoie : $\lambda_1=-5$ et $\lambda_2=3$.\\
On recherche les vecteurs propres associés aux valeurs propres $\lambda_1$ et $\lambda_2$.\\
La recherche des vecteurs et sous-espaces propres nous donne :\\
La matrice $A$ est diagonalisable, on en déduit les matrices $D$ et $P$ :
$$D=\begin{pmatrix}-5&0\\0&3\end{pmatrix}\ \text{et}\ P=\begin{pmatrix}-1&3\\2&2\end{pmatrix}$$
On peut donc réécrire le système comme :
$$Z^{\prime}=\begin{pmatrix}z_1^{\prime}\\z_2^{\prime}\end{pmatrix}=\begin{pmatrix}-5&0\\0&3\end{pmatrix}\begin{pmatrix}z_1\\z_2\end{pmatrix}$$
Il nous faut donc résoudre :
$$\begin{cases}
z_1^{\prime}=-5z_1\\
z_2^{\prime}=3z_2
\end{cases}$$
Par le théorème précédent, on déduit :
$$\begin{cases}
z_1(t)=C_1 e^{-5t}\\
z_2(t)=C_2 e^{3t}
\end{cases}$$
Pour trouver les solution pour $y$ il suffit de poser :
\begin{align*}
    \Leftrightarrow & Y=PZ\\
    \Leftrightarrow & Y=\begin{pmatrix}-1&3\\2&2\end{pmatrix}\begin{pmatrix}z_1\\z_2\end{pmatrix}\\
    \Leftrightarrow & Y=\begin{pmatrix}-z_1+3z_2\\2(z_1+z_2)\end{pmatrix}\\
    \Leftrightarrow & Y=\begin{pmatrix}-C_1e^{-5t}+3C_2e^{3t}\\2(C_1e^{-5t}+C_2e^{3t})\end{pmatrix}
\end{align*}
Les solutions du système sont donc :
$$\begin{cases}
x(t)=-C_1e^{-5t}+3C_1e^{3t}\\
y(t)=2(C_1e^{-5t}+C_2e^{3t})\\
\end{cases}$$
\end{ex}
On s'intéresse désormais aux solutions particulières de tels systèmes.
\begin{thm}[Théorème]
Soit un système différentiel linéaire d'ordre 1 et sa représentation matriciel.
$$
\begin{cases}
y_1^{\prime}=a_1 y_1 +\hdots+a_n y_n + b\\
\vdots\\
y_{n}^{\prime}=c_1 y_1 +\hdots+c_n y_n + b_n
\end{cases}\Leftrightarrow\ Y^{\prime}=AY+B$$
On pose $Y=PV$ où $P$ est la matrice de passage entre la base canonique et la base où se trouve la matrice diagonale.\\
Les solutions particulières d'un tel système sont données comme il suit :
$$V^{\prime}=DV+\underbrace{P^{-1}B}_{C}\ \Leftrightarrow\ 
\begin{cases}
v_1^{\prime}=\lambda_1 v+c_1\\
\vdots\\
v_{n}^{\prime}=\lambda_n +c_n
\end{cases}$$
Pour revenir sur la variable $y$ il suffit d'appliquer la matrice $P^{-1}$ sur le système défini par la variable $v$.
\end{thm}
\begin{demo}
Soit le système sous forme matricielle, $Y^{\prime}=AY+B$.\\
On considère le vecteur $V$ tel que $Y=PV$
\begin{align*}
    \Leftrightarrow & Y^{\prime}=AY+B\\
    \Leftrightarrow & (PV)^{\prime}=APV+B\\
    \Leftrightarrow & PV^{\prime}=APV+B\\
    \Leftrightarrow & V^{\prime}=\underbrace{P^{-1}AP}_{D}V+P^{-1}B\\
    \Leftrightarrow & V^{\prime}=DV+P^{-1}B
\end{align*}
\end{demo}
\begin{ex}
On considère le système différentiel :
$$\begin{cases}
-z_1^{\prime}+z_1-z_2=e^{t}\\
-z_2^{\prime}+2z_1+4z_2=0
\end{cases}$$
On isole la plus haute dérivée :
$$\begin{cases}
z_1^{\prime}=z_1-z_2+e^{t}\\
z_2^{\prime}=2z_1+4z_2
\end{cases}$$
On réécrit le système sous sa forme matricielle, pour ce faire on pose $Y=\begin{pmatrix}z_1\\z_2\end{pmatrix}$. On a donc :
$$Y^{\prime}=\begin{pmatrix}1&-1\\2&4\end{pmatrix}Y+\begin{pmatrix}e^t\\0\end{pmatrix}$$
On essaie de diagonaliser la matrice $A$.\\
On commence donc par rechercher les éléments propres.\\
Ici $P_A(\lambda)=(1-\lambda)(4-\lambda)-4$ et sa résolution nous donne $\lambda_1=3$ et $\lambda_2=2$.\\
La recherche des sous-espaces propres nous renvoie :
$$E_{\lambda_1}=\text{vect}\begin{pmatrix}\begin{pmatrix}1\\-2\end{pmatrix}\end{pmatrix}$$
$$E_{\lambda_2}=\text{vect}\begin{pmatrix}\begin{pmatrix}1\\-1\end{pmatrix}\end{pmatrix}$$
La matrice est diagonalisable. On déduit donc :\\
$$D=\begin{pmatrix}3&0\\0&2\end{pmatrix}\ \text{et}\ P=\begin{pmatrix}1&1\\-2&-1\end{pmatrix}$$
On en profite pour déterminer la matrice de passage inverse :
\begin{align*}
    \Leftrightarrow & \begin{pmatrix}[cc|cc]
   1 & 1 & 1 & 0 \\  
   -2 & -1 & 0 & 1
 \end{pmatrix}\\
 \Leftrightarrow & \begin{pmatrix}[cc|cc]1 & 1 & 1 & 0\\
 0&1&2&1\end{pmatrix}\\
 \Leftrightarrow & \begin{pmatrix}[cc|cc]1&0&-1&-1\\0&1&2&1\end{pmatrix}
\end{align*}
On déduit que $P^{-1}=\begin{pmatrix}-1&-1\\2&1\end{pmatrix}$.\\
On défini $V$ tel que $Y=PV$, on a donc $V'=DV+P^{-1}B$.\\
\begin{align*}
    \Leftrightarrow & V^{\prime}=DV+P^{-1}B \\
    \Leftrightarrow & V^{\prime}=\begin{pmatrix}3&0\\0&2\end{pmatrix}\times V +\begin{pmatrix}-1&-1\\2&1\end{pmatrix}\begin{pmatrix}-e^t\\0\end{pmatrix}\\
    \Leftrightarrow & V^{\prime}=\begin{pmatrix}3&0\\0&2\end{pmatrix}\times V +\begin{pmatrix}e^t\\-2e^t\end{pmatrix}\\
    \Leftrightarrow & \begin{cases}
    v_1^{\prime}=3v_1+e^t\\
    v_2^{\prime}=2v_2-2e^t
    \end{cases}
\end{align*}
On résout donc les deux équations indépendamment.
\begin{enumerate}
    \item Recherche de la solution générale de l'équation $v_1^{\prime}=3v_1+e^t\ (E)$.
    \begin{enumerate}
        \item Résolution de l'équation homogène $v_{1h}$ :\\
        On détermine l'équation homogène associée $(E_0)$ : $v_1^{\prime}=3v_1$.\\
        Par l'un des théorèmes précédent, on sait que les solutions de cette équation sont de la forme :
        $$v_{1h}(t)=C_1e^{3t}$$
        La fonction $v_{1h}=C_1e^{3t}$ est solution générale de l'équation $(E)$
        \item Recherche d'une solution particulière $v_{1p}$:\\
        On utilise la méthode de variation de la constante.
        Pour ce faire on pose $C_1=k(t)$, on a donc :
        $$v_1(t)=k(t)e^{3t}$$
        $$v_1^{\prime}=k^{\prime}(t)e^{3t}+3k(t)e^{3t}$$
        On peut réécrire $(E)$ de la manière suivante :
        \begin{align*}
            \Leftrightarrow & k^{\prime}(t)e^{3t}+3k(t)e^{3t}=3k(t)e^{3t}+e^t\\
            \Leftrightarrow & k^{\prime}(t)=e^{-2t}\\
            \Leftrightarrow & k(t) = \int e^{-2t}\ dt\\
            \Leftrightarrow & k(t) = \frac{1}{-2}\int e^{-2t}\\
            \Leftrightarrow & k(t) = \frac{1}{-2}e^{-2t}
        \end{align*}
        On injecte la fonction $k(t)$ dans la solution homogène :\\
        $$v_{1p}=\frac{e^{-t}}{-2}e^{3t}=\frac{e^t}{-2}$$
        La fonction $v_{1p}(t)$ est une solution particulière de l'équation $(E)$.
        \item Déduction de la solution générale de $(E)$ :
        Par le principe de superposition, on déduit la solution générale de l'équation :
        $$v_1(t)=v_{1h}+v_{1p}$$
        $$v_1(t)=C_1e^{3t}-\frac{e^t}{2}$$
    \end{enumerate}
    \item Recherche de la solution générale $v_2^{\prime}=2v_2-2e^t$ :\\
    \begin{enumerate}
        \item Résolution de l'équation homogène associée $(E_0)$ :\\
        Par le même raisonnement que précédemment, on trouve que :
        $$v_{1h}(t)=C_2e^{2t}$$
        \item Recherche d'une solution particulière :\\
        On réutilise la méthode de variation de la constante.
        On pose $C_2=k(t)$, alors :
        $$v_2=k(t)e^{2t}$$
        $$v_2^{\prime}=k^{\prime}(t)e^{2t}+2k(t)e^{2t}$$
        On a donc :
        \begin{align*}
            \Leftrightarrow & k^{\prime}(t)e^{2t}+2k(t)e^{2t}=2k(t)e^{2t}-2e^t\\
            \Leftrightarrow & k^{\prime}(t)=-2e^{-t}\\
            \Leftrightarrow & k(t) = \int -2e^{-t}\ dt\\
            \Leftrightarrow & k(t) = -2\int e^{-t}\\
            \Leftrightarrow & k(t) = 2e^{-t}
        \end{align*}
        On injecte la fonction $k(t)$ dans la solution homogène :
        $$v_{1p}=2e^{-t}e^{2t}=2e^{t}$$
        La fonction $v_{1p}$ est solution particulière de l'équation $(E)$.
        \item Déduction de la solution générale :
        Par le principe de superposition, on a :
        $$v_2(t)=C_2e^{2t}+2e^t$$
    \end{enumerate}
\end{enumerate}
On peut alors réécrire le système précédent comme :
$$\begin{cases}
v_1(t)=C_1e^{3t}-\frac{e^t}{2}\\
v_2(t)=C_2e^{2t}+2e^t
\end{cases}$$
Par la définition du vecteur $V$ on peut écrire :
\begin{align*}
    \Leftrightarrow & Y=PV\\
    \Leftrightarrow & Y=\begin{pmatrix}1&1\\-2&-1\end{pmatrix}\begin{pmatrix}v_1\\v_2\end{pmatrix}\\
    \Leftrightarrow & Y=\begin{pmatrix}v_1 +v_2\\-2v_1-v_2\end{pmatrix}\\
    \Leftrightarrow & Y=\begin{pmatrix}C_1e^{3t}-\frac{e^t}{2}+C_2e^{2t}+2e^t\\-2C_1e^{3t}+e^t-C_2e^{2t}-2e^t\end{pmatrix}
\end{align*}
Le système a donc pour solutions :
$$\begin{cases}
z_1(t)=C_1e^{3t}-\frac{e^t}{2}+C_2e^{2t}+2e^t\\
z_2(t)=-2C_1e^{3t}+e^t-C_2e^{2t}-2e^t
\end{cases}$$
\end{ex}
\subsection{Comportement des solutions selon les valeurs propres}
\subsection{Tracés de systèmes différentiels}
\begin{ex}
On reprend l'exemple de la résolution d'un système homogène. On a le système suivant, ses solutions et ses conditions :
$$\begin{array}{ccc}
    \begin{cases}x^{\prime}=x+3y\\y^{\prime}=4x-3y\end{cases} & \begin{cases}x(t)=-C_1e^{-5t}+3C_2e^{3t}\\y(t)=2(C_1e^{-5t}+C_2e^{3t})\\\end{cases} & \text{les conditions}\ x(0)=a\ \text{et}\ y(0)=b\\
\end{array}$$
On commence par déterminer les constantes $C_1=\mu$ et $C_2=\nu$ en fonction des conditions initiales :
\begin{align*}
    \Leftrightarrow & \begin{cases}x(0)=-\mu e^{-5t}+3\nu e^{3t}\\y(0)=2(\mu e^{-5t}+\nu e^{3t})\end{cases}\\
    \Leftrightarrow & \begin{cases}-\mu e^{-5t}+3\nu e^{3t}=a\\2(\mu e^{-5t}+\nu e^{3t})=b\end{cases}\\
    \Leftrightarrow & \begin{cases}-\mu +3\nu=a\\2\mu+2\nu=b\end{cases}\\
    \Leftrightarrow & \begin{cases}-\mu+3\nu=a\\8\nu=b+2a\end{cases}\\
    \Leftrightarrow & \begin{cases}-\mu+3\nu=a\\\nu = \frac{b+2a}{8}\end{cases}\\
    \Leftrightarrow & \begin{cases}-\mu =a-\frac{3b-6a}{8}\\\nu=\frac{b+2a}{8}\end{cases}\\
    \Leftrightarrow & \begin{cases}\mu = \frac{3b+2a}{8}\\\nu=\frac{b+2a}{8}
    \end{cases}
\end{align*}
On peut donc réécrire les solutions :
$$\begin{cases}
x(t)=\left( \frac{-3b-2a}{8} \right)e^{-5t}+\left( \frac{3b+6a}{8} \right)e^{3t}\\
y(t)=\left(\frac{6b+2a}{8}\right)e^{-5t}+\left(\frac{2b+4a}{8}\right)e^{3t}
\end{cases}$$
$$\Leftrightarrow\ \begin{cases}x(t)=\frac{1}{8}\left(\left(-3b-2a\right)e^{-5t}+3\left(b+2a\right)e^{3t}\right)\\y(t)=\frac{1}{4}\left(\left(3b+a\right)e^{-5t}+\left(b+2a\right)e^{3t}\right)\end{cases}$$
\end{ex}
\section{Intégrales premières ou constantes du mouvement}
Lorsqu'il n'est pas possible de résoudre de manière explicite une équation ou un système différentielle, il peut être judicieux de passer par une intégrale première (autrement appelée constante du mouvement.
\begin{defi}
Pour un système autonome, la fonction $V$ est une intégrale première si et seulement si la quantité $\overrightarrow{\nabla}V\times f=0$
\end{defi}



\begin{ex}
\begin{itemize}
    \item Équations de prédation de Lokta-Volterra :\\
    Il s'agit d'un couple d'équation différentielle, décrivant l'évolution des proies et des prédateurs présent dans un milieu.
    On pose $x(t)$ le nombre de proie et $y(t)$ le nombre de prédateurs.
    On suppose que le nombre de proie est linéairement proportionnel au nombre de prédateur et réciproquement.
    $$\begin{cases}
    x^{\prime}(t)=ax(t)-bx(t)y(t)\\
    y^{\prime}(t)=-cy(t)+dy(t)x(t)
    \end{cases}$$
    Où les constantes $a$, $b$, $c$ et $d$ sont positives dont chacune représente respectivement :
    \begin{itemize}
        \item la natalité naturelle des proies
        \item la mortalité liée aux prédateurs des proies
        \item la mortalité naturelle des prédateurs
        \item la natalité des prédateurs liée au nombre de prédateur
    \end{itemize}
    On constate que le système n'est pas linéaire, on ne peut donc pas résoudre le système par les méthodes présentés précédemment.
    On donne la constante du mouvement :\\
    $$V(x,y)=dx-c\ln(x)+by-a\ln(y)$$
    Les solutions du systèmes sont donc des courbes de niveaux de $V(x,y)$
\end{itemize}
\end{ex}
\section{Utilisation des formes différentielles}
\section{Autres méthodes}


\section{Comportement des solutions}
\subsection{Comportement asymptotique}
\subsection{Comportement selon les valeurs propres}














Il est nécessaire pour la résolution des systèmes de les "convertir" en problèmes de Cauchy.
\begin{prop}
On peut réécrire une système différentiel sous une forme matriciel.
\begin{itemize}
    \item Pour un système d'ordre  1 : $$Y^{\prime}=AY+B$$
    \item Pour un système d'ordre n :$$Y^{(n)}=AY+B$$
    où $Y$ est le vecteur contenant toutes les dérivées d'ordre inférieur à $n$ :
    $$Y=\begin{pmatrix}y_1^{\prime}\\\vdots\\y_k^{\prime}\\\vdots\\y_1^{(n)}\\\vdots\\y_k^{(n)}\end{pmatrix}$$
\end{itemize}
\end{prop}
\begin{ex}
On considère le système :\\
\begin{center}
    $\begin{cases}\dot x = x-y+e^t\\\dot y = 2x+4y\end{cases}$ Ici :  $Y=\begin{pmatrix} x \\ y\end{pmatrix}$ , $A=\begin{pmatrix}1&-1\\2&4\end{pmatrix}$ et $b(t)=\begin{pmatrix}e^t\\0\end{pmatrix}$.
\end{center}.
On peut donc écrire : $\dot Y = \begin{pmatrix}1&-1\\2&4\end{pmatrix}Y+\begin{pmatrix}e^t\\0\end{pmatrix}$\\
Soit le système :\\
\begin{center}
    $\begin{cases}x^{\prime\prime}=-2\omega^2x+\omega^2y\\y^{\prime\prime}=\omega^2x-2\omega^2y\end{cases}$
\end{center}
On pose donc le vecteur $Y$ contenant toutes les dérivées d'ordres inférieurs à $2$.\\
On a donc $Y=(x,y,x^{\prime},y^{\prime})$.
Puis on fait sorte de retrouver le même système par multiplication. On obtient :
$$A=\begin{pmatrix}0&0&1&0\\0&0&0&1\\-2\omega^2&\omega^2&0&0\\\omega^2&-2\omega^2&0&0\end{pmatrix}$$
Donc on peut écrire : $Y^{\prime}=\begin{pmatrix}x^{\prime}\\y^{\prime}\\x^{\prime\prime}\\y^{\prime\prime}\end{pmatrix}=\begin{pmatrix}0&0&1&0\\0&0&0&1\\-2\omega^2&\omega^2&0&0\\\omega^2&-2\omega^2&0&0\end{pmatrix}\begin{pmatrix}x\\y\\x^{\prime}\\y^{\prime}\end{pmatrix}$
\end{ex}
\section{Systèmes différentiels linéaires d'ordre 1}

\subsubsection{Diagonalisation dans $\mathbb{C}$}
\subsection{Résolution par triangulation}
