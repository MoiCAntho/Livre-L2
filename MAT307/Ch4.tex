\chapter{Généralités}

\begin{defi}
Une équation différentielle est une équation reliant une fonction et ses dérivées successives.\\
L'inconnue est donc une fonction.\\
On peut l'écrire de façon très générale sous la forme $F(x,y,y^{\prime},y^{\prime\prime},\hdots,y^{(n)})=0$ où $y(x)$ est la fonction inconnue.
\end{defi}
\begin{defi}
On appelle système différentiel un ensemble d'équations différentielles, qui ne peuvent être résolues indépendamment les unes des autres.\\
Il se présente sous la forme :
$$\begin{cases}
a_n y_1^{(n)}+a_{n-1}y_1^{(n-1)}(t)+\hdots+a_2 y_1^{\prime\prime}(t)+a_1 y_1^{\prime}(t)+a_0y_1(t) = b_1(t)\\
\vdots \\
c_n y_k^{(n)} + c_{n-1}y_k^{(n-1)}(t)+\hdots+c_2 y_k^{\prime\prime}(t)+c_1 y_k^{\prime}(t)+c_0y_k(t) = b_k(t)
\end{cases}$$
où les $y_k$ sont les fonctions inconnues, $a_n$ et $c_n$ des coefficients et $b_k(t)$ représente les second membres des équations.
\end{defi}

Courbe intégrale