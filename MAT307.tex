\part{Courbes}
\chapter{Courbes paramétrées}
La trajectoire d'un corps dans un plan est déterminé par le couple de coordonnées $(x,y)$ dépendant du temps $t$, c'est une équation paramétrique.
\begin{defi}
Soient $f$ et $g$ deux fonctions définies sur $I\subseteq
\mathbb{R}$.\\
Le point $M(t)$ de coordonnées $(f(t),g(t))$ décrit une courbe du plan $C$ appelée courbe paramétrée (de paramètre $t$).
La fonction de $I$ sur $(C)$ qui à $t$ associe $M(t)$ est un paramétrage de $(C)$.\\
Les équations $\begin{cases}x=f(t)\\y=g(t)\end{cases}$ définissent une représentation paramétrique de $\mathscr{C}$.\\
Notations : $(x=x(t),y=y(t))$ ou $t\mapsto(x(t),y(t))$
\end{defi}
Nous étudierons les propriétés des courbes paramétrées, qui peuvent être de deux natures :
\begin{itemize}
    \item Cinématique : dépendantes du paramètre $t$.\\
    ex : vitesse, accélération, ...
    \item Géométrique : indépendante du paramètre $t$.\\
    ex : tangentes, ...
\end{itemize}
\begin{rmq}
Par convention, on nomme le paramètre $t$ le temps, bien que ce dernier peut n'avoir aucun rapport avec ce dernier.\\
De même les vecteurs correspondants aux dérivées première et seconde, sont appelés respectivement vitesse et accélération.
\end{rmq}
\section{Paramétrage et représentation graphique}
La paramétrisation d'une courbe n'est jamais unique et il est possible de passer d'un paramétrage a l'autre.
\section{Étude analytique d'une courbe paramétrée}
\subsection{Domaine de définition et intervalle d'étude}
\begin{defi}
Le domaine de définition $I$ du paramétrage est l'intersection des domaines de définition des fonctions $x(t)$ et $y(t)$.
$$\mathscr{D}_\mathscr{C}=\mathscr{D}_x\cap\mathscr{D}_y$$
\end{defi}
Une fois le domaine de définition déterminée, on cherche à réduire le domaine de définition à un intervalle d'étude afin de simplifier l'étude.
Pour ce faire, on utilise les propriétés de périodicité et de parités.
\begin{prop}
Soit la courbe $\mathscr{C}$ défini par $(x(t),y(t))$.\\
On étudie les propriétés de parités de chacune des coordonnées.
\begin{itemize}
    \item Si $x$ et $y$ sont impaires,
    \item Si $x$ est impaire et $y$ est paire,
    \item Si $x$ et $y$ sont paires,
    \item Si $x$ est paire et $y$ est impaire,
\end{itemize}
\end{prop}
\subsection{Étude des branches infinies}
Une fois l'intervalle d'étude
\subsection{Étude locale et points singuliers}
On commence dans un premier temps par  définir le vecteur $\overrightarrow{OM(t)}$ (autrement appelé $\overrightarrow{M(t)}$), qui est défini par $\begin{pmatrix}x(t)\\y(t)\end{pmatrix}$.\\
Afin d'amorcer une étude locale d'une courbe paramétrée il convient de dériver ce vecteur, on a donc : $\overrightarrow{M^{\prime}(t)}=\begin{pmatrix}x^{\prime}(t)\\y^{\prime}(t)\end{pmatrix}$, on obtient le vecteur vitesse.\\
Tout comme l'étude d'une fonction, tout ce qu'il de plus classique, l'essentiel de l'étude déroule au niveau des points d'annulations de ces dérivées.
\subsubsection{Tangentes}
\begin{prop}
Soit $\mathscr{C}$ une courbe paramétrée par ses coordonnées $x(t)$ et $y(t)$.\\
Nous considérons leurs dérivées, respectivement $x^{\prime}(t)$ et $y^{\prime}(t)$.
\begin{itemize}
    \item Si $x^{\prime}(t_0)=0$ et $y^{\prime}(t_0)\neq0$, la courbe admettra une tangente verticale en $t=t_0$.
    \item Si $x^{\prime}(t_0)\neq0$ et $y^{\prime}(t_0)=0$, la courbe admettra une tangente horizontale en $t=t_0$.
\end{itemize}
\end{prop}
On peut se convaincre assez facilement de la direction des tangentes à l'aide du petit raisonnement suivant.\\
En effet, si $x^{\prime}(t_0)=0$, la dérivée est entièrement portée par le vecteur unitaire $\overrightarrow{y}$, de ce fait la tangente ne peut être que verticale, et réciproquement.
\begin{ex}
On considère la courbe $\mathscr{C}$ définie par :
$$\begin{array}{cccc}
    \mathscr{C} \ : & \mathscr{D}_{\mathscr{C}} & \to & \mathbb{R} \\
         & t & \mapsto & \begin{cases}x(t)=\frac{4t^2-1}{t^3+1}\\y(t)=\frac{4t^3-t}{t^3+1}\end{cases}
\end{array}$$
\begin{minipage}{0.6\linewidth}
On dérive la composante $x(t)$ :
$$x^{\prime}(t)=\frac{x(-4x^3+3x+8)}{(t^3+1)^2}$$
On résout l'équation $x^{\prime}(t)=0$.\\
On obtient $S=\{0;1,46\}$\\
La courbe admet deux tangentes verticales.\\
\\
On dérive la composante $y(t)$ :

\end{minipage}
\begin{minipage}{0.4\linewidth}
\begin{pspicture*}(-3, -3)(3,5)
\psgrid[subgriddiv=0,griddots=10,gridlabels=7pt,gridcolor=gray]
\parametricplot[plotstyle=curve,algebraic,linecolor=red]{-5}{5}{(4*t^2-1)/(t^3+1) | (4*t^3-t)/(t^3+1) }
\psaxes[ticks=none,labels=none]{->}(0,0)(-3,-3)(3,5)
\end{pspicture*}
\end{minipage}
\end{ex}

\subsubsection{Points réguliers et singuliers}
\begin{defi}
Soit $\mathscr{C}$ une courbe paramétrée par ses coordonnées $x(t)$ et $y(t)$.
Nous considérons le vecteur vitesse $\overrightarrow{v(t)}$ du point $M(t)$.
\begin{itemize}
    \item Si $v(t_0)\neq\overrightarrow{0}$ alors la courbe admet en $t=t_0$ un point régulier.\\
    \item Si $v(t_0)=\overrightarrow{0}$ alors la courbe admet en $t=t_0$ un point singulier.
\end{itemize}
\end{defi}
Il est nécessaire de bien comprendre ce que sont ces deux points.
Un point régulier est "un point normal de la courbe", c'est à dire que la courbe est tangente au vecteur vitesse $\overrightarrow{v(t)}$.\\
Un point singuliers quant à lui est un point très particulier qui peut être de plusieurs natures.
Néanmoins, son étude locale est, par définition, impossible en se cantonnant uniquement au vecteur vitesse.
\\
\\
Pour remédier à cela, on va faire un développement limité en $t_0$, au minimum à l'ordre 3.
On rappel la formule de Taylor, pour une fonction $f$ en un point $x_0$ à l'ordre $n$ : $$P_n(x)=\sum_{i=1}^{n}f^{(i)}(x_0)\frac{(x-x_0)^{i}}{i!}$$
On calcul les développements limités des deux coordonnées.
$$x(t)=x(t_0)+v_x(t_0)(t-t_0)+x^{\prime\prime}(t_0)\frac{(t-t_0)^2}{2!}+x^{(3)}(t_0)\frac{(t-t_0)^3}{3!}+...+o((t-t_0)^n)$$
et
$$y(t)=y(t_0)+v_y(t_0)(t-t_0)+y^{\prime\prime}(t_0)\frac{(t-t_0)^2}{2!}+y^{(3)}(t_0)\frac{(t-t_0)^3}{3!}+...+o((t-t_0)^n)$$
Puis on rassemble ces deux développement limités en un vecteur :
$$\overrightarrow{M(t)}=\begin{pmatrix}x(t_0)\\y(t_0\end{pmatrix}+\begin{pmatrix}v_x(t_0)\\v_y(t_0)\end{pmatrix}(t-t_0)+\begin{pmatrix}x^{\prime\prime}(t_0)\\y^{\prime\prime}(t_0)\end{pmatrix}\frac{(t-t_0)^2}{2}+\begin{pmatrix}x^{(3)}(t_0)\\y^{(3)}(t_0)\end{pmatrix}\frac{(t-t_0)^3}{6}+...+o(\|\overrightarrow{M(t)}\|^n)$$
Or par définition d'un point singulier ($\overrightarrow{v(t_0)}=\overrightarrow{0}$), on a :
$$\overrightarrow{M(t)}=\begin{pmatrix}x(t_0)\\y(t_0)\end{pmatrix}+\begin{pmatrix}x^{\prime\prime}(t_0)\\y^{\prime\prime}(t_0)\end{pmatrix}\frac{(t-t_0)^2}{2}+\begin{pmatrix}x^{(3)}(t_0)\\y^{(3)}(t_0)\end{pmatrix}\frac{(t-t_0)^3}{6}+...+o(\|\overrightarrow{M(t)}\|^n)$$
Une fois cela fait nous cherchons les deux premiers vecteurs (associées a des degrés supérieurs à $2$ dans le développement limité), non colinéaires.
Ces deux vecteurs nous permettrons de définir la nature du point singulier, ainsi que le sens de parcours de la courbe à travers celui-ci.

\begin{prop}
Soit une courbe $\mathscr{C}$ ayant un point singulier au point $t=t_0$.\\
Le développement limité de la courbe $\mathscr{C}$ au point $t=t_0$ à l'ordre $n$ est le suivant :
$$\overrightarrow{M(t)}=\begin{pmatrix}x(t_0)\\y(t_0)\end{pmatrix}+...+\underbrace{\begin{pmatrix}x^{(p)}(t_0)\\y^{(p)}(t_0)\end{pmatrix}}_{\overrightarrow{l}}\frac{(t-t_0)^p}{p!}+...+\underbrace{\begin{pmatrix}x^{(q)}(t_0)\\y^{(q)}(t_0)\end{pmatrix}}_{\overrightarrow{m}}\frac{(t-t_0)^q}{q!}+...+o(\|\overrightarrow{M(t)}\|^n)$$
Avec les vecteurs $\overrightarrow{l}$ et $\overrightarrow{m}$, les deux premiers vecteurs non colinéaires.\\
La nature du point singulier est donnée par les critères suivants :
\begin{itemize}
    \item Si $p$ est impair et $q$ est pair, alors il s'agit d'un point régulier.
    \item Si $p$ est impair et $q$ est impair, alors il s'agit d'un point d'inflexion.
    \item Si $p$ est pair et $q$ est impair, alors il s'agit d'un point de rebroussement de $1^{\text{ère}}$ espèce.
    \item Si $p$ est pair et $q$ est pair, alors il s'agit d'un point de rebroussement de $2^{\text{nde}}$ espèce.
\end{itemize}
\end{prop}
propriétés sur le sens de parcours demander à la prof

Images à faires
\begin{ex}
...
\end{ex}

\subsubsection{Convexité}
Dans le cas d'un point d'inflexion, il peut-être utile de chercher si avant et après lui la courbe est convexe ou concave.
\subsection{Tableau de variation}
\subsection{Applications}
\begin{meth}
Soit une courbe $\mathscr{C}$, voici le déroulement de son étude :
\begin{enumerate}
    \item Détermination de son ensemble de définition
    \item Étude de la périodicité et des symétries pour un éventuelle réduction de l'intervalle d'étude.
    \item Étude des limites et des branches infinies, déterminer les asymptotes.
    \item Étude locale, recherche des tangentes et des possibles points de rebroussements et d'inflexion.
    \item Dressage du tableau de variation.
    \item Dessin de la courbe $\mathscr{C}$.
\end{enumerate}
\end{meth}
\begin{ex}

On considère la courbe $\mathscr{C}$ définie par :
$$\begin{array}{cccc}
    \mathscr{C} \ : & \mathscr{D}_{\mathscr{C}} & \to & \mathbb{R} \\
         & t & \mapsto & \begin{cases}x(t)=\frac{2t}{1+t^2}\\y(t)=\frac{2+t^3}{1+t^2}\end{cases}
\end{array}$$
On commence dans un premier temps par définir l'ensemble de définition.
Dans notre cas, le dénominateur est commun aux deux fonctions.\\
On cherche les valeurs interdites du dénominateur:
\begin{align*}
    & 1+t^2=0\\
    \Leftrightarrow & t^2=-1\\
    \Leftrightarrow & S=\{\varnothing\}
\end{align*}
En effet, une racine carrée ne pouvant être négative dans $\mathbb{R}$, les fonctions n'admettent aucune valeur interdite.
On obtient : $\mathscr{D}_\mathscr{C}=\mathbb{R}$

\begin{center}
\begin{pspicture*}(-3, -4)(3,5)
\psgrid[subgriddiv=0,griddots=10,gridlabels=7pt,gridcolor=gray]
\parametricplot[plotstyle=curve,algebraic,linecolor=red]{-5}{5}{2*t/(1+t^2) | (2+t^3)/(1+t^2) }
\psaxes[ticks=none,labels=none]{->}(0,0)(-3,-4)(3,5)
\end{pspicture*}
\end{center}
\end{ex}
\section{Courbes en polaire}
Cette section se concentrera sur l'étude des fonctions défini
\begin{defi}
Une courbe en polaire est une courbe paramétrée par :
$$\begin{array}{cccc}
    \mathscr{C} \ : & \mathscr{D}_{f} & \to & \mathbb{R} \\
         & \theta & \mapsto & r(\theta)
\end{array}$$
où $r(\theta)$ est la distance algébrique du point $M$ à l'origine.\\
Autrement dit, $\overrightarrow{OM(\theta)}=r(\theta)\overrightarrow{u_r(\theta)}=r(\theta)\begin{pmatrix}\cos\theta\\\sin\theta\end{pmatrix}$
\end{defi}
Le fait que $r(\theta)$ soit une distance algébrique traduit le fait que ...\\
Le point $M(\theta)$ est donc bien à une distance $|r(\theta)|$ mais dans la direction $-\overrightarrow{u_r(\theta)}$.
\begin{prop}
Soit $\mathscr{C}$ une courbe en polaire, un paramétrage de la courbe $\mathscr{C}$ serait :
$$r(\theta)\Leftrightarrow\begin{cases}x(\theta)=r(\theta)\cos(\theta)\\y(\theta)=r(\theta)\sin(\theta)\end{cases}$$
\end{prop}
\subsection{Domaine de définition et intervalle d'étude}
Périodicité :
Si la période n'est pas multiple de $2\pi$, alors il faut faire des rotations pour déterminer la courbe dans son ensemble.
Symétries :

\subsection{Étude des branches infinies}
\subsection{Étude locale}
\subsection{Tableau de variation}
\subsection{Applications}
\section{Coniques}
\chapter{Propriétés métrique des courbes}
\chapter{Intégrales curvilignes}
\part{Équations différentielles}
\chapter{1}
\chapter{Méthodes numérique pour les équations différentielles}
\chapter{Méthodes explicite pour les équations différentielles}
