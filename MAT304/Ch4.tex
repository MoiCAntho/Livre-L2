\chapter{Opérateurs différentielles}
Attention, l'expression des opérateurs changent selon le système de coordonnées.
\section{Le gradient $\protect\overrightarrow{\nabla}$}
Nous avons déjà définit le gradient d'une fonction évalué en un point, on va maintenant définir l'opérateur gradient.
\begin{defi}
On appelle gradient, l'opérateur définit comme il suit :
$$\overrightarrow{\nabla}=\begin{pmatrix}\frac{\partial}{\partial x_1}\\\vdots\\\frac{\partial}{\partial x_n}\end{pmatrix}$$
\newline
Le gradient d'une fonction scalaire $f:\mathbb{R}^n\to\mathbb{R}$ est donc :
$$\overrightarrow{\nabla}f=\begin{pmatrix}\frac{\partial}{\partial x_1}\\\vdots\\\frac{\partial}{\partial x_n}\end{pmatrix}\times f=\begin{pmatrix}\frac{\partial f}{\partial x_1}\\\vdots\\\frac{\partial f}{\partial x_n}\end{pmatrix}=\overrightarrow{grad(\overrightarrow{f})}$$
\end{defi}
\begin{prop}
Soit un champ scalaire $\overrightarrow{f}$ dans l'espace $\mathbb{R}^3$, nous avons pour expression du gradient :
\begin{itemize}
    \item En coordonnées cartésiennes (par définition) : $$\overrightarrow{\nabla}f=\frac{\partial f}{\partial x}\overrightarrow{u_x}+\frac{\partial f}{\partial y}\overrightarrow{u_y}+\frac{\partial f}{\partial z}\overrightarrow{u_z}$$
    \item En coordonnées cylindriques : $$\overrightarrow{\nabla}f=\frac{1}{r}\frac{\partial f}{\partial r}\overrightarrow{u_r}+\frac{\partial f}{\partial \theta}\overrightarrow{u_\theta}+\frac{\partial f}{\partial z}\overrightarrow{u_z}$$
    \item En coordonnées sphériques : $$\overrightarrow{\nabla}f=\frac{\partial r}{\partial x}\overrightarrow{u_r}+\frac{1}{r}\frac{\partial f}{\partial \theta}\overrightarrow{u_\theta}+\frac{1}{r\sin{\theta}}\frac{\partial f}{\partial \phi}\overrightarrow{u_\phi}$$
\end{itemize}
\end{prop}
\begin{demo}
soit
\end{demo}
\section{La divergence $div$}
\begin{defi}
Soit un champ de vecteurs $\overrightarrow{f}:\mathbb{R}^n\mapsto\mathbb{R}^n$ de classe $\mathscr{C}^1$ sur $\mathscr{D}_f$.\\
La divergence de $\overrightarrow{f}$ est donné par :
$$div(\overrightarrow{f})=\overrightarrow{\nabla}.\overrightarrow{f}=\sum_{i=1}^n \frac{\partial f_i}{\partial x_i}$$
\end{defi}
\begin{rmq}
La divergence est également égale à la trace de la matrice jacobienne.
$$div(\overrightarrow{f})=\text{Tr}(J_f)$$
\end{rmq}
\begin{prop}
Soit un champ scalaire $\overrightarrow{f}$ dans l'espace $\mathbb{R}^3$, nous avons pour expression de la divergence :
\begin{itemize}
    \item En coordonnées cartésiennes (par définition) : $$div(\overrightarrow{f})=\frac{\partial f_x}{\partial x}+\frac{\partial f_y}{\partial y}+\frac{\partial f_z}{\partial z}$$
    \item En coordonnées cylindriques : $$div(\overrightarrow{f})=\frac{1}{r}\frac{\partial (r f_r)}{\partial r}+\frac{1}{r}\frac{\partial f_{\theta}}{\partial \theta}+\frac{\partial f_z}{\partial z}$$
    \item En coordonnées sphériques : $$\overrightarrow{\nabla}f=\frac{\partial r}{\partial x}\overrightarrow{u_r}+\frac{1}{r}\frac{\partial f}{\partial \theta}\overrightarrow{u_\theta}+\frac{1}{r\sin{\theta}}\frac{\partial f}{\partial \phi}\overrightarrow{u_\phi}$$
\end{itemize}
\end{prop}
\begin{thm}[Théorème de Green-Ostrogradski]

\end{thm}
\section{Le rotationnel $rot$}
\begin{defi}
Soit un champ de vecteurs $\overrightarrow{f}:\mathbb{R}^3\mapsto\mathbb{R}^3$ de classe $\mathscr{C}^1$ sur $\mathscr{D}_f$, de coordonnées\\ $f=(f_1,f_2,f_3)$.\\
Le rotationnel de $\overrightarrow{f}$ de ce champ est défini par :

$$rot(\overrightarrow{f})=\overrightarrow{\nabla}\land\overrightarrow{f}=\begin{pmatrix}\frac{\partial}{\partial x}\\\frac{\partial}{\partial y}\\\frac{\partial}{\partial z}\end{pmatrix}\land\begin{pmatrix}f_1\\f_2\\f_3\end{pmatrix}=\begin{pmatrix}\frac{\partial f_3}{\partial y}-\frac{\partial f_2}{\partial z}\\\frac{\partial f_1}{\partial z}-\frac{\partial f_3}{\partial x}\\\frac{\partial f_2}{\partial x}-\frac{\partial f_1}{\partial y}\end{pmatrix}$$
\end{defi}

\section{Le laplacien $\Delta$}
\begin{defi}
Soit une champ scalaire $f:\mathbb{R}^n\mapsto\mathbb{R}$ de classe $\mathscr{C}^2$ sur $\mathscr{D}_f$.
Le laplacien, noté $\Delta$ $\nabla^2$, est donné par :
$$\Delta \overrightarrow{f}=\nabla . \overrightarrow{\nabla} f = div(\overrightarrow{grad}f)=\sum_{i=1}^n \frac{\partial^2 f}{\partial {x_i}^2}$$
\end{defi}
\section{Équations aux dérivées partielles}