\chapter{Diagonalisation}
  La diagonalisation est le second "problème principal" d'algèbre, le premier étant la résolution de systèmes
  linéaires.
Diagonaliser une matrice revient à la "simplifier".\\
L'intérêt d'un tel procédé est qu'il simplifie certains calculs tel que la multiplication ou l'exponentiation.\\
La diagonalisation consiste à chercher une base $\mathscr{B}$ de l'espace, dans laquelle la matrice $A$ est diagonale.\\
Dans la suite de ce chapitre nous ne considérerons que des matrices carrées.\\
\begin{bclogo}[couleur=blue!30,couleurBord=blue,arrondi=0.1,logo=\bcbook,ombre=true]{Définition}
Une application linéaire $f:\mathbb{R}^{n}\rightarrow\mathbb{R}^{n}$  est dite diagonalisable si et seulement si $\exists \mathscr{B}$ une base de $\mathbb{R}^{n}$ tel que sa matrice représentative $A_{\mathscr{B},\mathscr{B}}(f)$ est diagonale.
\end{bclogo}
\section{Éléments propres}
Il convient dans un premier temps de définir les différents objets qui servirons à la diagonnalisation, ces objets sont appelés éléments propres.
\begin{bclogo}[couleur=blue!30,couleurBord=blue,arrondi=0.1,logo=\bcbook,ombre=true]{Définition}
Soit A une matrice $n\times n$.
\begin{itemize}
    \item[$\bullet$] On dit que $\lambda\in\mathbb{C}$ est une valeur propre de $A$ s'il existe $x\in\mathbb{C}^{n}$ avec $x\neq 0$ tel que $Ax = \lambda x$.
    \item[$\bullet$] On appelle alors le vecteur $x$ le vecteur propre de $A$ associé à la valeur propre $\lambda$.
    \item[$\bullet$] On appelle spectre de $A$ l'ensemble des valeurs propres de $A$.
    \item[$\bullet$] On appelle sous espace propre de $A$ (associé à la valeur propre $\lambda$), noté $E_{\lambda}$, l'ensemble de tous les vecteurs $x$ tel que
$Ax=\lambda x \Leftrightarrow (A-\lambda I_n)x=0$.\\
Autrement dit, $E_{\lambda}=\ker(A-\lambda I_n)=\{x\in E | Ax=\lambda x\}$\\
\end{itemize}
\end{bclogo}
\begin{bclogo}[logo=\bccrayon,noborder=true,barre=snake]{Exemple}
Soit la matrice $A=\begin{pmatrix}
5 & 2 \\
4 & 3
\end{pmatrix}$\\
Par définition d'une valeur propre $\lambda$, nous cherchons un vecteur $x$ tel que $Ax=\lambda x$.
Dans le cas présent, on a :\\
$$\begin{pmatrix} 5 & 2\\ 4 & 3\end{pmatrix}\times\begin{pmatrix}x_1\\x_2\end{pmatrix}=\lambda\times\begin{pmatrix}x_1\\x_2\end{pmatrix}$$\\
Il nous faut donc résoudre le système suivant :\\
$$\begin{cases}5x_1+2x_2=\lambda x_1\\4x_1 + 3x_2=\lambda x_2\end{cases}$$
Sa résolution nous renvoie notamment le vecteur $x=(7,7)$ et il se trouve que $Ax=7\times x$.\\
On a donc que $\lambda = 7$ une valeur propre de la matrice $A$ et $x=(1,1)$ est vecteur un propre de la matrice $A$ associé à la valeur propre $\lambda = 7$.\\
\end{bclogo}
L'utilisation de la définition d'une valeur propre pour son calcul est une opération assez fastidieuse, c'est pour cela que l'on passe par d'autres moyens pour les déterminer.\\
On utilise pour cela le polynôme caractéristique de la matrice $A$.
\section{Polynôme caractéristique et calcul des éléments propres}
Le calcul des éléments propres est plus facile en passant par le polynôme caractéristique.
\begin{bclogo}[couleur=blue!30,couleurBord=blue,arrondi=0.1,logo=\bcbook,ombre=true]{Définition}
Soit une application linéaire $f:\mathbb{R}^n\to\mathbb{R}^n$ et sa matrice représentative dans la base $\mathscr{B}$ $A_{\mathscr{B}}(f)$.
On appelle polynôme caractéristique de l'application $f$, le polynôme défini de la façon suivante :
$$P_{f}(\lambda)=P_{A}(\lambda)=\det(A-\lambda I_{n})=\begin{vmatrix}
a_{11}-\lambda & a_{12} & \hdots & a_{1n}\\
a_{21} & a_{22}-\lambda &&\vdots \\
\vdots & & \ddots & \vdots\\
a_{n1} &\hdots&\hdots& a_{nn}-\lambda
\end{vmatrix}$$
\\
\end{bclogo}

\begin{bclogo}[couleur=green!30,couleurBord=green,logo=\bccle ,ombre=true,arrondi=0.1]{Proposition}
Les valeurs propres $\lambda$ de $A$ sont les racines du polynôme caractéristique.
$$\lambda \text{ est une valeur propre de la matrice } A \Leftrightarrow P_A(\lambda) = 0$$
\end{bclogo}

\begin{bclogo}[logo=\bccrayon,noborder=true,barre=snake]{Exemple}
Soit la matrice $A$ définie comme $A=\begin{pmatrix}
7 & 4 \\
3 & 6 \\
\end{pmatrix}$.\\
On commence par chercher les valeurs propres de $A$.
Par la proposition précédente, on a :\\
$$P_A(\lambda) = | A-\lambda I_3 | =\begin{vmatrix}
7 - \lambda & 4\\
3 & 6- \lambda \\
\end{vmatrix} = (7-\lambda)(6-\lambda) -12$$\\
On cherche donc les racines de $P_A(\lambda)$.
\begin{align*}
   & P_A(\lambda)=0\\
\Leftrightarrow & {\lambda}^{2}-13\lambda +30 =0\\
\Leftrightarrow & {\lambda}_{1}=3 \text{ et } {\lambda}_{2}=10 
\end{align*}

Les valeurs propres de la matrice $A$ sont donc ${\lambda}_{1}=3 \text{ et } {\lambda}_{2}=10$
\end{bclogo}
On constate, assez aisément, que la détermination des valeurs propres à l'aide du polynôme caractéristique est beaucoup plus facile et rapide.
\begin{bclogo}[couleur=red!30,couleurBord=red,ombre=true,arrondi=0.1,logo=\bcoutil]{Propriétés}
Soit la matrice $A$ et son polynôme caractéristique $P_A(\lambda)$.
\begin{itemize}
\item Si l'on injecte $0$ dans le polynôme caractéristique il nous renverra la valeur du déterminant de cette matrice :\
$$P_A(0)=\det(A)$$
\item Le polynôme caractéristique possède $n$ racines dans l'ensemble $\mathbb{C}$
\item Deux matrices semblables ont le même polynôme caractéristique.
\item Le polynôme caractéristique d'une matrice est égale à celui de sa transposée $P_A(\lambda)=P_{A^\text{T}}(\lambda)$.
\end{itemize}
\end{bclogo}

Le polynôme caractéristique nous donne un moyen simple de déterminer les valeurs propres.\\
De ces valeurs propres, on peut déduire le reste des éléments propres de la matrice.\\
\\
Afin de pouvoir continuer sereinement, nous allons introduire les multiplicités algébriques et géométriques, qui seront utiles pour la suite.

\begin{bclogo}[couleur=blue!30,couleurBord=blue,arrondi=0.1,logo=\bcbook,ombre=true]{Définition}
\begin{itemize}
    \item On appelle multiplicité géométrique d'une valeur propre $\lambda$ : la dimension du sous espace propre associé à la valeur propre $\lambda$.
    \item On appelle multiplicité algébrique d'une valeur propre $\lambda$ : la multiplicité de $\lambda$ en tant que racine du polynôme caractéristique.
\end{itemize}
\end{bclogo}
\begin{prop}
Soit $A\in\mathbb{M}^{n\times n}$, la trace de $A$ est égale à la somme des valeurs propres multipliées avec leur multiplicité propre.
$$\text{Tr}(A)=\sum_{i=1}^n \lambda_i\times a_i$$
\end{prop}
Cette propriété permet une première vérification de la validité des éléments propres.
\begin{bclogo}[logo=\bclampe,arrondi=0.1,ombre=true, couleur=yellow!60,couleurBord=yellow]{Méthode : Détermination des éléments propres}
\begin{enumerate}
    \item Déterminer le polynôme caractéristique
    \item Trouver les valeurs propres de $A$, en déduire le spectre de $A$
    \item Rechercher les vecteurs propres associés au valeurs propres $\lambda$
    \item Déterminer les sous espaces propres $E_{\lambda}$ associés aux valeurs propres $\lambda$.\\
    Pour ce faire, il suffit de trouver le noyau de la matrice $A-\lambda I_n$ $\Leftrightarrow \ker(A-\lambda I_n)$.\\
\end{enumerate}
\end{bclogo}
\begin{ex}
On se donne la matrice $A=\begin{pmatrix}3&-1&1\\0&2&0\\1&-1&3\end{pmatrix}$\\
On commence par poser $A-\lambda I_3$ :
$$A-\lambda I_n = \begin{pmatrix}3&-1&1\\0&2&0\\1&-1&3\end{pmatrix}-\begin{pmatrix}\lambda&0&0\\0&\lambda&0\\0&0&\lambda\end{pmatrix}=\begin{pmatrix}3-\lambda&-1&1\\0&2-\lambda&0\\1&-1&3-\lambda\end{pmatrix}=B$$
Le polynôme caractéristique nous est donné par le déterminant de cette nouvelle matrice $B$.
$$P_A(\lambda)=\det(A-\lambda I_3)=\det(B)=\begin{vmatrix}3-\lambda&-1&1\\0&2-\lambda&0\\1&-1&3-\lambda\end{vmatrix}$$
On choisi de développer le déterminant selon la deuxième ligne, en effet celui-ci nous serra plus facile a calculer.  On a donc :
$$P_A(\lambda)=\det(A-\lambda I_3)=(2-\lambda)((3-\lambda)^2-1)$$
On sait que les racines du polynôme caractéristique, sont les valeurs propres de $A$, donc :
\begin{align*}
    \Leftrightarrow & P_A(\lambda) = 0\\
    \Leftrightarrow & (2-\lambda)((3-\lambda)^2-1) = 0\\
    \Leftrightarrow & (2-\lambda)(\lambda^2-6\lambda+8) = 0
\end{align*}
Le premier facteur nous renvoie $\lambda = 2$ et le deuxième facteur nous donne $\lambda_1=2$ et $\lambda_2=4$.\\
On a donc les valeurs propres de $A$ qui sont: $\lambda_1 =2$ valeur propre de multiplicité algébrique $2$, $\lambda_2=4$ valeur propre de $A$ de multiplicité algébrique $1$.\\
On en déduit le spectre de $A$ : $Spec(A)=\{2;4\}$
\\
\\
On souhaite maintenant déterminer les vecteurs propres $x$ associées aux valeurs propres $\lambda$.\\
Pour une valeur propre $\lambda$ donnée, la recherche du vecteur propre associée passe par la résolution de l'égalité $(A-\lambda I_3)x=0$.\\
Pour la valeur propre $\lambda_1 = 2$ on a $(A-2I_3)x=0$.On commence par poser $A-2I_3$ :
$$A-2I_3=\begin{pmatrix}3&-1&1\\0&2&0\\1&-1&3\end{pmatrix}-\begin{pmatrix}2&0&0\\0&2&0\\0&0&2\end{pmatrix}=\begin{pmatrix}1&-1&1\\0&0&0\\1&-1&1\end{pmatrix}=C$$
Il nous faut donc résoudre :
\begin{align*}
    \Leftrightarrow & (A-2I_3)x=0\\
    \Leftrightarrow & Cx=0\\
    \Leftrightarrow & \begin{pmatrix}1&-1&1\\0&0&0\\1&-1&1\end{pmatrix}\times\begin{pmatrix}x_1\\x_2\\x_3\end{pmatrix}=\overrightarrow{0}\\
    \Leftrightarrow & \begin{cases}x_1-x_2+x3=0\\0=0\\x_1-x_2+x_3=0\end{cases}\\
    \Leftrightarrow & x_1-x_2+x_3=0\\
    \Leftrightarrow & x_1=x_2-x_3
    \Rightarrow \begin{pmatrix}x_2-x_3\\x_2\\x_3\end{pmatrix}=x
\end{align*}
On peut décomposer le vecteur :
$$\begin{pmatrix}x_2-x_3\\x_2\\x_3\end{pmatrix}=\begin{pmatrix}x_2\\x_2\\0\end{pmatrix}+\begin{pmatrix}-x_3\\0\\x_3\end{pmatrix}=x_2\times\begin{pmatrix}1\\1\\0\end{pmatrix}+x_2\times\begin{pmatrix}-1\\0\\1\end{pmatrix}$$
Il y a donc deux vecteurs propres $v_1=\begin{pmatrix}1\\1\\0\end{pmatrix}$ et $v_2=\begin{pmatrix}-1\\0\\1\end{pmatrix}$ associées a la valeur propre $\lambda_1=2$.\\
\newline
Pour la valeur propre $\lambda_2=4$ on pose $(A-4I_3)x=0$.
On effectue un raisonnement analogue au précédent :
$$A-4 I_3=\begin{pmatrix}3&-1&1\\0&2&0\\1&-1&-1\end{pmatrix}-\begin{pmatrix}4&0&0\\0&4&0\\0&0&4\end{pmatrix}=\begin{pmatrix}-1&-1&1\\0&-2&0\\1&-1&-1\end{pmatrix}=D$$
On cherche $\ker(A-4I_3)$ :
\begin{align*}
    \Leftrightarrow & \ker(A-4I_3)\\
    \Leftrightarrow & \ker D\\
    \Leftrightarrow & Dx=0\\
    \Leftrightarrow & \begin{pmatrix}-1&-1&1\\0&-2&0\\1&-1&-1\end{pmatrix}\times\begin{pmatrix}x_1\\x_2\\x_3\end{pmatrix}=\begin{pmatrix}0\\0\\0\end{pmatrix}\\
    \Leftrightarrow & \begin{cases}-x_1-x_2+x_3=0\\-2x_2=0\\x_1-x_2-x_3=0\end{cases}\\
    \Leftrightarrow & \begin{cases}-x_1+x_3=0\\ x_2=0\\x_1-x_3\end{cases}\\
    \Leftrightarrow & x_1-x_3=0\\
    \Leftrightarrow & x_1=x_3 \Rightarrow \begin{pmatrix}x_1\\0\\x_1\end{pmatrix}
\end{align*}
Ici il y a un vecteur propre associé à la valeur propre $\lambda_2=4$ qui est $v_1=\begin{pmatrix}1\\0\\1\end{pmatrix}$.\\
\newline 
Et enfin, a partir des vecteurs propres on déduit les sous espaces propres, ici :
$$E_{\lambda_1}=\text{Vect}\begin{pmatrix}\begin{pmatrix}1\\1\\0\end{pmatrix},\begin{pmatrix}-1\\0\\1\end{pmatrix}\end{pmatrix}$$

$$E_{\lambda_2}=\text{Vect}\begin{pmatrix}\begin{pmatrix}1\\0\\1\end{pmatrix}\end{pmatrix}$$
\end{ex}


\section{Diagonalisation}
Revenons sur la définition d'une matrice diagonalisable, pour en donner une définition plus mathématique.
\begin{bclogo}[couleur=blue!30,couleurBord=blue,arrondi=0.1,logo=\bcbook,ombre=true]{Définition}
La matrice $A\in\mathbb{M}^{n\times n}$ est diagonalisable, s'il existe une matrice diagonale $D$ et une matrice $P$ inversible telles que $A=PDP^{-1}$, où $P$ est la matrice de passage entre la base canonique et la base $\mathscr{B}$ où la matrice $D$ existe.
\end{bclogo}
Une propriété qui viens directement a l'esprit est que les matrices $A$ et $D$ sont semblables (par définition).\\
Tout l'enjeu sera donc de déterminer la matrice de passage $P$.\\
Par ailleurs, il peut être intéressant de savoir si une matrice est diagonalisable. 
\begin{thm}[Théorème: Critère de diagonalisation]
Soit $A\in\mathbb{M}^{n\times n}(\mathbb{R})$.
$\text{La matrice}\ A\ \text{est diagonalisable}\ \Leftrightarrow\forall i=1,\hdots,k\ \text{on a}\ \text{dim}(E_{\lambda_i})=\text{multiplicité algébrique de}\ \lambda_i$
\end{thm}
\begin{thm}[Corollaire]
Si $P_A(\lambda)$ admet $n$ racines réelles distinctes $\Leftrightarrow$ $A$ est diagonalisable
\end{thm}
Une fois le critère de diagonalisation ou son corollaire vérifié on peut déduire la matrice diagonale $D$ ainsi que le la matrice de passage $P$.
\begin{prop}
Soit la matrice $A\in\mathbb{M}^{n\times n}$ et ses valeurs propres $\lambda_1,\hdots, \lambda_k$ et leurs multiplicités $a_1,\hdots,a_k$ associées.
La matrice diagonale $D$ est donné par :\\
$$\begin{pmatrix}
\lambda_1 & 0 & \hdots & \hdots & \hdots & \hdots & \hdots & \hdots & \hdots & 0\\
0 & \ddots & 0 & & & & & & & \vdots\\
\vdots & 0 & \lambda_1 & 0 & & & & & & \vdots\\
\vdots & & 0 & \lambda_2 & 0 & & & & & \vdots \\
\vdots & & & 0 & \ddots & 0 & & & & \vdots\\
\vdots & & & & 0 & \lambda_2 & 0 & & & \vdots\\
\vdots & & & & & 0 & \ddots & 0 & & \vdots\\
\vdots & & & & & & 0 & \lambda_k & 0 & \vdots\\
\vdots & & & & & & & 0 & \ddots & 0 \\
0 & \hdots & \hdots & \hdots & \hdots & \hdots & \hdots & \hdots & 0 & \lambda_k
\end{pmatrix}$$
où les valeurs propres $\lambda_i$ se répète autant fois que la valeur de leur multiplicité $a_i$.
\end{prop}
Nous avons la matrice diagonale, maintenant il nous faut trouver la matrice de passage $P$.
\begin{prop}
Soit la matrice $A\in\mathbb{M}^{n\times n}$ et ses vecteurs propres $x_1,\hdots,x_n$.
La matrice de passage entre la base canonique $\mathscr{B}_c$ et la base où la matrice diagonale se trouve $\mathscr{B}_D$, est la matrice composée de tous les vecteurs propres de $A$.
$$P_{\mathscr{B}_c,\mathscr{B}_D}=\begin{pmatrix}x_1\ |& \hdots\ | & x_n\end{pmatrix}$$
\end{prop}

L'ordre des valeurs propres n'influe pas sur le résultat.\\
Il est néanmoins nécessaire de bien faire attention a garder l'ordre choisi sinon les calculs s'en trouveront erroné.
\begin{meth}[Diagonalisation d'une matrice]
\begin{enumerate}
    \item Déterminer le polynôme caractéristique
    \item Trouver les valeurs propres de $A$.
    \item Factoriser le polynôme caractéristique.
    \item Rechercher les sous espaces propres et leurs dimensions (multiplicité géométrique)
    \item Déduire la matrice $P$ et $P^{-1}$
\end{enumerate}
\end{meth}
\begin{ex}
frv
\end{ex}
La diagonalisation est une notion qui nous aide a déterminer des extremums de fonctions (\textit{Cf.Chapitre 8 : Extremums en plusieurs variables}), permet d'analyser des relations de récurrences ou encore permet de résoudre des systèmes différentiels linéaires (\textit{Cf.Mat-307,Partie 2,Chapitre 2 : Systèmes différentiels}).
\begin{meth}[]
\begin{enumerate}
    \item Puissance d'une matrice :
    \begin{enumerate}
        \item Diagonaliser la matrice $A$
        \item Calculer la puissance de la matrice diagonale $D$ $$D^n=\begin{pmatrix}\lambda_1^n&\hdots&0\\\vdots&\ddots&\vdots\\0&\hdots&\lambda_k^n\end{pmatrix}$$
        \item Utiliser les matrices $P$ et $P^{-1}$ pour repasser dans la base canonique : $$A=PDP^{-1}$$
    \end{enumerate}
    \item Étude d'une suite :
\end{enumerate}
\end{meth}
\begin{ex}
\begin{enumerate}
    \item Étude d'une suite :\\
    On considère la suite de Fibonnacci définie de manière récurrente comme il suis : 
$$u_{n+1}=u_{n}+u_{n-1}\ \text{avec}\ u_0=u_1=1$$
On cherche a exprimer le terme générale de la suite (forme explicite).\\
Pour ce faire on pose $\begin{pmatrix}u_{n+1}\\u_n\end{pmatrix}$.
$$\begin{cases}u_{n+1}=u_n+u_{n-1}\\
u_n = 
\end{cases}$$ Bla bla...\\
On a donc : $x_n=Ax_{n-1}\ \text{avec}\ A=\begin{pmatrix}1&1\\1&0\end{pmatrix},\ x_0=\begin{pmatrix}1\\1\end{pmatrix}$.\\
Calculer le n-ième terme de la suite $x_n=Ax_{n-1}$ revient à calculer $x^n=Ax$ ???
On diagonalise la matrice $A$ :\\
On commence par rechercher les éléments propres de la matrice.
On cherche le polynôme caractéristique :
$$P_A(\lambda)=|A-\lambda I_2|=\begin{vmatrix}1-\lambda&1\\1&-\lambda\end{vmatrix}=-\lambda(1-\lambda)-1=\lambda^2-\lambda-1$$
On résoud $P_A(\lambda)=0$, on obtient les valeurs propres $\lambda_1=\frac{1+\sqrt{5}}{2}$ et $\lambda_2=\frac{1-\sqrt{5}}{2}$.\\
On déduit la matrice diagonale :
$$D=\begin{pmatrix}\frac{1+\sqrt{5}}{2}&0\\0&\frac{1-\sqrt{5}}{2}\end{pmatrix}$$
    \item Puissance d'une matrice :\\
On se donne la matrice $A=\begin{pmatrix}\end{pmatrix}$.
On cherche a calculer $A^3$.\\
On commence par diagonaliser la matrice.\\
\end{enumerate}
\end{ex}
\section{Matrices symétriques et formes quadratiques}
\subsection{Éléments propres et diagonalisation d'une matrice symétrique}
On rappel la définition d'une matrice symétrique et d'une matrice orthogonale.
\begin{defi}
Soit $A\in\mathbb{M}^{n\times n}(\mathbb{R})$.\
$$A\text{ est symétrique }\Leftrightarrow\ A=A^{\text{T}}$$
$$A\text{ est orthogonale }\Leftrightarrow\ AA^{\text{T}}=I_n\ \Leftrightarrow\ A^{-1}=A^{\text{T}}$$
\end{defi}
\begin{ex}
On considère les matrices $A=\begin{pmatrix}1&2&3\\2&4&5\\3&5&6\end{pmatrix}$ et $B=\begin{pmatrix}0&0&1\\1&0&0\\0&1&0\end{pmatrix}$.\\
La matrice $A$ est une matrice symétrique car $A=A^{\text{T}}$.\\
La matrice $B$ est une matrice orthogonale car :
$$BB^{\text{T}}=\begin{pmatrix}0&0&1\\1&0&0\\0&1&0\end{pmatrix}\begin{pmatrix}0&1&0\\0&0&1\\1&0&0\end{pmatrix}=\begin{pmatrix}1&0&0\\0&1&0\\0&0&1\end{pmatrix}=I_3$$
\end{ex}
\begin{thm}[Théorème des axes principaux]
Toute matrice symétrique $A$ peut être diagonalisée à l'aide d'une matrice orthogonale : il existe $\Gamma\in\mathbb{M}^{n\times n}(\mathbb{R})$ orthogonale et une matrice diagonale réelle $\Lambda$ telles que :
$$A=\Gamma\Lambda\Gamma^{-1}=\Gamma\Lambda\Gamma^{T}$$
\end{thm}
\begin{ex}
On se donne la matrice $A=\begin{pmatrix}1&2&0\\2&4&0\\0&0&5\end{pmatrix}$.
\end{ex}
\subsection{Formes quadratiques}
\begin{defi}
Soit $q\ :\ \mathbb{R}^n\to\mathbb{R},\ q(x)=\sum_{i=1}^n\sum_{j=1}^n a_{ij}x_i x_j$.\\
On dit que $q$ est une forme quadratique sur $\mathbb{R}^n$
\end{defi}
\begin{defi}
On dit que la forme $q(x)$ est :
\begin{itemize}
    \item semi-définie positive si $q(x)\ge 0\ \forall x\in\mathbb{R}^n$.
    \item semi-définie négative si $q(x)\leq 0\ \forall x\in\mathbb{R}^n$.
    \item définie positive si $q(x)> 0\ \forall x\in\mathbb{R}^n$.
    \item définie négative si $q(x)< 0\ \forall x\in\mathbb{R}^n$.
    \item non définie si $q(x)$ change de signe.
\end{itemize}
\end{defi}
\begin{thm}[Proposition]
La forme quadratique $q$ est semi-définie (définie) positive si et seulement si toutes les valeurs propres de la matrice $A$ de la forme sont positives.
\end{thm}
\begin{bclogo}[couleur=green!30,couleurBord=green,logo=\bccle ,ombre=true,arrondi=0.1]{Critère de Silvester}
Une matrice $A$ est semi-définie (définie) positive si et seulement si tous ses mineurs principales sont positifs, c'est à  dire :
$$a_{11}\ge 0, \begin{vmatrix} a_{11}&a_{12}\\ a_{21} & a_{22}\end{vmatrix}\ge 0,\begin{vmatrix}a_{11} & a_{12} & a_{13}\\ a_{21} & a_{22} & a_{23}\\ a_{31} & a_{32} & a_{33}\end{vmatrix}\ge 0, ...$$
Une matrice $A$ est semi-définie (définie) négative si et seulement si les signes des mineurs principaux de $A$ alternent :
$$a_{11}\leq 0, \begin{vmatrix} a_{11}&a_{12}\\ a_{21} & a_{22}\end{vmatrix}\leq 0,\begin{vmatrix}a_{11} & a_{12} & a_{13}\\ a_{21} & a_{22} & a_{23}\\ a_{31} & a_{32} & a_{33}\end{vmatrix}\leq 0, ...$$
Avec les $\leq$ ou $\ge$ sont respectivement remplacés pas $<$ et $>$ dans le cas des matrices définies
\end{bclogo}
\begin{ex}
fnefbo
\end{ex}