\chapter{Calcul matriciel}
\begin{defi}
Une matrice $n\times m$ à  coefficients réels est un tableau de nombres réels de $n$ lignes et $m$ colonnes.\\
On note $a_{ij}$ le coefficient à la i-ème  et à la j-ème colonne.\\
On représente une matrice de la manière suivante :
$$A = \begin{pmatrix}a_{11} & a_{12} & \hdots & a_{1m}\\a_{21} & a_{22} & \hdots & a_{2m}\\ \vdots & \vdots & \ddots & \vdots\\ a_{n1} & a_{n2} & \hdots & a_{nm} \end{pmatrix}$$
\newline
Une matrice $n\times m$ à valeurs réelles appartient à l'ensemble des matrices $n\times m$, noté $\mathbb{M}^{n\times m}(\mathbb{R})$.
\end{defi}
Toute application linéaire ou système d'équation linéaire peut être écrit sous forme matricielle.
\begin{ex}
ex de base
On considère l'application
\end{ex}
\section{Calcul matriciel}
On commence par définir l'addition de deux matrices.
\begin{defi}
Soient deux matrices $A$ et $B$ de dimension $n\times m$.\\
L'addition de ces deux matrices est donnée par :
$$\begin{pmatrix}a_{11} & a_{12} & \hdots & a_{1m}\\a_{21} & a_{22} & \hdots & a_{2m}\\ \vdots & \vdots & \ddots & \vdots\\ a_{n1} & a_{n2} & \hdots & a_{nm} \end{pmatrix}+\begin{pmatrix}b_{11} & b_{12} & \hdots & b_{1m}\\b_{21} & b_{22} & \hdots & b_{2m}\\ \vdots & \vdots & \ddots & \vdots\\ b_{n1} & b_{n2} & \hdots & b_{nm} \end{pmatrix}=\begin{pmatrix}a_{11}+b_{11} & a_{12}+b_{12} & \hdots & a_{1m}+b_{1m}\\a_{21}+b_{21} & a_{22}+b_{22} & \hdots & a_{2m}+b_{2m}\\ \vdots & \vdots & \ddots & \vdots\\ a_{n1}+b_{n1} & a_{n2}+b_{n2} & \hdots & a_{nm}+b_{nm} \end{pmatrix}$$
\end{defi}
\begin{defi}
Soient deux matrices $A$ de dimension $n\times m$ et $B$ de dimension $m\times p$.\\
La multiplication de ces deux matrices nous donnera une matrice $C$ de dimension $n\times p$.\\
Le calcul des coefficients de la matrice $C$ est donné par la formule :
$$c_{ij}=\sum_{k=1}^m a_{ik}\times b_{kj}=a_{i1}b_{1j}+a_{i2}b_{2j}+\hdots+a_{in}b_{nj}$$
\end{defi}
Attention, la multiplication matricielle n'est pas commutative
\begin{ex}
On considère les matrices $A=$ et $B=$
\end{ex}
\begin{defi}
Soit une matrice $A$ de dimension $n\times m$.\\
L'opération de transposition de cette matrice nous renverra la matrice transposée de $A$, $B$ de dimension $m\times n$.\\
Cet opération est défini comme il suit :
$$A^\text{T}$$
\end{defi}
\begin{defi}
Soit une matrice $A$ de dimension $n\times m$.\\
On appelle trace de la matrice $A$ la somme de tous ses termes diagonaux.
$$\text{Tr}(A)=\sum_{i=1}^n a_{ii}$$
\end{defi}
\begin{prop}
Soient $A$ et $B$ de matrice et $\alpha\in\mathbb{R}$.\\
\begin{itemize}
    \item $\text{Tr}(A+B)=\text{Tr}(A)+\text{Tr}(B)$
    \item $\text{Tr}(\alpha A)=\alpha\text{Tr}(A)$
    \item $\text{Tr}(A^\text{T})=\text{Tr}(A)$
    \item Si $A$ et $B$ sont multipliable : $\text{Tr}(AB)=\text{Tr}(BA)$
    \item Si $A$ et $B$ sont semblables : $\text{Tr}(A)=\text{Tr}(B)$
\end{itemize}
\end{prop}
\subsection{Multiplication matricielle}
\subsection{Transposition matricielle}
\subsection{Propriétés et caractéristiques de matrices}
\section{Matrices particulières}
\subsection{Matrice carrée et rectangulaire}
\subsection{Matrice élémentaire et algorithme de Gaus-Jordan}
\subsection{Matrices de passage}
\subsection{Matrices triangulaires}
\section{Déterminants et inversion de matrices}
\subsection{Comatrice et matrice adjointe}
