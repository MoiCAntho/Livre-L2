\chapter{Transformations thermodynamiques}
\section{Description d'un système thermodynamique}
\subsection{Systèmes thermodynamiques}
Afin d'amorcer une étude thermodynamique (tout comme n'importe quel domaine de la physique), il faut définir le système étudié.
\begin{defi}
On appelle système thermodynamique, l'ensemble des corps étudiés contenus dans un volume délimité par une enveloppe, réelle ou fictive.\\
On distingue donc un milieu intérieur (le système) et un milieu extérieur.
Selon les échanges que peut avoir le système avec un milieu extérieur on peut le qualifier de différents adjectifs.
\newline
\begin{itemize}
    \item On dit que le système est ouvert, s'il peut échanger avec le milieu extérieur de la matière et de l'énergie.
    \item On dit que le système est fermé, s'il peut échanger que de l'énergie avec le milieu extérieur.
    \item On dit que le système est isolé, s'il ne peut échanger ni énergie ni matière avec le milieu extérieur.
\end{itemize}
\end{defi}
\begin{ex}
\begin{itemize}
    \item Un verre d'eau est un système ouvert, il peut échanger de l'énergie et de la matière avec le milieu extérieur (évaporation, liquéfaction).
    \item Le circuit de refroidissement d'un réfrigérateur est un système fermé, il ne peut pas échanger de matière (liquide en circuit fermé) mais il peut échanger de l'énergie avec le milieu extérieur.
    \item L'univers est considéré comme un système isolé (il n'est pas censé avoir de milieu extérieur donc aucun échange n'est possible).
\end{itemize}
\end{ex}
\subsection{Grandeurs thermodynamiques et variables d'états}
\begin{defi}
Une variable d'état est une grandeur physique (mesurable) caractérisant l'état d'un système.
\end{defi}
On se sert de tels variables pour l'établissement d'équations d'état.\\
\newline
Ces variables peuvent-être qualifier :
\begin{itemize}
    \item d'extensive, c'est à dire une grandeur qui dépend de la "taille".\\
    Pour le dire autrement, une grandeur est extensive si pour deux systèmes disjoints, leur réunion est la somme de ces grandeurs.
    \item d'intensive, c'est a dire une grandeur qui peut être mesuré de manière ponctuelle, elle ne dépend pas de la "taille" du système. Ces grandeurs ne sont pas additives.
\end{itemize}
\begin{ex}
Le volume $V$, la masse $m$ et la quantité de matière $n$ sont des grandeurs extensives.\\
La température $T$ ou $\theta$ et la pression $P$ sont des grandeurs intensives
\end{ex}
A partir de ces grandeurs on définit l'équilibre d'un système thermodynamique.
\begin{defi}
Un système est à l'équilibre thermodynamique si toutes ses variables d'état sont invariantes dans le temps (constantes) sans transfert de matière ou d'énergie.
\end{defi}
\subsubsection{Pression}
\subsubsection{Température}
\subsection{Équations d'états}
\begin{defi}
On appelle équation d'état une relation mathématique entre les différentes variables d'état caractérisant le système.
\end{defi}
\subsection{Coefficients thermoélastiques}
\begin{defi}
    On définit trois grandeurs intensives nommées coefficients thermoélastiques.\\
    \begin{itemize}
        \item Coefficient de dilatation isobare : $\alpha = \frac{1}{V}\left(\frac{\partial V}{\partial T}\right)_P$
        \item Coefficient de compression isochore : $\beta = $
        \item Coefficient de compressibilité isotherme : $\chi_T = $
    \end{itemize} 
\end{defi}
L'intéret de tels coefficients est qu'il facilement accessible de manière expérimentale,
et permettent d'accéder très facilement a une équation d'état de n'importe quel matériau.
\section{Transformations}
\section{Représentations graphique}
\chapter{Premier principe}
\chapter{Second principe}
\chapter{Machines thermiques}
\chapter{Transitions de phase des corps purs}